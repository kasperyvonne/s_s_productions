\begin{frame}{Phänomenologie}
\textbf{Supraleitung}: Ab einer Sprungtemperatur $T_{\mathup{C}}$ fällt elektrischer Widerstand auf $\SI{0}{\ohm}$ \\
\begin{columns}
\begin{column}{0.49\textwidth}
  \begin{figure}
    \fbox{\includegraphics[width = \textwidth]{quecksilber_supra.pdf}}
    \label{fig: hg_supraleitung}
  \end{figure}
\end{column}
\begin{column}{0.49\textwidth}

\begin{table}
  \caption{Sprungtemperaturen \cite{dem2}}
  \label{tab: sprungtemperaturen}
\begin{tabular}{l S}
  & $T_{\mathup{C}} / \si{\kelvin}$ \\
  \text{Queksilber} & \num{4.15} \\
    \text{Aluminium} & \num{1.17} \\
      \text{\ce{TlCaBaCuO}} & \num{125}
\end{tabular}
\end{table}
\begin{itemize}
  \item Hochtemperatur-Supraleiter für Versuch geeignet
\end{itemize}

\end{column}
\end{columns}
\end{frame}



\begin{frame}{Meißner-Ochsenfeld-Effekt}
\begin{columns}
\begin{column}{0.49\textwidth}
  \begin{figure}
    \includegraphics[width = \textwidth]{supra_1.pdf}
    \caption{Normale Leitung $T > T_{C}$}
    \label{fig: bfeld_normale_leitung}
  \end{figure}
\end{column}
\begin{column}{0.49\textwidth}
  \begin{figure}
    \includegraphics[width = \textwidth]{supra_2.pdf}
    \caption{Supraleitung $T < T_{C}$}
    \label{fig: bfeld_supraleitung}
  \end{figure}
\end{column}
\end{columns}
\end{frame}

\begin{frame}{Quantitative Beschreibung}
Londongleichungen ersetzen das ohmsche Gesetz
\begin{equation}
  \vec{\jmath} = \sigma \vec{E}\quad  \longrightarrow \quad
  \begin{cases}
    \frac{\partial }{\partial t}\vec{\jmath} = \frac{nq^2}{m}\vec{E} \\
    \nabla \times \vec{\jmath} = - \frac{n q^2}{m} \vec{B}
  \end{cases}
\end{equation}
\pause
Mit Maxwell Gleichung $\nabla \times \vec{B} = \mu_0 \vec{\jmath}$
\begin{equation}
-\nabla \times (\nabla \times \vec{B}) = -\nabla\underbrace{(\nabla \cdot \vec{B})}_{= 0} + \Delta \vec{B} = \frac{\mu_0 n q^2}{m}\vec{B}
\end{equation}
\end{frame}

\begin{frame}{Quantitative Beschreibung}
Nun $\vec{B} = B_0 \vec{e}_z$, Grenzfläche $x$-$y$-Ebene
\begin{equation}
  \text{Lösung:}\quad \vec{B}_z = B_0 \exp[- \sqrt{\frac{\mu_0 n q^2}{m}}   x ]
\end{equation}
\end{frame}
