\begin{table}
\centering
\caption{Gemessene Drücke bei der Leckkratenmethode für die Drehschieberpumpe mit $p_{\mathrm{l}}=\SI{0.3}{\micro\bar}$. Messung bei Raumtemperatur.}
\label{tab: leck_turbo_leck_0.3.pdf}
\begin{tabular}{S[table-format=1.1]@{${}\pm{}$} S[table-format=1.1] S S S S[table-format=1.1]@{${}\pm{}$} S[table-format=1.1] }
\toprule
\multicolumn{2}{c}{$p \:/\: \si{\micro\bar}$} & {$t_1 / \si{ \second}$} & {$t_2 / \si{ \second}$} & {$t_3 / \si{ \second}$} & \multicolumn{2}{c}{$\overline{t} \:/\: \si{ \second}$} \\
\midrule
0.3 & 0.1 & 0.0 & 0.0 & 0.0 & 0.0 & 0.0\\
0.8 & 0.3 & 0.7 & 0.6 & 0.8 & 0.7 & 0.0\\
2.0 & 0.8 & 2.7 & 2.7 & 3.3 & 2.9 & 0.2\\
3.0 & 1.2 & 4.3 & 4.2 & 4.7 & 4.4 & 0.2\\
4.0 & 1.6 & 5.7 & 5.7 & 6.1 & 5.9 & 0.1\\
5.0 & 2.0 & 7.0 & 7.0 & 7.4 & 7.2 & 0.1\\
6.0 & 2.4 & 8.3 & 8.2 & 8.7 & 8.4 & 0.1\\
7.0 & 2.8 & 9.2 & 9.3 & 9.8 & 9.4 & 0.2\\
8.0 & 3.2 & 10.4 & 10.4 & 11.0 & 10.6 & 0.2\\
9.0 & 3.6 & 11.5 & 11.5 & 12.0 & 11.7 & 0.2\\
\bottomrule
\end{tabular}
\end{table}
