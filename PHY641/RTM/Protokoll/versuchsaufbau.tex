\section{Versuchsaufbau/-durchführung}

\subsection{Versuchsaufbau}
Der grundlegende Aufbau ist in der Abbildung \ref{} abgebildet.
In der Abbildung ist einmal die Spitzenhalterung zu sehen und
einmal der Carrier mit Probe.
Die Spitze kann mittels Piezokeramiken filigran bewegt werden, um
so ein ab rastern über die Probe zu ermöglichen.

Beim abfahren der Probe, gibt es zwei Methoden die Struktur der Probe zu erfassen.
Zum einen ist es möglich den Abstand zwischen Probe und Spitze konstant zu lassen
und die Veränderung des Tunnelstroms zu messen. Mit dieser Methode kannes aber passieren,
auf Grund des konstanten Abstandes, das die Spitze die Probe rammt und so beschädigt wird.
Umgekehrt ist es möglich die den Tunnelstrom zu lassen und den Abstand zu variieren.
