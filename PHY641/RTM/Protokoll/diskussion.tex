\section{Diskussion}
Die gefundenen Ergebnisse zeige, dass der Versuch durch systematische Fehler beeinflusst ist.
Die Gittervektoren der HOPG Struktur weichen sowohl in Betrag als auch Innenwinkel
deutlich von den Literaturwerten ab. Es ist jedoch anzumerken, dass durch die simple Korrketur der Vektoren
mit einer Diagonalmatrix \eqref{eq: korrekturmatrix} der erwartete Innenwinkel von $\SI{60}{\degree}$ mit dem so
gefunden Wert $\SI{66.4(6)}{\degree}$ noch relativ gut übereinstimmt. HOPG ist damit ein hervorragender Stoff
zur Kallibrierung des Rastertunnelmikroskops.
Die starken Verzerrungen der Bilder wie in Abbildung \ref{} sind auf thermische Drifts und andere Fehlerquellen
zurückzuführen. Da über die gesamte Messzeit keine merkliche Verbesserung in der sichtbaren Differenz zwischen
up und down Bilder festzustellen war, ist auf einen defekten Aufbau (etwa in der Befestigung der Spitze) zu schließen.

Der Gitterabstand von Gold beträgt gemäß \cite{} $\SI{4.08}{\angstrom}$. Die gefundene Plateaudifferenz
liegt mit $\SI{4.75(9)}{\angstrom}$ in einer nahen Umgebung dieses Wertes, was darauf hinweist,
dass die in \ref{fig: au} erkennbaren Kanten durch Höhendifferenzen einer Atomlage hervorgerufen werden.
Das Ergebnis ist jedoch in anbetracht der Tatsache,
dass die in \ref{fig: höhenprofil} eingezeichnete Kante nicht wirklich ausgeprägt ist
und unter Berücksichtigung der genannten Fehlerquellen nicht signifikant.
