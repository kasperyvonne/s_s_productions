\setcounter{page}{1}
\section*{Zielsetzung}
In dem Versuch V27 soll der \emph{Zeeman-Effekt} experiementell beobachtet werden.
Hierzu wird die Aufspaltung der Energieniveaus einer Cadmium Spektrallampe
unter Einwirkung eines äußeren Magnetfeldes untersucht.

\section{Theorie}

\subsection{Magnetisches Moment und Energieaufspaltung}
\subsubsection{Magnetische Momente}
Aus der Quantenmechanik geht hervor das Elektronen neben ihrem Drehimpuls $\vec{l}$ auch einen
Eigendrehimpuls oder auch Spin gennant $\vec{s}$ besitzen. Charakterisiert werden diese beiden Größen durch die
Quantenzahlen $l$ und $s$ mit den aus den Eigenwertgleichung resultierenden Werten:
\begin{align*}
\be{\vec{l}}&=\sqrt{l(l+1)}\hbar & l&=0,1,2,\dots,n-1\\
\be{\vec{s}}&=\sqrt{s(s+1)}\hbar & s&=\frac{1}{2}.
\end{align*}
Auf Grund der Ladung des Elektrons $\map{e}_0$ entstehen magnetische Momente.
Die aus der Theorie resultierenden Momente sind proptotional zum Bohrschen Magneton $\map{\mu}_B$, welches durch
\begin{equation}
  \label{eq:bohrsche_magneton}
  \map{\mu_B}:=-\frac{1}{2}\map{e}\ua{0} \frac{\hbar}{\map{m}\ua{0}}
\end{equation}
definiert ist. Die in Gleichung \eqref{eq:bohrsche_magneton} auftretende Größe $\map{m}_0$ ist die Elektronenmasse.
Für die magnetischen Momente ergibt sich:
\begin{align*}
  \vec{\mu}\ua{l}&=-\map{\mu_B}\frac{\vec{l}}{\hbar}=-\map{\mu_B}\sqrt{l(l+1)}\vec{e}\ua{\vec{l}} \\
  \vec{\mu}\ua{s}&=-\map{g_S}\map{\mu_B}\frac{\vec{s}}{\hbar}=-\map{g_S}\map{\mu_B}\sqrt{s(s+1)}\vec{e}\ua{\vec{s}}
\end{align*}
Mit $\map{g_S}$ wird der Landé-Faktor des Elektrons bezeichnet und hat den Wert $\map{g_S}\approx 2$.
Es fällt auf, dass das magnetische Spinmoment $\vec{\mu}\ua{s}$ des Elektrons in etwa doppelt
so groß ist wie das magentische Bahnmoment $\vec{\mu}\ua{l}$.

\subsubsection{Wechselwirkung der Magnetischen Momente}

Bei dem Übergang zu Mehrelektronatomen müssen die Wechselwirkungen der Momente und Drehimpulse untereinander und
deren der einzelnen Elektronen untereinander unterschieden werden. Die Wechselwirkungen hängen im wesentlichen
von der Kernladungszahl $Z$ ab.
Bei Atomen mit einer niedrigen Kernladungszahl ist die Wechselwirkung der Bahndrehimpulse $\vec{l}_i$ so groß das diese zu einem
Geasamtbahndrehimpuls $\vec{L}$ zusammengefasst werden können. Dieser wird definiert durch
\begin{equation*}
  \vec{L}=\sum_{i=1}^{Z} \vec{l}_i \qquad \be{\vec{L}}=\sqrt{L(L+1)}\hbar.
\end{equation*}
Genauso gilt für den Gesamtspin
\begin{equation*}
  \vec{S}=\sum_{i=1}^{Z} \vec{s}_i \qquad \be{\vec{S}}=\sqrt{S(S+1)}\hbar.
\end{equation*}
Da ein einzelner Spin $\vec{s}_i$ halbzahlig ist, kann auch $\vec{S}$ halbzahlige Werte annehmen.
Die jeweiligen Impulse legen wiederum auch ein magnetische Momente fest:
\begin{align*}
  \be{\vec{\mu}}\ua{L}&=\map{\mu_B}\sqrt{L(L+1)}\\
  \be{\vec{\mu}}\ua{S}&=\map{g_S}\map{\mu_B}\sqrt{S(S+1)}.
\end{align*}

Unter der Prämisse, dass sich die Atome in keinen großen Magnetfeld befinden ist es möglich einen Gesamtdrehimpuls $\vec{J}$ zu
definieren
\begin{align*}
  \vec{J}&=\vec{L}+\vec{S}\\
\shortintertext{mit}
\be{\vec{J}}&=\sqrt{J(J+1)}\hbar.
\end{align*}
In diesem Fall wird von \emph{LS-Kopplung} gesprochen.
Neben der LS-Kopplung gibt es noch die \emph{jj-Kopplung}, diese wird bei Atomen mit hoher Kernladungszahlen beobachtet.
Bei der jj-Kopplung ist die Wechselwirkung zwischen Spin und Bahndrehimpuls des Elektrons groß gegenüber der Wechselwirkung der $\vec{l}_i$ und
$\vec{s}_i$. Auf Grund dessen wird die Größe
\begin{equation*}
  \vec{j}_i=  \vec{l}_i+  \vec{s}_i
\end{equation*}
eingeführt, somit sind $\vec{L}$ und $\vec{S}$ nicht mehr definiert.
Stattdessen exestiert
\begin{equation*}
  \vec{J}=\sum_{i=1}^{Z} \vec{j}_i
\end{equation*}
der Gesamtdrehimpuls der Hülle.
Zwischen der LS-Kopplung und der jj-Kopplung gibt es einen kontinuirlichen Übergang.

\subsubsection{Energieaufspaltung}
Bei der Berücksichtigung der LS-Kopplung kann ein magnetisches Moment $\vec{\mu}$ für den Gesamtdrehimpuls
\begin{equation*}
  \vec{\mu}=\vec{\mu}\ua{L} + \vec{\mu}\ua{S}.
\end{equation*}
Die Richtung von $\vec{J}$ und $\vec{\mu}$ fallen im allgemeinen nicht zusammen. Zusätzlich ist
$\vec{mu}$ auf Grund des Lande Faktors betragsmäßig größer als $\vec{J}$, daher präzediert das magnetische Moment um den Gesamtdrehimpuls.
Wird über die Präzessionsbewegung zeitlich gemittelt, so fällt auf das die $\vec{\mu}_{\perp}$ verschwindet.
Um den Lande-Faktor zu berechnen kann die Formel,
\begin{equation}
  \label{eq:lande_faktor}
  \map{g_J}=\frac{3J(J+1)+S(S+1)-L(L+1)}{2J(J+1)}
\end{equation}
ohne Angabe eines Beweis, verwendet werden.
Aus der Quantenmechanik folgt weiter das nur solche Winkel zwischen $\vec{\mu}$ und
$\vec{B}$ erlaubt sind, bei denen die Komponenten von $\mu\ua{J_z}$ in Feldrichtung ein ganzzahliges Vielfaches von
$\map{g_J}\map{\mu_B}$ sind
\begin{equation}
  \label{eq:quantelung_mu}
  \mu\ua{J_z}=-m\map{g_J}\map{\mu_B}.
\end{equation}
Die ganzzahligkeit wird in der Gleichung \eqref{eq:quantelung_mu} durch die Quantenzahl
$m\in\left{-J,-J+1,\dots,1,\dotsJ\right}$ sichergestellt.
Mit der Definiton von $\mu\ua{J}$ ist es nun Möglich die Energie die vom Magnetfeld
übertragen wird
\begin{equation}
  \label{eq:energie_mag}
  E\ua{mag}=-\vec{\mu}\ua{J}\vec{\vec{B}}=m\map{g_J}\map{\mu_B}\vec{B}.
\end{equation}
Das bedeutet das sich die Energieniveaus $E_0$ eines sich im Magnetfeld befindlichen Atoms
in $2J+1$ äquidistante Niveaus aufspaltet.

%%% Hier ein Bild






\subsection{Auswahlregel und Polarisation}
