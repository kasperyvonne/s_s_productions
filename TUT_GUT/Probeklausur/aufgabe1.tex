\section{Kiste auf schiefer Ebene}
\begin{figure}
   \centering
   \includegraphics[width=0.65\textwidth]{aufg1.pdf}
   \label{fig: aufg1}
 \end{figure}
Eine Kiste der Masse $m_1 = 10 \si{\kilo\gram}$ befindet sich auf einer schiefen Ebene mit dem Winkel $\Theta = 36.9\si{\degree}$.
Sie ist über eine Umlenkrolle mit einem Gegengewicht der Masse $m_2$ verbunden. Der Haftreibungskoeffizient sei $\mu_H = 0.4$ und
der Gleitreibungskoeffizient sei $\mu_G = 0.3$.
  \begin{enumerate}[label=\roman*]
    \item Für welche Werte der Masse $m_2$ ist das System in Ruhe? \\
    \emph{Hinweis}: Bevor du rechnest, mache dir klar, dass die Massen in zwei Richtungen rutschen können. Untersuche also beide Fälle und
    beachte, wie jeweils die Haftreibungskraft $F_{R,H} = \mu_{H} \cdot F_{N}$ (mit der Normalkraft $F_N$) gerichtet ist. Als Lösung für $m_2$
    wird sich ein Intervall von möglichen Massen ergeben.
    \item Nun sei $m_2 = 10\si{\kilo\gram}$. Wie stark werden die beiden Kisten beschleunigt?
  \end{enumerate}
Verwende: $\sin\left(36.9\si{\degree}\right) = 0.6,\, \cos\left(36.9\si{\degree}\right) = 0.8$, sowie $g = 9.8 \si{\meter \per \second^2}$
\newpage
