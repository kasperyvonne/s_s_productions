\section{Diskussion}
Die gefundenen Werte für die Schallgeschwindigkeit $c\ua{E} = \SI{+2.79(2)e+03}{\meter\per\second}$ und
$c\ua{D} &= \SI{+2.74(3)e+03}{\meter\per\second}$ weichen nur geringfügig von dem Literaturwert
$c\ua{lit} = \SI{2.73e3}{\meter\per\second}$ ab. Der mit dem Durchschallungsverfahren bestimmte Wert $c\ua{D}$
enthält den Literaturwert im Vertrauensbreich. Als Maß für den systematischen Fehler der Längenmessungen dienen
die Absizzenabschnitte der durchgeführten Regressionsrechnungen ($b\ua{E} = \SI{-1.6(5)}{\milli\meter}$ &
$b\ua{B} &= \SI{-3.7(9)}{\milli\meter}$). Diese liegen in einer Größenordnung die im Rahmen der Präzesion
der durchgeführten Messungen als plausibel erscheinen. \\
Die Auswertung des Cepstrums \ref{fig: cepstrum}
