\section{Versuchsaufbau/-durchführung}

\subsection{Versuchsaufbau}
Im Versuch US $1$ wird ein Ultraschallechoskop. Ultraschallsonden und einen Rechner
zum verarbeiten der Daten verwendet.
Als Probe verwendet man aus Acryl gefertigte Zylinder und Quader.
Die Höhe von jedem Zylinder muss vermessen werden. Zusätzlich wird auch
ein Augenmodel (vgl. \ref{}) untersucht.
Die Messungen werden im Impuls-Betrieb des Echoskop durchgeführt. Als Messverfahren werden
das Impuls-Echo-Verfahren und das Durchschallungs-Verfahren benutzt. Das Echoskop kann mittels
eines Kippschalters auf eines der beiden Verfahren eingestellt werden.
Die verwendete Sonde arbeitet in einem Bereich von $0-35\,\map{dB}$ und mit einer
Frequenz von $\SI{2}{\mega\hertz}$.
Wie in der Theorie angesprohen besitzt Luft einen hohen Absopttionskoeffizient, deshalb wird
auf den Proben entweder \emph{bidestilliertes Wasser} oder eine \emph{spezielle Kontaktcreme}
aufgetragen.

\subsection{Bestimmung der Dämpfung und der Schallgeschwindigkeit mit dem Impuls-Echo-Verfahren}
Der zu untersuchende Zylinder wird auf ein Papiertuch gestellt und anschließend
mit bidestilliertem Wasser benetzt. Die Sonde, im Impuls-Echo-Modus, wird auf dem
Zylinder plaziert. An dem Computer sind nun zwei Peaks zu erkennen. Der erste Peak
entstteht beim eintreten des Schalles in die Probe. Hingegen wird der zweite Peak
durch die reflektierte Welle erzeugt. Die Amplitude und die Laufzeit der Peaks
werden aus dem vom Computer erzeugten A-Scan abgelesen und notiert.
Insgesamt werden $7$ Zylinder untersucht.

\subsection{Bestimmung der Schallgeschwindigkeit mit dem Durschallungs-Verfahren}

Die Arcylzylinder werden auf eine Halterung gelegt, von beiden Seiten wird
mit der Kontaktcreme eine Sonde gekoppelt. Das Echoskop befindet sich im
Durchschallungsmodus. Aus dem erzeugten A-Scan kann die Durchlaufzeit der
Schallwelle bestimmt und aufgeschrieben werden. Die Messung wird für alle unterschiedlich
hohen Zylinder wiederholt.

\subsection{Spektrale Anaylse und Ceptstrum}

Für diesen Versuchsteil werden zwei Acrylscheiben und ein Zylinder (Höhe $\approx \SI{40}{\milli\meter}$)
gestapelt und dabei mit bidestillierten Wasser gekoppelt.
Damit die Sonde ideal mit dem Zylinder verbunden ist, wird auf die Oberseite des Zylinders
bidestillierten Wasser gegeben.
Auf dem A-Scan sollte nun eine Mehrfachecho (drei Peaks) erkennbar sein.
Neben dem eigentlich A-Scan zeigt der Computer auch das Spektrum und das Cepstrum des
eingegangenden Signal auf.
Das Ceptstrum gibt die logaritmierte inverse Fouriertransformation an.
Mit Hilfe des Spektrums und des Cepstrums sollen die Laufzeiten der reflektierten Wellen
bestimmt werden.

\subsection{Untersuchung eines Augenmodells}

Der schematische Aufbau des benutzten Auges ist in Abbildung \ref{fig: auge} dargestellt.
\begin{figure}[h]
  \centering
  \includegraphics[width=0.4\textwidth]{pics/auge.png}
  \caption{Schematische Aufbau des Auges \cite{anleitungus1}.}
  \label{fig: auge}
\end{figure}
Mit der Ultraschallechographie ist es möglich, den Abstand der Hornhaut, Iris, Linseneingang, Linsenausgang
und der Retina (reflektiert den Schall) zur Augenöffnung zu bestimmen.
Hierzu wird das Echo-Impuls Verfahren der Sonde und Kontaktcreme verwendet.
Die Sonde wird solange mit leichten Druck das Augenmodel gefahren, bis auf dem
A-Scan fünf Ausschläge erkennbar sind.
