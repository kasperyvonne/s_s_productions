\section{Auswertung}
Im Folgenden sollen die Messergebnisse analysiert werden und die relevanten Größen aus der Theorie berechnet werden. \\
Die Werte für die beiden Temperaturen $T_1$ und $T_2$ sowie die Drücke $p_a$ und $p_b$ sind in Tablle \ref{tab: tempdruck} aufgetragen.
\FloatBarrier
  \begin{table}
   \centering
   \begin{tabular}{S S S S S }
   \toprule
  {Zeit in $\si{\second}$} & {$T_1$ in $\si{\kelvin}$} & {$p_b$ in $\si{\bar}$} & {$T_1$ in $\si{\kelvin}$} & {$p_b$ in $\si{\bar}$} \\
  \midrule 
   0 & 293.75 & 5.25 & 294.05 & 5.20 \\
  60 & 295.05 & 7.25 & 294.15 & 2.40 \\
  120 & 296.65 & 7.50 & 294.15 & 2.90 \\
  180 & 298.15 & 8.00 & 294.05 & 3.00 \\
  240 & 300.45 & 8.50 & 294.05 & 3.10 \\
  300 & 302.75 & 9.00 & 294.05 & 3.20 \\
  360 & 305.25 & 9.50 & 293.95 & 3.20 \\
  420 & 307.55 & 10.00 & 293.45 & 3.20 \\
  480 & 309.65 & 10.50 & 291.45 & 3.20 \\
  540 & 311.65 & 11.00 & 288.55 & 3.20 \\
  600 & 313.55 & 11.25 & 285.95 & 3.20 \\
  660 & 315.55 & 11.75 & 283.95 & 3.22 \\
  720 & 317.15 & 12.00 & 282.75 & 3.22 \\
  780 & 318.75 & 12.50 & 281.75 & 3.22 \\
  840 & 320.35 & 13.00 & 280.95 & 3.20 \\
  900 & 321.75 & 13.25 & 280.45 & 3.20 \\
  960 & 323.05 & 13.50 & 280.05 & 3.20 \\
  \bottomrule
   \end{tabular}
   \caption{Temperaturen und Drücke}
   \label{tab: tempdruck}
    \end{table}

\FloatBarrier
Der zeitliche Verlauf der Temperaturen ist in den Abbildungen (...) dargestellt. Hierbei wurde mittels der Python Bibliothek $Scipy$ eine %... ersetzen, scipy am besten mit \emph{} setzen
Regression an eine Funktion der Form $T_i = A t^2 + B t + C$ bestimmt. Um im späteren Verlauf eine präzisere Aussage über den Temperaturverlauf
von $T_2$ zu gewährleisten, wurde selbiges für den Zeitraum nach $t_0 = 360 \si{\second}$ erneut durchgeführt ($T_2*$). In der Diskussion wird hierauf näher eingegangen. Die entsprechenden
Parameter sind in Tabelle \ref{tab: regress} zu finden. \\

\begin{table}
 \centering
 \begin{tabular}{S S S S }
 \toprule
{Funktion} & {$A$} & {$B$} & {$C$} \\
\midrule
 {$T_1$} & $\num{-6.77 \pm 1.67}$ & $\num{3.86\pm 0.18}$ & $\num{292.45 \pm 0.39}$ \\
{$T_2$} & $\num{-10.98 \pm 4.88}$ & $\num{-0.67 \pm 0.52}$ & $\num{295.52 \pm 1.13}$ \\
{$T_2*$} & $\num{32.83 \pm 7.24}$ & $\num{-6.91\pm 0.96}$ & $\num{315.85 \pm 3.01}$ \\
\bottomrule
 \end{tabular}
 \caption{Regressionsparameter}
 \label{tab: regress}
  \end{table}

\begin{figure}
  \centering
  \includegraphics[width = 14cm]{tabs/plot1.pdf}
  \caption{Temperaturverlauf $T_1$}
  \label{fig: plot1}
\end{figure}

\begin{figure}
  \centering
  \includegraphics[width = 14cm]{tabs/plot2_1.pdf}
  \caption{Temperaturverlauf $T_2$}
  \label{fig: plot2}
\end{figure}

\begin{figure}
  \centering
  \includegraphics[width = 14cm]{tabs/plot2_2.pdf}
  \caption{Temperaturverlauf $T_2$ für $t > 360 \si{\second}$}
  \label{fig: plot2*}
\end{figure}


Die gefundenen Regressionskurven erlauben es, die Differentialquotienten $\frac{dT_1}{dt}$ und $\frac{dT_2}{dt}$ zu berechnen. Die Werte für vier verschiedene  Zeiten sind %\mathup
in Tabelle \ref{tab: dTdt} aufgetragen, die entsprechenden Fehler $o_{dT_i}$ ergeben sich über die Gaußsche Fehlerfortpflanzung. Hierbei wurden die Zeiten $t > t_0$ gewählt, sodass die Kurve aus Abbildung \ref{fig: plot2*} verwendet werden konnte. %%\mathup

\subsection{Güteziffer}

Mittels des Zusammenhangs
\begin{equation}
  \nu_{real} = (m_1 c_w + m_k c_k) \frac{dT_1}{dt} \frac{1}{P} % %\mathup
\end{equation}
wurde die reale Güteziffer $\nu_{real}$ berechnet. Hierbei entsprechen $m_k c_k = 660 \si{\joule \per \second}$ und $c_1 c_W$ den Wärmekapzitäten von Apparatur und Wasser. Die Masse $m_1 = V_1 \cdot \rho_w$ ergibt sich mit
der Dichte des Wassers $\rho_w \approx 1000 \si{\kilo \gram \per \meter ^3}$  und der entsprechenden Füllmenge $V_1 = 3 l$. Der Wert $c_w = 4182\si{\joule \kilo \gram^{-1} \kelvin^{-1}}$ wurde der Literatur \cite{demtröder} entnommen. %^3 durch \cubic ersetzen ->\per\cubic\meter, ^{-1} durch \per ersetzen
Da hier lediglich die Größe $\frac{dT_1}{dt}$ fehlerbehaftet ist, berechnet sich der Fehler zu: %\mathup
\begin{equation}
  o_{\nu_{real}} = \left| const \right| \cdot o_{dT_1} %\mathup
  \label{eq: errorconst}
\end{equation}

Die Ergebnisse der Rechnung, sowie die ideale Güteziffer $\nu_{ideal}$ (formel) (fehlerunbehaftet) sind in Tabelle \ref{tab: dTdt} aufgetragen.
\begin{table} 
 \centering 
 \begin{tabular}{S S S S S } 
 \toprule  
{Zeit in $\si{\second}$} & {$\frac{dT_1}{dt}$ in $\si{\kelvin \per \second}$} & {$\frac{dT_2}{dt}$ in $\si{\kelvin \per \second}$} & {$\nu_{real}$}& {$\nu_{ideal}$} \\ 
\midrule  
 0 & $\num{ 0.0386 \pm 0.0018 }$ & $\num{ -0.0691 \pm 0.0096 }$ & $\num{ 510.14 \pm 23.37 }$ & $\num{ -68.67 }$\\ 
60 & $\num{ 0.0378 \pm 0.0018 }$ & $\num{ -0.0651 \pm 0.0097 }$ & $\num{ 2.85 \pm 0.13 }$ & $\num{ 24.33 }$\\ 
120 & $\num{ 0.0370 \pm 0.0018 }$ & $\num{ -0.0612 \pm 0.0098 }$ & $\num{ 2.71 \pm 0.13 }$ & $\num{ 9.40 }$\\ 
180 & $\num{ 0.0362 \pm 0.0019 }$ & $\num{ -0.0572 \pm 0.0100 }$ & $\num{ 2.58 \pm 0.13 }$ & $\num{ 6.10 }$\\ 
240 & $\num{ 0.0354 \pm 0.0019 }$ & $\num{ -0.0533 \pm 0.0102 }$ & $\num{ 2.40 \pm 0.13 }$ & $\num{ 4.27 }$\\ 
300 & $\num{ 0.0346 \pm 0.0020 }$ & $\num{ -0.0494 \pm 0.0106 }$ & $\num{ 2.28 \pm 0.13 }$ & $\num{ 3.40 }$\\ 
360 & $\num{ 0.0338 \pm 0.0021 }$ & $\num{ -0.0454 \pm 0.0110 }$ & $\num{ 2.17 \pm 0.14 }$ & $\num{ 2.84 }$\\ 
420 & $\num{ 0.0329 \pm 0.0023 }$ & $\num{ -0.0415 \pm 0.0114 }$ & $\num{ 2.12 \pm 0.15 }$ & $\num{ 2.44 }$\\ 
480 & $\num{ 0.0321 \pm 0.0024 }$ & $\num{ -0.0375 \pm 0.0119 }$ & $\num{ 2.04 \pm 0.15 }$ & $\num{ 2.01 }$\\ 
540 & $\num{ 0.0313 \pm 0.0025 }$ & $\num{ -0.0336 \pm 0.0124 }$ & $\num{ 1.97 \pm 0.16 }$ & $\num{ 1.67 }$\\ 
600 & $\num{ 0.0305 \pm 0.0027 }$ & $\num{ -0.0297 \pm 0.0130 }$ & $\num{ 1.91 \pm 0.17 }$ & $\num{ 1.46 }$\\ 
660 & $\num{ 0.0297 \pm 0.0028 }$ & $\num{ -0.0257 \pm 0.0136 }$ & $\num{ 1.86 \pm 0.18 }$ & $\num{ 1.34 }$\\ 
720 & $\num{ 0.0289 \pm 0.0030 }$ & $\num{ -0.0218 \pm 0.0142 }$ & $\num{ 1.81 \pm 0.19 }$ & $\num{ 1.28 }$\\ 
780 & $\num{ 0.0281 \pm 0.0032 }$ & $\num{ -0.0178 \pm 0.0149 }$ & $\num{ 1.76 \pm 0.20 }$ & $\num{ 1.23 }$\\ 
840 & $\num{ 0.0273 \pm 0.0033 }$ & $\num{ -0.0139 \pm 0.0155 }$ & $\num{ 1.71 \pm 0.21 }$ & $\num{ 1.20 }$\\ 
900 & $\num{ 0.0264 \pm 0.0035 }$ & $\num{ -0.0100 \pm 0.0162 }$ & $\num{ 1.68 \pm 0.22 }$ & $\num{ 1.18 }$\\ 
960 & $\num{ 0.0256 \pm 0.0037 }$ & $\num{ -0.0060 \pm 0.0169 }$ & $\num{ 1.64 \pm 0.24 }$ & $\num{ 1.16 }$\\ 
1020 & $\num{ 0.0248 \pm 0.0038 }$ & $\num{ -0.0021 \pm 0.0176 }$ & $\num{ 1.60 \pm 0.25 }$ & $\num{ 1.15 }$\\ 
\bottomrule 
 \end{tabular} 
 \caption{Differenzenquotienten und reale Güteziffer} 
 \label{tab: dTdt} 
  \end{table}

\subsection{Massendurchsatz und Mechanische Leistung}
Zur Bestimmung des Massendurchsatzes $\frac{dm}{dt}$ wird zunächst die Verdampfungswärme $L$ benötigt. Hierzu wird der aus $V203$ \cite{anleitung203} bekannte Zusammenhang des Druckverlaufs für $p_b$ in Abhängigkeit von der Temperatur%%\mathup
\begin{equation}
  p_b = p_0 \exp{-\frac{L}{R T_1}} %Klammern setzen
\end{equation}
ausgenutzt. Durch einmaliges Anwenden des Logarithmus erhält man einen linearen Zusammenhang der Form:
\begin{equation}
  \log{p_b} = \log{p_0} -\frac{L}{R} \cdot \frac{1}{T_1} 
\end{equation}
Durch halb-logarithmisches Auftragen der Werte für $p_b$ gegen die Reziproken der Temperatur $T_1$ (siehe Abb. \ref{fig: plot3}) kann also mit Hilfe einer lineare Regression die Steigung $-\frac{L}{R}$ ermittelt werden.
Steigung und zugehöriger Fehler berechnen sich mit:
\begin{equation}
  m= \frac{\left( N  (\sum x_i y_i) - (\sum x_i)(\sum y_i)\right)}{N (\sum x_i^2)- (\sum x_i)^2 }    \quad   o_m=\frac{N o_y^2}{N (\sum x_i^2)- (\sum x_i)^2 }
\end{equation}
Es ergibt sich:
\begin{equation}
  -\frac{L}{R} =  (-2228.58 \pm 111.63) \si{\kelvin}
\end{equation}

\begin{figure}
  \centering
  \includegraphics[width = 14cm]{tabs/plot3.pdf}
  \caption{Regression der Druckkurve}
  \label{fig: plot3}
\end{figure}

Mit der allgemeinen Gaskonstante $R = 8.314\si{\joule \mol^{-1} \kelvin^{-1}}$ ergibt sich $L$  mit \eqref{eq: errorconst} zu: %^{-1} mit \per ersetzen
\begin{equation}
  L = (1.85 \pm 0.09)\cdot 10^{4} \si{\joule \mol^{-1}} %^{-1}
\end{equation}
Der Massendurchsatz berechnet sich nun mit:
\begin{equation}
  \frac{dm}{dt} = (m_2 c_w + m_k c_k)\frac{dT_2}{dt} \frac{1}{L}
\end{equation}
Mit der identischen Wärmekapazitäten $m_2 c_w$ und $m_k c_k$ wie oben. Die berechneten Werte sind in Tabelle \ref{tab: dmdtNmech} aufgetragen, der relative Fehler berechnet sich gemäß der üblichen Formel
für Produkte und Quotienten:
\begin{equation}
  \frac{o_{dm}}{\left| dm \right|} = \left| const \right| \sqrt{\left(\frac{o_{dT_2}}{\left| dT_2 \right|}\right)^2 + \left(\frac{o_{L}}{\left| L \right|}\right)^2} %%\mathup
\end{equation}
Des Weiteren wurde mittels der molaren Masse des Gases $120,91 \si{g \mol ^{-1}}$ \cite{demtröder} umgerechnet in die Einheit $\si{\gram \per \second}$. \\%g? (wenn gramm dann \gram) 
Abschließend soll nun noch die mechanische Leistung des Kompressors errechnet werden. In Formel (x) kann $\frac{1}{\rho}$
näherungsweise mit der idealen Gasgleichung bestimmt werden.
\begin{equation}
  pV = nRT \Leftrightarrow  \frac{1}{\rho} = \frac{\rho_0 T_0 p_a}{T_2 p_0}
\end{equation}
Mit den aus der Versuchsanleitung \cite{anleitung206} entnommenen Konstanten des Gases $CI_2F_2C$. Die entsprechenden Mittelwerte befinden sich in Tabelle \ref{tab: dmdtNmech}.





\begin{table}
 \centering
 \begin{tabular}{S S S }
 \toprule
{Zeit in $\si{\second}$} & {$\frac{dm}{dt}$ in $\si{\gram \per \second}$} & {$N_{mech}$ in $\si{\watt}$}  \\
\midrule

360  & $\num{ 3.88 \pm 0.96 }$ & $\num{ 77.49  }$ \\

540  & $\num{ 2.87 \pm 1.07 }$ & $\num{ 64.45  }$ \\

720  & $\num{ 1.86 \pm 1.22 }$ & $\num{ 43.84  }$ \\

900  & $\num{ 0.85 \pm 1.39 }$ & $\num{ 21.64  }$ \\

\bottomrule
 \end{tabular}
 \caption{Massendurchsatz und Kompressorleistung}
 \label{tab: dmdtNmech}
  \end{table}

