\section{Auswertung}


\subsection{Bestimmung der Kugeldichten}

In Tabelle \ref{tab:messwerte_kugel} sind die gemessenen Radien und Masses der Kugel 
einzusehen.

\begin{table}
\centering
\begin{tabular} {ccccccc}
	\toprule
  & \multicolumn{3}{c}{Radius in $\si{\meter}\cdot \num{e-3}$}  & \multicolumn{3}{c}{Masse in $\si{\kilogram}\cdot \num{e-3}$} \\
\midrule \\
Kugel 1 & $\num{7.565} $&  $\num{7.5625} $ & $\num{7.565} $  & $\num{4.45}$ & $\num{4.44} $ & $\num{4.44} $ \\
Kugel 2  & $\num{7.65} $&  $\num{7.65} $ & $\num{7.65} $ & $\num{4.6}$ & $\num{4.6} $ & $\num{4.61} $ \\
\bottomrule
\end{tabular}
\caption{Abmessung der Kugeln}
\label{tab:messwerte_kugel}
\end{table}

Der Mittelwert der Messreihen wird mittels

\begin{equation}
\label{eq:mittel}
\bar{x}=\frac{1}{n}\sum_{i=1}x_i
\end{equation}

berechnet. Dabei wird der zugehörige Fehler
durch 
\begin{equation}
\label{eq:stand_ab}
\bar{\sigma}_{\bar{x}}=\sqrt{\frac{1}{n(n-1)}\sum_{i=1}^{n}(x_i-\bar{x})^2}.
\end{equation}

bestimmt.

Für die Messungen aus \ref{tab:messwerte_kugel} ergeben sich 
folgende gemittelten Werte:

\begin{align}
\label{eq:abmessungen_kugel}
\begin{aligned}
\overline{m}_{1}&= \left(\num{0.0044}\pm\num{e-6}\right) \si{\kilogram} \\
\overline{m}_{2}&= \left(\num{0.0046}\pm\num{e-6}\right) \si{\kilogram} \\
\hfill \\
\overline{r}_{1}&= \left(\num{0.0076}\pm\num{e-7}\right) \si{\meter} \\
\overline{r}_{2}&= \left(\num{0.0077}\pm\num{0}\right) \si{\meter} \\
\end{aligned}
\end{align}

Mithilfe der gemittelten Werte und der Formel

\begin{equation*}
\rho=\frac{m}{V_{k}} \qquad \text{mit} \, V_{k}=\frac{4}{3}\pi r^3
\end{equation*}

können die Dichten der Kugeln bestimmt werden.
Es ergibt sich:

\begin{align*}
\rho_{1}&=\left(\num{2451.0}\pm\num{1.6}\right) \si{\kilogram\per\cubic\meter}\\
\rho_{2}&=\left(\num{2454.7}\pm\num{1.5}\right) \si{\kilogram\per\cubic\meter}
\end{align*}

\subsection{Bestimmung der Viskosität von Wasser bei Zimmertemperatur}
Um die Viskosität zu bestimmen wird neben der Dichte von Wasser die Fallzeit von der kleinen Kugel benötigt.
Die gemessenen Werte sind in Tabelle \ref{tab:messwerte_fallzeit_kugel_klein} abgebildet. 
\begin{table}
\centering
\begin{tabular} {c}
	\toprule
  Fallzeit $t$ in $\si{\second}$ \\
  \midrule
  $\num{12.7}$ \\
  $\num{12.8}$ \\
  $\num{12.9}$ \\
  $\num{12.8}$ \\
  $\num{12.8}$ \\
  $\num{12.8}$ \\
  $\num{12.7}$ \\
  $\num{12.7}$ \\
  $\num{12.7}$ \\
  $\num{12.8}$ \\
\bottomrule
\end{tabular}
\caption{Fallzeiten der kleinen Kugel im Viskosimeter bei $\SI{20}{\degreeCelsius}$}
\label{tab:messwerte_fallzeit_kugel_klein}
\end{table}

Gemittelt ergibt sich

\begin{equation}
\label{eq:gemittelte_fallzeit_klein}
t_{1}=\left(\num{12.78}\pm\num{0.02}\right) \si{\second}.
\end{equation}
Mittels der Gleichung \ref{eq: eta} ergibt sich als Viskosität
für Wasser bei $\SI{20}{\degreeCelsius}$:

\begin{equation}
\label{eq:viskosi_wasser}
\eta_{20}=\left(\num{0.0014}\pm\num{e-5}\right) \si{\pascal\second}
\end{equation}

Für die Dichte von Wasser bei $\SI{20}{\degreeCelsius}$ wurde als
Literaturwert $\rho_{20}=\SI{998.21}{\kilogram\per\cubic\meter}$ angenommen. %Wichtig: Literaturverweis angeben

\subsection{Bestimmungen der Apperaturkonstante für die große Kugel}





