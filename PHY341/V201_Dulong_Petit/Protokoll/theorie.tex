\section*{Zielsetzung}
Der Versuch soll überprüfen, ob die Dulong-Peptitsche Gesetz
angesprochenden Ozillation der Atome. Einer klassischen Beschreibung genügt,
oder ob eine quantenmechanische Beschreibung von Nöten ist.


\section{Theorie}

\subsection{Definition der spezifischen Wärmekapazität}

Findet an einem Körper eine Temperaturänderung $\Delta T$ statt, ohne das an ihm 
Arbeit verrichet wird. So kommt es zu einer Wärmeaufnahme bzw. Abgabe $\Delta Q$.
Mit dem ersten Hauptsatz der Thermodynamik ergibt sich dann der Zusammenhang

\begin{equation*}
\Delta Q=m c \Delta T.
\end{equation*}

Dabei sei $m$ die Masse und $c$ die Wärmekapazität 
des Körpers.
Wird zusätzlich die Masseneinheit berücksichtig
so wird $c$ als spezifische Wärmekapazität bezeichnet.

Desweitern gibt die Molwärme $C$
an, was für eine Wärmemenge $\map{d}Q$ benötigt wird,
um ein Mol eines Stoffes um $\map{d}T$ zu erwärmen.

Dabei hängt der Betrag der Molwärme  davon ab unter welchen Bedingungen 
die Temperaturänderung hervor gerufen wird.
Hierbei wird zwischen der spezfischen Wärmekapazität bei konstantem
Volumen $C_{V}$ und bei kosntantem Druck $C_{p}$ unterschieden.
Die Größe $C_{V}$ sei gegeben durch:

\begin{equation}
\label{eq:molwarme}
C_V=\left(\frac{\map{d}U}{\map{d}T}\right)_V
\end{equation}


\subsection{Dulong-Petitsch-Gesetz}

Das Dulong-Petitsche-Gesetz sagt aus, dass die Atomwärme (im festem Aggregatzustand) bei 
konstantem Volumen unabhängig von den chemischen Eigenschaften eines Elementes 
und gleich $3R$ sei.
Man bezeichnet $R$ als die allgemeine Gaskonstante.
Das Gesetz begründet die makroskopischen thermodynamischen Phänomene, 
auf zufällige mikroskpoischen Bewegungen der Atome.
Das Dulong-Petitsch-Gesetz wird zum Teil aus dem 
das sogenannte Äquipartitionstheorem hergeleitet. 
Das Theorem gibt Rückschlüsse auf die kinetische Energie eines Atom
im thermischen Gleichgewicht, druch:

\begin{equation*}
\left< E_{kin} \right>=\frac{1}{2}kT
\end{equation*} 

Mit diesem und weitern Zusammenhängen folgt für die innere Energie $U$ 
einer Gitterstruktur mit drei Freiheitsgraden:

\begin{equation*}
\left< U_{fest} \right> = 3 \left< U \right>=3RT \quad \Rightarrow \quad C_{V}\overset{\eqref{eq:molwarme}}{=}3R
\end{equation*}

\subsubsection{Quantenmechanische Betrachtung}
Bei der quantenmechanischen Betrachtung wird angenommen, 
dass die ozillierenden Atome nur quantisierte Energiezustände $\Delta u=n\hbar \omega \quad n\in\mathbb{N}$
annehmen können.
Mit zu Hilfe nahme der Boltzmann-Verteilung kann dann auf den Zusammenhang 
\begin{equation}
\label{eq:quant}
\left< U_{qm} \right> =\frac{3N_L \hbar \omega}{\exp\left(\hbar \frac{\omega}{k} T\right) -1}
\end{equation}

geschlossen werden. Dieser gilt für Bewegungen mit drei Freiheitsgraden.
Die Konstante $N_L = \SI{6.02e23}{\per\mol}$ wird als Loschmidtsche Zahl bezeichnet.
Für $T\to\infty$ läuft \eqref{eq:quant} gegen, den aus der klassichen Physik bekannten Wert, 
$3RT$.

\subsection{Bestimmung der spezifischen Wärmekapazität}
Da eine experiementelle Bestimmung der spezifischen Wärmekapazität 
bei konstanten Volumen $C_V$ kaum möglich ist.
Bestimmt stattdessen die Wärmekapazität bei konstanten Druck $C_P$.
Beide Größen könenn durch den Zusammenhang

\begin{equation}
\label{eq:umrechnung_cp_cv}
C_P-C_V=9\alpha^2\kappa V_0 T \quad \text{mit} \kappa=V\left(\frac{\partial p}{\partial V}\right)_T
\end{equation}

ineinander umgerechnet werden.
Hierbei sei $\alpha$ ein linearer Ausdehnungskoeffizient, $\kappa$ das Kompressionsmodul und
$V_0$ das Molvolumen.

Die Bestimmung erfolgt mithilfe eines Kalorimeters,
eins rohrförmigen Probekörpers mit der Masse $m_k$ und der Temperatur 
$T_k$. Zusätlich gibt es ein mit Wasser gefülltes Gefäß.
Das Wasser besitzt die Masse $m_w$ und die Temperatur $T_w$.
Nach eintauchen de Körper stellt sich im Gefäß die 
Mischtemperatur $T_w$ ein. 
Es ergibt sich folgende Abhängigkeit:

\begin{equation*}
\label{eq:zusammenhang_ck}
c_k=\frac{\left(c_wm_w+c_gm_g\right)\left(T_m-T_w\right)}{m_k\left(T_k-T_m\right)}
\end{equation*}

Die Wärmekapiztät des Kaloriemeters wird  druch $c_gm_g$ dargestellt.
Sie wird in einer weitern Messung bestimmt.
Bei dieser werden zwei Wassermengen mit $m_x$ und $m_y$ und den 
dazugehörigen Temperaturen $T_x$ und $T_y$ miteinander vermischt.
Mit der sich einstellenden Temperatur $T_M$ kann dann
durch den Zusammenhang
\begin{equation}
\label{eq:zusammenhang_cgmg}
c_gm_g=\frac{c_wc_y\left(T_y-T_M\right)-c_wm_x\left(T_M-T_x\right)}{\left(T_M-T_x\right)}
\end{equation}
auf die Wärmekapazität geschlossen werden.

\subsection{Thermoelemente und ihre Funktionsweise} %Skizze

Ein Thermoelement wird häufig eingesetzt um eine  
Temperatur mit einer hohen Einstellungsschwindigkeit zu messen.
Es besteht aus zwei Metallen, die sich jeweils durch
verschiedene Wärmeleitkoeffizienten auszeichnen.
Am Ende des Themoelementes befindet sich ein Punkt an dem sich
beide Elemente berrühren. Dieser Punkt wird als Berührungspunkt 
bezeichnet.
Ein Thermoelement besitzt zwei Berührungsstellen.
Diese messen jeweils unterschiedliche Temperaturen $T_1$ und $T_2$.
Herrscht eine Temperaturdifferenz zwischen $T_1$ und $T_2$, so 
stellt sich zwischen den Kontaktstellen ein Kontaktpotential 
ein. 
Durch Messung der Spannung, kann anschließend auf 
Temperaturdiffernz geschlossen werden.







