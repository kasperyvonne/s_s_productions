\section*{Zielsetzung}
Der Versuch soll eine Aussage darüber treffen, ob das Dulong-Petitsche Gesetz, der klassichen Physik genügt, oder
eine quantenmechanische Beschreibung von Nöten ist.

\section{Theorie}

\subsection{Definition der spezifischen Wärmekapazität}

Findet an einem Körper eine Temperaturänderung $\Delta T$ statt, ohne das an ihm 
Arbeit verrichet wird. So kommt es bei ihm zu einer Wärmeaufnahme oder -abgabe $\Delta Q$.
Mit dem ersten Hauptsatz der Thermodynamik ergibt sich der Zusammenhang

\begin{equation*}
\Delta Q=m c \Delta T.
\end{equation*}

Dabei sei $m$ die Masse und $c$ die Wärmekapazität 
des Körpers.
Wird zusätzlich die Masseneinheit berücksichtig
so wird $c$ als spezifische Wärmekapazität bezeichnet.

Weiter ist die Molwärme $C$ eine für den Versuch relevante Größe.
Sie gitb an was für eine Wärmemenge $\map{d}Q$ benötigt wird,
um ein Mol eines Stoffes um $\map{d}T$ zu erwärmen.

Der Betrag der Molwärme hängt davon ab unter welchen Bedingungen 
die Temperaturänderung hervor gerufen wurde.
Zum Beispiel wird zwischen der spezfischen Wärmekapazität beo konstantem
Volumen $C_{V}$ und bei kosntantem Druck $C_{p}$.
Dabei sei $C_{V}$:

\begin{equation}
\label{eq:molwarme}
C_V=\left(\frac{\map{d}U}{\map{d}T}\right)_V
\end{equation}


\subsection{Dulong-Petitsch-Gestz}

Das Dulong-Petitsche-Gesetz sagt aus, dass die Atomwärme (im festem Aggregatzustand) bei 
konstantem Volumen unabhängig von den chemischen Eigenschaften eines Elementes 
und gleich $3R$, hier sei $R$ die allgemeine Gaskonstante.
Das Gestz begründet die makroskopischen thermodynamsichen Phänomene, 
auf zufälligen mikroskpoischen Bewegung deer Atome bzw. Moleküle.
Um die Aussage herzuleiten, nutzt das Dulong-Petitsch-Gestz
das sogenannte Äquipartitionstheorem. 
Dieses liefert gibt Rückschlüsse auf die kinetische Energie eines Atom
im thermischen Gleichgewicht:

\begin{equation*}
\left< E_{kin} \right>=\frac{1}{2}kT
\end{equation*} 

Mit diesem und weitern Zusammenhängen folgt danbn für drei Bewegungsrichtungen:

\begin{equation*}
\left< U_{fest} \right> = 3 \left< U \right>=3RT \quad \Rightarrow \quad C_{V}\overset{\eqref{eq:molwarme}}{=}3R
\end{equation*}

\subsubsection{Quantenmechanische Betrachtung}
Bei der quantenmechanischen Betrachtung wird angenommen 
das die ozillierenden Atome nur quantisierte Energiezustände $\Delta u=n\hbar \omega \quad n\in\mathbb{N}$
annehmen können.
Mit zu Hilfe nahme der Boltzmann-Verteilung kann dann auf den Zusammenhang 
für drei Raumrichtung 
\begin{equation}
\label{eq:quant}
\left< U_{qm} \right> =\frac{3N_L \hbar \omega}{\exp\left(\hbar \frac{\omega}{k} T\right) -1}
\end{equation}

geschlossen werden.
Die Konstante $N_L = \SI{6.02e23}{\per\mol}$ wird als \emph{Loschmidtsche Zahl} bezeichnet.
Für $T\to\infty$ läuft \eqref{eq:quant} gegen den aus der klassichen Physik bekannten Wert 
$3RT$.

\subsection{Bestimmung der spezifischen Wärmekapazität}
Da eine experiementelle Bestimmung der spezifischen Wärmekapazität 
bei konstanten Volumen $C_V$ kaum möglich ist.
Bestimmt man die Wärmekapazität bei konstanten Druck $C_P$.
Beide Größen kann man durch den Zusammmenhang

\begin{equation}
\label{eq:umrechnung_cp_cv}
C_P-C_V=9\alpha^2\kappa V_0 T \quad \text{mit} \kappa=V\left(\frac{\partial p}{\partial V}\right)_T
\end{equation}

Hierbei sei $\alpha$ ein linearer Ausdehnungskoeffizient, $\kappa$ das Kompressionsmodul und
$V_0$ das Molvolumen.

Für die Bestimmmung wird ein Kaloriemeter verwendet.
Dazu ein Rohrförmiger Probekörper mit der Masse $m_k$ der eine Temperatur 
$T_k$ besitzt. Zusätlich gibt es ein mit Wasser gefülltes Gefäß.
Das Wasser besitzt die Masse $m_w$ und die Temperatur $T_w$.
Nach eintauchen des Körper stellt sich im Gefäß die 
Temperatur $T_w$ ein. Also Zusammenhang für die 
spezifische Wärmekapazität ergibt sich:
\begin{equation*}
\label{eq:zusammenhang_ck}
c_k=\frac{\left(c_wm_w+c_gm_g\right)\left(T_m-T_w\right)}{m_k\left(T_k-T_m\right)}
\end{equation*}

\subsection{Thermoelemente und ihre Funktionsweise} %Skizze

Ein Thermoelement wird häufig eingesetzt um eine  
Temperatur mit einer hohen Einstellungsschwindigkeit zu messen.
Es besteht aus zwei Metallen, die sich durch zwei 
verschiedene Wärmeleitkoeffizienten auszeichnen.
Am Ende des Themoelemts befindet sich eine Berührungstelle zwischen 
den beiden Elementen.
An dieser kommt es später zu einem Elektronenfluss.
Ein Thermoelement besitzt zwei Berührungsstellen, die 
jeweils zwei verschiedene Temperaturen messen.
Werdenn unterschiedliche Temperaturen gemessen, so 
stellt sich zwischen den Kontaktstellen ein Kontaktpotential 
ein. 
Durch Messung der Spannung, kann auf die 
Temperatur der beiden Wasserbäder geschlossen werden.





