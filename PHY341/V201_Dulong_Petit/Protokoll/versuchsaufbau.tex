\section{Versuchsaufbau/-durchführung}

\subsection{Versuchsaufbau}
\begin{figure}
  \centering
  \includegraphics[width=0.65\textwidth]{bilder/versuchsaufbau_dulon_peptip.pdf}
  \caption{Schematische Darstellung eines Thermoelements}
  \label{fig:aufbau}
\end{figure}
Der Versuchsaufbau ist in Abbildung \ref{fig:aufbau} dargestellt.
Im Wesentlichen besteht der Versuch aus einem Kaloriemeter, einem Thermoelement, %kalori
Probemassen (Graphit, Zinn und Aluminium). %und
Zusätzlich werden noch ein Wasserbad mit Heizplatte und
ein mit Eiswasser gefüllten Dewargefäß benötigt.

\subsection{Versuchsdurchführung}
%\subsubsection{Kalibierung des Thermoelemtentes} Soll ich das auch machen ?

\subsubsection{Bestimmung der Wärmekapazität des Kalorimeters}
Um die Wärmekapazität des Kalorimeters zu bestimmen, werden
zwei Wassermengen $m_x$ und $m_y$ mit der dazugehörigen Temperatur $T_x$ und $T_y$
im Kalorimeter vermischt.
Die sich einstellende Mischtemperatur $T_M$ wird durch ein Thermoelement gemessen.
Dabei sei einer der beiden Kontaktstellen %wieso immer sei?
in einem mit Eiswasser gefüllten Deware-Gefäß.
Mittels $T_M$ und \eqref{eq:zusammenhang_cgmg} kann dann auf die Wärmekapazität des
Kaloriemeters geschlossen werden. %kalori

\subsubsection{Bestimmung der Wärmekapazität}
Zu Anfang des Versuches wird ein Dewargefäß mit Eiswasser gefüllt.
Eine Kontaktstelle des Thermoelements wird in das Eiswasser gegeben.
Es dient für weitere Temperaturmessung als Referenzpunkt. %messungen
Generell werden alle Temperaturmessungen mit einem Thermoelement ausgeführt.
Anschließend wird ein Probekörper in einem Wasserbad erhitzt.
Während dessen wird ein weiters Deware-Gefäßs mit $\SI{600}{\milli\liter}$ Wasser gefüllt. %weiteres
In ihm soll die Messung der Mischtemperatur erfolgen.
Dazu wird der erhitzte Probekörper in das Wasser geben. %gegeben
Bevor wird jedoch noch die Temperatur vom Probekörper und die des Wassers gemessen. %davor...ausdruck, lieber andersrum aufschreiben
Nachdem der Probekörper im Deware-Gefäß platziert wurde, wird die Mischtemperatur des
Wassers gemessen. Der Vorgang wird für Graphit und Zinn jeweils drei Mal und für
Aluminium einmal wiederholt.
