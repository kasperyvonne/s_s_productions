\section{Auswertung}
\subsection{Eichung des Thermoelements}
Der Zusammenhang zwischen dem Kontaktpotential des Kalorimeters und der Temperatur soll im späteren Verlauf
durch einen linearen Zusammenhang angenähert werden. Hierzu wurde eine Kontakstelle in Eiswasser($0 \si{\celsius}$)
und eine in kochendes Wasser ($100 \si{\celsius}$) gehalten. Folgender Wert wurde hierbei am Kalorimeter abgelesen:
\begin{equation}
  U_{eich} = 3.98 \si{\milli\volt}
\end{equation}
Die Steigung der Geraden ergibt sich zu:
\begin{equation}
  m = \frac{100 \si{\kelvin}}{U_{eich}} = 25.13 \si{\kelvin \per \milli\volt}
\end{equation}
Hiermit lässt sich die Temperatur in Kelvin über den Zusammenhang
\begin{equation}
  T(U) = m \cdot U + 275.13 \si{\kelvin}
\end{equation}
berechnen. Im späteren Verlauf werden die Spannungen direkt durch die zugeordnete Temperatur ersetzt.

\subsection{Bestimmung der Wärmekapazität des Kalorimeters}
Zur Bestimmung der spezifischen Wärmekapazitäten wird zunächst die Wärmekapazität des Kalorimeters über den Zusammenhang
(Formel x) berechnet. Die Massen $m_x$ und $m_y$ ergeben sich mittels der Dichte von Wasser bei $40\si{\celsius}$: $\rho_w = ...\si{\gram\meter^3}$ (quelle).
Der Wert $c_w = ...$ wurde der Praktikumsanleitung entnommen. Die gemessenen Werte, sowie das Ergebnis für $c_g m_g$ sind in $tab$ aufgetragen.
\begin{table}
  \centering
  \begin{tabular}{S S S S S}
      \toprule
    {$m_x$ in $\si{\gram}$} & {$m_y$ in $\si{\gram}$} & {$T_x$ in $\si{\kelvin}$} & {$T_y$ in $\si{\kelvin}$} & $c_g m_g$ in  \\
    \midrule
  \end{tabular}
  \caption{Massen, sowie Temperaturen der beiden Wassermengen zur Bestimmung der spezifischen Wärmekapazität des Kalorimeters}
  \label{tab: cgmg}
\end{table}

\subsection{Bestimmung der Wärmekapazitäten von Blei, Graphit und Aluminium}
Die berechneten Temperaturen aus den gemessenen Kontaktpotentialen zur Bestimmung der Wärmekapazitäten befinden sich in den Tabellen (...).
\begin{table}
