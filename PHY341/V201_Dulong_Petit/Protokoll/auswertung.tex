\section{Auswertung}
\subsection{Eichung des Thermoelements}
Der Zusammenhang zwischen dem Kontaktpotential des Kalorimeters und der Temperatur soll im späteren Verlauf
durch einen linearen Zusammenhang angenähert werden. Hierzu wurde eine Kontakstelle in Eiswasser($0 \si{\celsius}$)
und eine in kochendes Wasser ($100 \si{\celsius}$) gehalten. Folgender Wert wurde hierbei am Kalorimeter abgelesen:
\begin{equation}
  U_{eich} = 3.98 \si{\milli\volt}
\end{equation}
Anschließend wurden beide Kontaktstellen in Eiswasser gehalten. Hier wurde eine Spannung von $0 \si{\volt}$ gemessen.
Die Steigung der Geraden ergibt sich damit zu:
\begin{equation}
  m = \frac{100 \si{\kelvin}}{U_{eich}} = 25.13 \si{\kelvin \per \milli\volt}
\end{equation}
Hiermit lässt sich die Temperatur in Kelvin über den Zusammenhang
\begin{equation}
  T(U) = m \cdot U + 275.13 \si{\kelvin}
  \label{eq: UtoTemp}
\end{equation}
berechnen.

\subsection{Bestimmung der Wärmekapazität des Kalorimeters}
Zur Bestimmung der spezifischen Wärmekapazitäten wird zunächst die Wärmekapazität des Kalorimeters über den Zusammenhang
(Formel \eqref{eq:zusammenhang_cgmg}) berechnet. Die Massen $m_x$ und $m_y$ ergeben sich mittels der Dichte von Wasser bei $40\si{\celsius}$: $\rho_w = 992.2\si{\kilo\gram\meter^3}$ \cite{lit_dichte}. %Formel und Quelle fehlen
Der Wert $c_w = 4.18 \si{\joule \per (\gram \kelvin)}$ wurde der Praktikumsanleitung \cite{anleitung201} entnommen. Die gemessenen Werte, sowie das Ergebnis für $c_g m_g$ sind in Tabelle \ref{tab: cgmg} aufgetragen.
\begin{table}
  \centering
  \begin{tabular}{S S S S S S S}
      \toprule
    {$m_x$ in $\si{\gram}$} & {$m_y$ in $\si{\gram}$} &  {$U_x$ in $\si{\milli \volt}$} & {$T_x$ in $\si{\kelvin}$} & {$U_y$ in $\si{\milli \volt}$} &{$T_y$ in $\si{\kelvin}$} & {$c_g m_g$ in $\si{\joule \per \kelvin}$} \\
    \midrule
    $\num{297.7}$  &    $\num{297.7}$  &   $\num{0.75}$  &  $\num{291.97}$ &   $\num{2.76}$  &  $\num{342.50}$ &    $\num{472.66}$ \\
  \end{tabular}
  \caption{Massen, sowie Temperaturen der beiden Wassermengen zur Bestimmung der spezifischen Wärmekapazität des Kalorimeters und berechneter Wert}
  \label{tab: cgmg}
\end{table}

\subsection{Bestimmung der Wärmekapazitäten von Zinn, Graphit und Aluminium}
Die berechneten Temperaturen aus den gemessenen Kontaktpotentialen zur Bestimmung der Wärmekapazitäten befinden sich in den Tabellen \ref{tab: graphit} bis \ref{tab: alu}.
\begin{table}
  \centering
  \begin{tabular}{S S S S}
    \toprule
  {Durchgang} &  {$U_k$ in $\si{\milli \volt}$} & {$U_w$ in $\si{\milli \volt}$} &{$U_m$ in $\si{\milli \volt}$}  \\
  \midrule
  {$1$} &  {$\num{2.18}$} & {$\num{0.75}$}  & {$\num{0.80}$}  \\
  {$2$} &  {$\num{3.02}$} & {$\num{0.80}$}  & {$\num{0.87}$}   \\
  {$3$} &  {$\num{3.42}$} & {$\num{0.87}$} &  {$\num{0.96}$}  \\
      \toprule
    { } &  {$T_k$ in $\si{\kelvin}$} & {$T_w$ in $\si{\kelvin}$} &{$T_m$ in $\si{\kelvin}$}  \\
    \midrule
    {$1$} &  {$\num{327.00}$} & {$\num{291.94}$}  & {$\num{293.25}$}  \\
    {$2$} &  {$\num{349.03}$} & {$\num{293.25}$}  & {$\num{295.01}$}   \\
    {$3$} &  {$\num{359.08}$} & {$\num{295.01}$} &  {$\num{297.27}$}  \\
  \end{tabular}
  \caption{Spannungen und Temperaturen aus der Messung mit Graphit}
  \label{tab: graphit}
\end{table}


\begin{table}
  \centering
  \begin{tabular}{S S S S}
    \toprule
  {Durchgang} &  {$U_k$ in $\si{\milli \volt}$} & {$U_w$ in $\si{\milli \volt}$} &{$U_m$ in $\si{\milli \volt}$}  \\
  \midrule
  {$1$} &  {$\num{3.97}$} & {$\num{0.73}$}  & {$\num{0.80}$}  \\
  {$2$} &  {$\num{4.00}$} & {$\num{0.80}$}  & {$\num{0.85}$}   \\
  {$3$} &  {$\num{3.90}$} & {$\num{0.85}$} &  {$\num{0.91}$}  \\
      \toprule
    { } &  {$T_k$ in $\si{\kelvin}$} & {$T_w$ in $\si{\kelvin}$} &{$T_m$ in $\si{\kelvin}$}  \\
    \midrule
    {$1$} &  {$\num{372.90}$} & {$\num{291.49}$}  & {$\num{293.25}$}  \\
    {$2$} &  {$\num{373.65}$} & {$\num{293.25}$}  & {$\num{294.51}$}   \\
    {$3$} &  {$\num{371.14}$} & {$\num{294.50}$} &  {$\num{296.01}$}  \\
  \end{tabular}
  \caption{Spannungen und Temperaturen aus der Messung mit Zinn}
  \label{tab: zinn}
\end{table}



\begin{table}
  \centering
  \begin{tabular}{S S S}
    \toprule
   {$U_k$ in $\si{\milli \volt}$} & {$U_w$ in $\si{\milli \volt}$} &{$U_m$ in $\si{\milli \volt}$}  \\
  \midrule
  {$\num{3.97}$} & {$\num{0.75}$}  & {$\num{0.82}$}  \\
      \toprule
      {$T_k$ in $\si{\kelvin}$} & {$T_w$ in $\si{\kelvin}$} &{$T_m$ in $\si{\kelvin}$}  \\
    \midrule
      {$\num{372.99}$} & {$\num{291.99}$}  & {$\num{293.75}$}  \\
  \end{tabular}
  \caption{Spannungen und Temperaturen aus der Messung mit Aluminium}
  \label{tab: alu}
\end{table}

In Tabelle \ref{tab: konstanten} sind die für die Berechnung der spezifischen Wärmekapazität bei konstantem Volumen notwendigen Größen aufgeführt. Von den gemessenen
Massen wurde das Gewicht der Aufhängung abgezogen.
\begin{table}
  \centering
  \begin{tabular}{l S S S S S}
      \toprule
    {Stoff} &  {$m_k$ in $\si{\kelvin}$} & {$\rho$ in $\si{\gram \per \centi\meter^3}$} & {$M$ in $\si{\gram \per \mol}$} & {$\alpha$ in $\si{10^{-6}\kelvin^{-1}}$}
     & {$\kappa$ in $\si{10^{-6}\kelvin^{-1}}$}  \\
      \midrule
    {Graphit} &  108.02 & 2.25 & 12.0 & 8 & 33 \\
    {Zinn} &     232.03 & 7.28 & 118.7 & 27.0 & 55 \\
    {Aluminium} & 114.57 & 2.7 & 27.0 & 23.5 & 75 \\
  \end{tabular}
  \caption{Massen und physikalische Eigenschaften der verwendeten Materialblöcke}
  \label{tab: konstanten}
\end{table}

Zunächst sollen die spezifischen Wärmekapazitäten $c_k$ gemäß Formel \eqref{eq:zusammenhang_ck} aus den Messreihen bestimmt werden. Die Werte für Graphit, Zinn und Aluminium sind in Tabelle \ref{tab: c_v} %formel, Tabelle
dargestellt. Hierbei wurde mit der jeweiligen molaren Masse multipliziert um das Ergebnis bezogen auf ein $\si{mol}$ angeben zu können. Mit dem Zusammenhang \eqref{eq:umrechnung_cp_cv} %formel
kann schließlich die zu untersuchende Größe $C_V$ bestimmt werden. Die Ergebnisse, sowie das Verhältnis $C_V / R$ zur Gaskonstanten R sind in Tabelle \ref{tab: c_v} aufgeführt. %Tabelle
\FloatBarrier
\begin{table} 
 \centering 
 \caption{Testtabelle} 
 \label{tab:some_data} 
 \begin{tabular}{S S } 
 \toprule \\ 
$s$  & $t$ \\ 
\midrule \\ 
 6.00 & 17973.00 \\ 
 6.00 & 17966.00 \\ 
 6.00 & 17968.00 \\ 
 6.00 & 17969.00 \\ 
 6.00 & 17970.00 \\ 
 12.00 & 17966.00 \\ 
 12.00 & 17967.00 \\ 
 12.00 & 17966.00 \\ 
 12.00 & 17963.00 \\ 
 12.00 & 17963.00 \\ 
 18.00 & 17963.00 \\ 
 18.00 & 17963.00 \\ 
 18.00 & 17964.00 \\ 
 18.00 & 17963.00 \\ 
 18.00 & 17964.00 \\ 
 24.00 & 17960.00 \\ 
 24.00 & 17963.00 \\ 
 24.00 & 17960.00 \\ 
 24.00 & 17962.00 \\ 
 24.00 & 17962.00 \\ 
 30.00 & 17957.00 \\ 
 30.00 & 17958.00 \\ 
 30.00 & 17957.00 \\ 
 30.00 & 17958.00 \\ 
 30.00 & 17960.00 \\ 
 \bottomrule 
 \end{tabular} 
 \end{table}
Als Mittelwert mit zugehörigen Standardabweichungen der spezifischen Wärmekapazitäten für Zinn und Graphit ergeben sich: %das hier lieber mit \quad abtrennen anstelle von &
\begin{align}
  C_{V, Graphit} \quad  &= (11.70 \pm 0.41) \si{\joule \per {\kelvin \mol}} \\
  C_{V, Zinn} \quad &= (29.10 \pm 2.26) \si{\joule \per {\kelvin \mol}}
\end{align}
Zur qualtitativen Diskussion der Ergebnisse werden nun abschließend Literaturwerte $C_{V,lit}$ \cite{lit_dichte} der spezifischen Wärmekapazität aufgeführt und die prozentuale
Abweichung der beobachteten Werte von diesen berechnet (Tabelle \ref{tab: comp}).

  \begin{table}
    \centering
    \begin{tabular}{l S S S}
        \toprule
      Stoff  &{$C_V$ in $\si{\joule \per {\kelvin \mol}}$} &  {$C_{V,lit}$ in $\si{\joule \per {\kelvin \mol}}$} &  {$\frac{C_V}{C_{V,lit}}-1$ in $\%$}  \\
        \midrule
      Graphit & 11.70 & 9.01 & 30 \\
      Zinn &  29.10 & 27.30 & 7 \\
      Aluminium & 15.51 & 24.19 & -36  \\

    \end{tabular}
    \caption{Vergleich mit Literaturwerten}
    \label{tab: comp}
  \end{table}
