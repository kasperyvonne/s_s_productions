\section{Auswertung}

Die im Folgenden auftretenden Fehler werden mit dem
Python-Paket \emph{uncertainties}\cite{uncertainties} berechnet.

\subsection{Bestimmung der Wellenlänge}

Die Wellenlänge wird mit Formel \eqref{} berechnet.
Die Ergebnisse sind in Tabelle \ref{tab: tab: messwerte_abstand} aufgelistet.
\begin{table} 
\centering 
\caption{Messergebnisse bei der Abstandsmessung} 
\label{tab: tab: messwerte_abstand} 
\begin{tabular}{S S S } 
\toprule  
{Weg in $\si{\meter}$} &{Anzahl} & {Wellenlänge in $\si{\nano\meter}$}  \\ 
\midrule  
 0.33  & 1024  & 0.000650268\\ 
0.32  & 1022  & 0.000620514\\ 
0.30  & 1028  & 0.000574481\\ 
0.32  & 1026  & 0.000618095\\ 
0.30  & 1021  & 0.000582302\\ 
0.30  & 1032  & 0.000579936\\ 
0.32  & 1034  & 0.000620979\\ 
0.30  & 1021  & 0.000586184\\ 
0.32  & 1026  & 0.000621958\\ 
0.30  & 1019  & 0.000587335\\ 
\bottomrule 
\end{tabular} 
\end{table}

Die Wellenlänge wird gemittelt zu

\begin{equation}
  \label{eq:wellenlaenge}
  \overline{\lambda}=\SI{}{\nano\meter}.
\end{equation}

Die Wellenlänge wird für Folgerechnung verwendet.

\subsection{Untersuchung des Brechungsindex von Luft und Kohlenstoffdioxid}

Die Brechungsindexänderung wird nach \eqref{} berechnet. Dabei wird
für $b=\SI{0.05}{\meter}$ angenommen. Anschließend wird mit Formel \eqref{
} der Brechungsindex bestimmt. Hierbei sind konstanten als $T=\SI{293.15}{\kelvin},\, T_0=\SI{273.15}{\kelvin} \,
\text{und} \, p_0=\SI{1.0132}{\bar}$ gegeben. Die Messergebnisse sind in Tabelle \ref{tab: tab: messwerte_luft}
dargestellt.
\begin{table}
\centering
\caption{Messergebnisse für die Brechungszahl bei Luft.}
\label{tab: tab: messwerte_luft}
\begin{tabular}{S S S S S S }
\toprule
{$p-p'$ in $\si{\bar}$} &{Anzahl} & {$\Delta n \cdot \num{e-4}$} & {$\sigma_{\Delta n} \cdot \num{e-6}$} &{$n$} & {$\sigma_n\cdot  \num{e-6}$}  \\
\midrule
 0.8  & 50  & 30  & 4  & 1.00041  & 5\\
0.8  & 60  & 36  & 5  & 1.00049  & 6\\
0.8  & 43  & 26  & 3  & 1.00035  & 4\\
0.8  & 37  & 22  & 3  & 1.00030  & 4\\
0.8  & 33  & 20  & 2  & 1.00027  & 3\\
0.8  & 33  & 20  & 2  & 1.00027  & 3\\
0.8  & 33  & 20  & 2  & 1.00027  & 3\\
0.8  & 33  & 20  & 2  & 1.00027  & 3\\
0.8  & 32  & 19  & 2  & 1.00026  & 3\\
0.8  & 33  & 20  & 2  & 1.00027  & 3\\
\bottomrule
\end{tabular}
\end{table}

Diese können gemittelt werden zu

\begin{align}
  \label{eq:mittelwerte_luft}
  \begin{aligned}
    \overline{\Delta n}\ua{luft}=\num{}\\
    \overline{n}\ua{luft}=\num{}.
  \end{aligned}
\end{align}

Die Messergebnisse für Kohlenstoff sind in Tabelle \ref{} aufgeführt.

\begin{table} 
\centering 
\caption{Messergebnisse für die Brechungszahl bei Kohlenstoff.} 
\label{tab: tab: messwerte_kohlenstoff} 
\begin{tabular}{S S S S S S } 
\toprule  
{$p-p'$ in $\si{\bar}$} &{Anzahl} & {$\Delta n$} & {$\sigma_{\Delta n}$} &{$n$} & {$\sigma_n$}  \\ 
\midrule  
 0.8  & 50  & 0.00030  & 0.000004  & 1.00041  & 0.000005\\ 
0.8  & 52  & 0.00031  & 0.000004  & 1.00043  & 0.000005\\ 
0.6  & 44  & 0.00027  & 0.000003  & 1.00048  & 0.000006\\ 
0.6  & 41  & 0.00025  & 0.000003  & 1.00045  & 0.000006\\ 
0.5  & 37  & 0.00022  & 0.000003  & 1.00045  & 0.000006\\ 
0.6  & 43  & 0.00026  & 0.000003  & 1.00047  & 0.000006\\ 
0.5  & 33  & 0.00020  & 0.000002  & 1.00042  & 0.000005\\ 
0.5  & 32  & 0.00019  & 0.000002  & 1.00044  & 0.000005\\ 
0.4  & 30  & 0.00018  & 0.000002  & 1.00045  & 0.000006\\ 
0.4  & 26  & 0.00016  & 0.000002  & 1.00043  & 0.000005\\ 
\bottomrule 
\end{tabular} 
\end{table}
Gemittelt ergibt sich:

\begin{align}
  \label{eq:mittelwerte_kohlenstoff}
  \begin{aligned}
    \overline{\Delta n}\ua{kohlenstoff}=\num{}\\
    \overline{n}\ua{kohlenstoff}=\num{}.
  \end{aligned}
\end{align}
