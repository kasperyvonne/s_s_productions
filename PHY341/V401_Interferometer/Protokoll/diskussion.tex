\section{Diskussion}
Im Folgenden sollen die Messergebnisse in Bezug auf die Messgenauigkeit des
Versuchsaufbaus diskutiert werden.
Zu aller erst ist die Empfindlichkeit des Versuchsaufbaus gegenüber Erschütterung %zu aller erst umgangsspracjlich, Erschütterungen
zu erwähnen. Selbst kleinste Erschütterung sorgen für falsche Registrierung von %Begleiter oder plural
Maxima. Die Empfindlichkeit ist Grund für eine signifikante Ungenauigkeit in den
Messergebnissen. Mit Hilfe zum Beispiel einer Wirbelstrombremse, ist es möglich die %schreib besser einfach dämpfung
Erschütterungsempfindlichkeit zu erheblich zu verringern. %zu
Eine weitere Fehlerquelle ist die Fotodiode, da diese bei einem zu schnellen
Wechsel von Maxima und Minima, keinen Impuls an den Verstärker überträgt.

Abschließend ist zu sagen, dass die Messergebnisse, trotz der Empfindlichkeiten
des Versuchsaufbaues, eine qualitative Aussage liefern.

%Diskussion ist zu schwammig. Bezieh dich auf die Ergebnisse.
