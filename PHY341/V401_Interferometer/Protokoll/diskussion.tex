\section{Diskussion}
Im Folgenden sollen die Messergebnisse in Bezug auf die Messgenauigkeit des
Vesuchsaufbaus diskutiert werden.
Zu aller erst ist die Empfindlichkeit des Versuchsaufbau gegenüber Erschütterung
zu erwähnen. Selbst kleinste Erschütterung sorgten für falsche Regrestrierung von
Maxima. Die Empfindlichkeit sorgt für eine signifikante Ungenauigkeit in den
Messergebnissen. Mit Hilfe einer Wirbelstrombremse ist es möglich die
Erschütterungsempfindlichkeit zu verringern.
Eine weitere Fehlerquelle ist die Fotodiode, da diese bei einem zu schnellen
Wechsel von Maxima und Minima kein Impuls an den Verstärker weitergibt.

Abschließend ist zu sagen, dass die Messergebnisse, unter Berücksichtigung
der Ungenauigkeiten, eine qualitative Aussage liefern.
