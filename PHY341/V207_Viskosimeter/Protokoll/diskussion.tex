\section{Diskussion}
In diesem Kapitel sollen die gewonnenn Ergebnise aus der Auswertung diskuttiert werden.
Bei dem Versuch ist die Messung der Zeit die einzge Größe die 
eine signifikante Unsicherheit erzeugt.
Dennnoch ist sie nach 
der Mittlung gering.
Eine weitere Ungenuaigkeit ensteht durch den Versuchaufbau selber.
Denn es ist kaum möglich die Temperatur des Wassers im Viskosimeter 
genau einzustellen. 
Da nicht direkt die Temperatur des destillierten Wasser gemessen wird, sondern die des Wasserbades.
Die Unsicherheit der Ergebnise wird auch erhöht dadurch, dass
die Literaturwerte für Dichte ohne Fehler angegeben waren.
Bei dem Vergleich der Messergebnise mit de Literatur in Abbildung \ref{fig:t_v_v} und \ref{fig:t_v_l_v} wird deutlich das es eine y-Achsenverschiebung gibt.
Dieser lässt sich eventuella auf die nicht so genaue Zeitmessung zurückführen.
Gut zu erkennen in Abbildung \ref{fig:t_v_l_v} ist das die Steigung beider 
Kurven nicht ganz gleich ist.
Dies lässt sich auf die ungenaue Temperaturmessung zurückführen.%Oder ?

Abschleißend bleibt zu sagen, dass der Versuch die Theorie wiederspiegelt.
Die Messergebnise sind also als plausibel einzustufen.