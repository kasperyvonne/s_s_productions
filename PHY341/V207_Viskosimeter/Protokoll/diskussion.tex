\section{Diskussion}
In diesem Kapitel sollen die gewonnen Ergebnisse
aus der Auswertung diskutiert werden.
Bei dem Versuch besitzt die Zeitmessung eine
signifikante Unsicherheit.
Durch die Mittlung der Zeit wird, aber die Aussagekraft des
Experiements sichergestellt.
Eine weitere Ungenauigkeit entsteht durch den Versuchsaufbau selber.
Denn eine genaue Temperaturmessung des Wassers im Viskosimeter 
ist kaum möglich.
Da die Temperatur des destillierten Wassers nicht direkt gemessen wird, 
sondern indirekt durch die Temperatur des Wasserbades.
Die Unsicherheit der Messergebnisse wird weiter erhöht dadurch, dass
die angenommenen Literaturwerte ohne Fehler angegeben sind.
Bei dem Vergleich der Messergebnisse mit de Literatur in Abbildung 
\ref{fig:t_v_v} und \ref{fig:t_v_l_v} wird eine y-Achsenverschiebung deutlich.
Diese lässt sich eventuell auf die ungenaue Zeitmessung bzw. Temperaturmessung zurückführen.
In Abbildung \ref{fig:t_v_l_v} ist gut zu erkennen, dass die Steigung beider 
Kurven gleich ist.
Dies lässt sich auch auf die ungenaue Temperatur- und Zeitmessung zurückführen.%Oder ?

Abschließend bleibt zu sagen, dass der Versuch die Theorie widerspiegelt.
Die Messergebnisse sind also als plausibel einzustufen.