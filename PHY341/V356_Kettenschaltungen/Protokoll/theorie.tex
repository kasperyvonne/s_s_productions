\setcounter{page}{1}
\section*{Zielsetzung}
Die Untersuchung einer Kettenschaltung von LC-Gliedern lässt sich mit der Analogie
zu einem eindimensionalen Festköper motivieren.
Der Versuch ermöglicht somit Einblicke in das Schwingverhalten eines
Festköpers. %redundant

\section{Theorie}

\subsection{Bestimmung der Schwingungsgleichungen für eine $LC$-Kette} %würde nicht $LC$
Die Schwingungsgleichung werden mit der Kirchhoffschen Regel aufgestellt. %plural
Zunächst wird mittels der ersten Kirchhoffschen Regel die Ströme für ein Kettenglied $n$ %singular/plural
bestimmt (vgl. Abbildung \ref{fig:kettenglied}):

\begin{figure}
  \centering
  \includegraphics[width=0.8\textwidth]{bilder/stöme_und_spannnungen_lc.png}
  \caption{Kettenglied $n$} %Zitat
  \label{fig:kettenglied}
\end{figure}

\begin{equation}
\label{eq:kircheins_lc}
I_n-I_{n+1}-I_{n,\map{quer}}=0
\end{equation}

Der Ansatz führt auf die Gleichung:

\begin{equation}
\label{eq:gleichung_lc}
-\omega^2CU\ua{n}+\frac{1}{L}\left(-U\ua{n-1}+2U\ua{n}-U\ua{n+1}\right)
\end{equation}
Diese wird durch
\begin{equation}
\label{eq:loesung_gleichung_lc}
U_n(t)=U_0 \map{e}^{j\omega t} \map{e}^{-jn\theta t}
\end{equation}
gelöst.
Es sei hierbei $\omega$ die Kreisfrequenz und $\theta$, die Phasenverschiebung %kein komma nach theta, dafür nach phasenverschiebung
die durch ein Kettenglied verursacht wird.
Für die Kreisfrequenz ergibt sich die Dispersionsrelation:

\begin{equation}
\label{eq:kreisfrequenz_lc_glied}
\omega^2=\frac{2}{LC}\left(1-\cos(\theta)\right)
\end{equation}

Da \eqref{eq:kreisfrequenz_lc_glied} nur für endlich viele Kettenglieder definert ,ist, liegt $\omega$ maximal im Intervall: %-komma

\begin{equation}
\label{eq:menge_omega_lc_glied}
\omega\in\left[\,0,\frac{2}{\sqrt{LC}}\,\right)
\end{equation}

\subsection{Bestimmung der Schwingungsgleichungen für eine $LC_1C_2$-Kette}
Im Gegensatz zum vorherigen Kapitel, werden bei diesem Aufbau zwei unterschiedliche
Kondensatoren $C_1$ und $C_2$ verwendet. Die Kondensatoren werden
alternierend hintereinander geschaltet (vgl. Abbidldung \ref{fig:alternierende_kette}).

\begin{figure}
  \centering
  \includegraphics[width=0.8\textwidth]{bilder/alternierende_kette.png}
  \caption{$C_1C_2$ Kette}
  \label{fig:alternierende_kette}
\end{figure}


Aufgrund dessen muss \eqref{eq:gleichung_lc} zu dem Gleichungssystem

\begin{align}
\label{eq:lc1c2_gleichungsy}
\begin{aligned}
-\omega^2C_1U_{2n+1}+\frac{1}{L}\left(-U_{2n}+2U_{2n+1}-U_{2n+2}\right)\\
-\omega^2C_2U_{2n}+\frac{1}{L}\left(-U_{2n-1}+2U_{2n}-U_{2n+2}\right)
\end{aligned}
\end{align}
angepasst werden.
Folglich ergeben sich die gleichen Lösungen wie bei \eqref{eq:gleichung_lc}
lediglich einmal mit $2n$ und einmal mit $2n+1$.
Durch anwenden der Cramerschen Regel kann auf die Kreisfrequenz geschlossen werden: %Anwenden
\begin{equation*}
\omega^4-\omega^2\frac{2}{L}\left(\frac{1}{C_1}+\frac{1}{C_2}\right)+\frac{4}{L^2C_1C_2}\left(1-\cos^2(\theta)\right)=0
\end{equation*}
Diese Gleichung kann nach $\omega^2$ aufgelöst werden.
Das Resultat lautet %doppelpunkt
\begin{equation}
\label{eq:omega_ceins_czwei}
\omega_{1,2}^{2}=\frac{1}{L}\left(\frac{1}{C_1}+\frac{1}{C_2}\right)\pm\frac{1}{L}\sqrt{\left(\frac{1}{C_1}+\frac{1}{C_2}\right)-\frac{4\sin^2(\theta)}{C_1C_2}}.
\end{equation}

Es gibt also zwei verschiedene Kreisfrequenzen $\omega_1$ und $\omega_2$.
Trägt man beide Frequenzen in einem Diagramm auf so ergeben sich %komma nach auf
zwei Kurven (vgl. Abbildung \ref{fig: dispersion_LC1C2}).
Zunächst soll die Lösung $\omega_2$ besprochen werden.
Wird diese bei $\theta=0$ betrachtet ist ebenso $\omega=0$.
Mit der Näherung $\sin(\theta)\approx\theta$ ergibt sich dann

\begin{equation}
\label{eq:omega_2_lceins_czwei}
\omega_2\approx\sqrt{\frac{2}{L\left(C_1+C_2\right)}}\theta.
\end{equation}
Die Nullstelle von \eqref{eq:omega_2_lceins_czwei} liegt im Nullpunkt $\theta=0$.
Für $\theta>0$ wächst die kurve monoton an und erreicht bei $\omega_2(\frac{\pi}{2})=\sqrt{\frac{2}{LC_1}}$
ihr Maximum. \newline
Hingegen beläuft sich bei $\omega_1$ die Lösung an der Stelle $\theta=0$ auf
\begin{equation*}
\omega_1(0)\approx\sqrt{\frac{2(C_1+C_2)}{LC_1C_2}}
\end{equation*}
Das Maximum von $\omega_1$ beträgt

\begin{equation}
\label{eq:max_omega_1_ceins_czwei}
\omega_1(\frac{\pi}{2})=\sqrt{\frac{2}{LC_2}}.
\end{equation}
Aufgrund von dem Unterschied der beiden Maxima, gibt es einen Frequenzbereich, in dem %aufgrund des Unterschiedes
die Kettenschaltung undruchlässig wird. Dieser ist auch in auch in Abbildung \ref{fig: dispersion_LC1C2} %-auch
zusehen.

\subsection{Ausbreitungsgeschwindigkeit von Wellen in einer Kettenschaltung}
Bei eine Welle in der Form \eqref{eq:loesung_gleichung_lc} %bei einer Welle der Form
ergibt sich für Ausbreitungsgeschwindigkeit gleicher Phasen %ist unabhängig von der Phase
\begin{equation*}
v\ua{ph}=\frac{\omega}{\theta}.
\end{equation*}
Es wird $v\ua{ph}$ als Phasengeschwindigkeit bezeichnet.
Die Zeitabhängigkeit der Phasengeschwindigkeit ist durch $\omega=\map{const}$
für alle Zeiten $t$ festgelegt. Wellen, die sich mit der Phasengeschwindigkeit
ausbreiten, sind aufgrund ihrer Unbeschränktheit in Raum und Zeit, nicht zur %-komma
Informationsübermittlung geeignet. Hierzu benötigt man Wellenpakete, die sich
mit der Gruppengeschwindigkeit ausbreiten. Auf diese hier nicht eingegangen. %verb
Speziell für den Versuch ergibt sich als Phasengeschwindigkeit:

\begin{equation}
\label{eq:phasen_esc}
v\ua{ph}=\frac{\omega}{\theta}=\frac{\omega}{\arccos\!\left(1-\frac{1}{2}\omega^2LC\right)}
\end{equation}
Dabei geht die Phasengeschwindigkeit im Grenzfall in

\begin{equation*}
\lim_{\omega\to 0}v\ua{ph}=\frac{1}{\sqrt{LC}}
\end{equation*}

über.
\subsection{Der Widerstand einer unendlichen langen Kette}
Da es sich bei der Kettenschaltung um eine reale Schaltung handelt,
besitzt diese einen elektrischen Widerstand $R$.
Der Widerstand kann mittels der Eingangspannung $U_0$ und dem Strom $I_0$ bestimmit werden.
Dazu wird mittels der Kirchhoffschen Knotenregel (vgl. Abbildung \ref{fig:bestimmung_impe}). %-punkt
\begin{figure}
  \centering
  \includegraphics[width=0.8\textwidth]{bilder/eigenimpendanz.png}
  \caption{Spannungen un Ströme in der $LC$-Kette}
  \label{fig:bestimmung_impe}
\end{figure}
\begin{equation*}
I_0 - \map{i}\omega\frac{C}{2}U_0+\frac{U_1-U_0}{\map{i}\omega L}=0
\end{equation*}
bestimmt.
Es sei $\map{i}$ das komplexe i. %merkse selber
Mit dem ohmschen Gesetz erhält man letzendlich %letztendlich

\begin{equation}
R(\omega)=Z(\omega)=\sqrt{\frac{L}{C}}\frac{1}{\sqrt{1-\frac{1}{4}\omega^2LC}}.
\label{eq: wellenwiderstand_LC}
\end{equation}
Es ist üblich $R$ bzw. $Z$ als frequenzabhängigen Wellenwiderstand zu bezeichnen.
Für den Fall einer unendlich langen Kette wäre der Widerstand rein reell, dass bedeutet %das bedeutet, dass es weder zu einer Phasenverschiebung noch zu einer Reflexion kommt.
das es zu keinem Zeitpunkt zu einer Phasenverschiebung noch zu einer Reflexion kommt.
Durch einen Abschlusswiderstand am Ende der Kette ist dieser Zustand auch bei einer endlichen Kette
realisierbar. Der Abschlusswiderstand muss dazu lediglich den Betrag $Z$ besitzen.
Jedoch ist $Z$ frequenzabhängig, doch bei Frequenzen weit unter der Grenzfrequenz, ist
$Z\approx \sqrt{\frac{L}{C}}$. %überarbeite diesen satz nochmal

\subsection{Randverhalten einer endlichen Kette}
Je nach Endwiderstand $r$ kann es bei einer endlichen Kette
zu Wellenreflexion kommen. In der Kette kommt es dann %zur Wellenreflexion oder zu Wellenreflexionen
zu Überlagerung z. B. stehende Wellen kommen. %überarbeite diesen satz nochmal
Für die Spannung bzw. Stromstärken am Kettenende (vgl. Bild \ref{fig:kettenende}) %gilt entweder beides plural oder beides singular

\begin{figure}
  \centering
  \includegraphics[width=0.6\textwidth]{bilder/wellenwiderstand.png}
  \caption{Kettenende}
  \label{fig:kettenende}
\end{figure}

\begin{equation*}
U\ua{R}=U\ua{E}+U\ua{ref} \qquad I\ua{R}=I\ua{E}+I\ua{ref}
\end{equation*}

Es ist $U\ua{E}/I\ua{E}$ die Spannung/Stromstärke der einfallenden und $U\ua{ref}/I\ua{ref}$ die Spannung der reflektierten
Welle. Besitzt der elektrische Aufbau nun den Wellenwiderstand $Z$ so folgt nach dem
ohmschen Gesetz
\begin{equation*}
I\ua{E}=\frac{U\ua{E}}{Z} \qquad I\ua{ref}=\frac{U\ua{ref}}{Z}.
\end{equation*}
Somit ergibt sich das Verhältnis

\begin{equation*}
\frac{U\ua{ref}}{U\ua{e}}=\frac{r-Z}{r+Z}
\end{equation*}

Es gibt drei intereseante Sonderfälle:

\renewcommand{\labelenumi}{\alph{enumi})}
\begin{enumerate}
\item{ $r=\infty$ (offenens Ende) \newline
Betrachten wir ein offenes Ende so ist $U\ua{ref}=U\ua{e}$ und die Welle wird am Kettenende ohne Phasensprung reflektiert.} %offenes Ende,
\item{ $r=0$ (kurzgeschlossenes Ende) \newline
Bei diesem Fall ist $U\ua{ref}=U\ua{e}$. Zusätzlich kommt es bei der reflektierten Welle zu einem Phasensprung von $\pi$.} %in diesem Fall
\item{ $r=Z$ \newline
Ist der Abschlusswiederstand genauso groß wie der Wellenwiderstand, so wird eine unendlich lange Kette simuliert und
es gilt $U_{ref}=0$. Es kommt zu keiner Reflexion.}
\end{enumerate}

In allen anderen Fällen kommt es zu Teilreflexionen.
Bei den Fällen $r=\infty$ und $r=0$ erfolgt eine Totalreflexion.
Bildet sich im Schaltkreis eine stehende Welle aus, so bildet sich ein Spannungsbau genau dann aus, wenn %ch
\begin{equation}
\label{eq:spannungsbauch}
n\ua{max}\theta_k=k\pi \quad k=1,2,\dots,n\ua{max}
\end{equation}
Es sind im Schwingkrei somit maximal $n\ua{max}$ stehende Wellen möglich. %s
Jede diese besitz eine zugehörige Frequenz $\omega_k$. %%überarbeite diesen satz nochmal
