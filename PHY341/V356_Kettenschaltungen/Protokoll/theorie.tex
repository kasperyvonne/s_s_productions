d\setcounter{page}{1}
\section*{Zielsetzung}
Die Untersuchng einer Kettenschaltung von LC-Gliedern lässt sich mit der Analogie  
zu einem eindimensionalen Festköper motivieren.
Der Versuch ermöglicht somit Einblicke in das Schwingverhalten eines 
Festköpers.

\section{Theorie}

\subsection{Bestimmung der Schwingungsgleichungen für eine $LC$-Kette}
Die Schwingungsgleichung werden mit der Kirchhoffschen Regel aufgestellt.
Zunäschst wird mittels der ersten Kirchhoffschen Regel die Stöme für ein Kettenglied $n$ 
bestimmt (vgl. Abbildung \ref{}):

\begin{equation}
\label{eq:kircheins_lc}
I\ua{n}-I\ua{n+1}-I\ua{n,quer}=0
\end{equation}