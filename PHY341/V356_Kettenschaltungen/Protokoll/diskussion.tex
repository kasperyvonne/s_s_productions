\section{Diskussion}
Im Folgenden sollen die Ergebnisse der Messungen interpretiert und in Beziehung zur Präzision des verwendeten Aufbaus gestellt werden. \\
Zunächst zu Bestimmung der Grenzfrequenzen mittels der Auswertung der Durchlasskurven. Der Vergleich
der ermittelten Bestwerte mit den Theoriewerten zeigt, dass die verwendete Methode sehr gute Ergebnisse liefert.
In allen drei Fällen liegt der Theoriewert im Vertrauensbereich der experimentell bestimmten Werte. Eine Minimierung
des Fehlers könnte erreicht werden, indem wesentlich mehr Referenzpunkte des Frequenzsweeps aufgenommen würden. Dies würde
die Abweichungen der Fitparameter \eqref{eq: params_LC}/\eqref{eq: params_LC1C2} verringern. \\
Ähnlich gute Resultate wurden bei der Unterschung der Dispersionsbeziehungen erzielt. Die Graphen \ref{fig: dispersion_LC} und \ref{fig: dispersion_LC1C2}
zeigen, dass sich Experiment- und Theorie-Kurve nur geringfügig voneinander unterscheiden. Gleiches gilt
für die Messung der Phasengeschwindigkeit. Die Bestimmung der Eigenfrequenzen mittels des Millivoltmeters ermöglicht
auch hier eine große Messpräzision [siehe Abbildung \ref{fig: v_phase}].\\
Das Ausbilden von stehenden Wellen auf der offenen LC-Kette ist deutlich in den Abbildungen \ref{fig: U_nu1} und \ref{fig: U_nu2}
zu erkennen. Hierbei handelt es sich um die Grund- bzw. 2. Oberschwingung. Auch im Fall
der abgeschlossenen Kette ist eine stehende Welle mit geringer Amplitude zu beobachten [siehe Graph \ref{fig: U_G}]. Dies erklärt
die in der Auswertung erwähnten Störschwingungen der Durchlasskurven. Zurückzuführen ist der Effekt auf eine unpräzise
Einstellung des Wellenwiderstandes. Nur wenn der Widerstand exakt auf den berechneten
Wert eingestellt ist, verschwinden sämtliche reflektierten Wellen. In anbetracht der kleinen Amplitude der beobachteten
Schwingung, ist jedoch auch dieses Ergebnis im Rahmen der Präzision des verwendeten Aufbaus als plausibel einzustufen.
