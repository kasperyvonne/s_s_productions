\section{Diskussion}
In diesem Kapitel soll eine Art Plausibilitätsprüfung in Bezug auf die
gewonnen Einblicke die durch die Experiemente erfolgt sind folgen.
Dabei beginnen wir zunächst bei der Messung der Schwingungsdauer.
Hier sei direkt der Mensch und seine Reaktionszeit als Fehlerquelle anzumerken. Denn es ist extrem schwierig Anfangs- und Endpunkt einer jeden Zeitmessung exakt und korrekt festzulegen. jedoch sei an dieser Stelle  zuerwähnen, dass eine Mittelung über 5 Messwerte dafür sorgt, dass der so enstehedne Fehler vergleichsweise klein wird.
Der Unterschied zwischen passiver und dynamischer Winkelrichtgröße liegt vorrausichtlich an einigen Fehlerquellen. Zum einen ist das Ablesen der Kraft bei der passiven Methode eher ungenau (grobe Skalierung) und zweitens ist es auch nicht möglich 
die Federwaage genau orthogonal zum Radius zu halten.
Bei der dynamischen Messung spielt die oben angesprochene Zeitmessung eine Rolle.
Dazu kommt auch das die Massen der beiden kleinen Zylinder nicht ganz genau von der Waage gemessen werden können.
 Diese Abweichung ist aber so klein, dass sie wohl kaum eine große Rolle spielt.
Kommt man zur Messung des Eigenträgheitsmoment der Drillachse so fällt sofort auf, dass experimentelle Größen mit theoretisch bestimmeten Größen (Trägheitsmoment der beiden Zylinder) gemischt worden sind. Dies sorgt für ein weitere Abweichung. Der für die Theorie benötigtgten Höhen und Radien der Zylinder konnte dabei auch nicht ganz genau bestimmt werden.
Dazu kommt auch noch die Positionierung der Massen auf dem Stab, denn zum einen ist nicht vollständig sichergestellt, dass diese symetrisch zur Drehachse stehen, noch das ihre Entfernung zur Drehachse genau bestimmt wurde. Aber der Stab selber ist auch eine Fehlerwuelle, denn er wurde in allen theoretischen Berrechnung außer acht gelassen.
Aber auch hier sei, wie die Oben angesprochene Zeitmessung anzumerken, obwohl die Auswirkung auf die Messgröße klein sind.
Die Messung der Trägheitsmoment von Kugel und Zylinder(grau) fiel Unterschiedlich aus.
Bei dem  Zylinder war eine Abweichung von $+33\%$ zur Theorie festzustellen, 
dies lässt sich wahrscheinlicha auf die ungenauen Größe $I_S$ zurückführen.
Hingegen war die Abweichung bei der Kugel $\pm 0\%$, dieser Wert überrascht zeigt, aber auch wie genau mechanische Messung von Hand getätigt werden können.
Die wohl größte Ungenauigkeit fand man bei der Messung des Trägheitsmoment der Puppe. Denn hier wurdern bei der theoretischen Bestimmung große geometrische Vereinfachung vorgenommen, so dass eine Abweichung von $+1200\%$ (Position 1) schon fast plausibel scheint. %kann man das so schreiben ??
Dazu kommt, dass bei der Puppe die Aufhängung (Stange) in allen theoretischen Rechnung vernachlässigt wurde. Mit Stange wäre der Unterschied zwischen Theorie und Praxis geringer.
Abschließend ist zu sagen, dass bei nicht Berachtung der Ergebnise der Puppe, auf Grund imenser Approximationen, ein Zusammenhang zwischen Theorie und den experiementellen Befunden plausibel scheint.