\section{Auswertung}

Zunächst wird im Versuch die Winkelrichtgröße $D$ und das
das Eigenttägheitsmoment der Drillachse $I_D$ bestimmt.
Anschließend die Trägheitsmoment einer Kugel und eines Zylinders (\emph{grau}).
Dazu komme noch die Trägheitsmomente von einer Holzpuppe in zwei verschieden Positionen.
Da das Trägheitsmoment nicht direkt gemessen werden kann, wurden immer 
die Schwingungsdauer $T$ von $5$ Schwingungen gemessen.
Um so dann auf das Trägheitsmoment zu schließen.

\subsection{Bestimmung der Winkelrichtröße $D$}

Die Winkelrichtgröße $D$ kann auf zwei verschiedene Wege bestimmt werden.
Zum einem durch eine passive und zum andern durch eie  dynamische Methode.
Beide Methoden werden im Anschluss genauer erklärt werden.

\subsubsection{Passive Methode}

Die passive oder auch als statisch bezeichnete Methothde wird mittels einem Drehmoment (gemessen mit einer Federwaage) 
das für eine Auslenkung der Torsionsfeder sorgt gemessen.
Wichtig dabei ist, dass die Federwaage immer orthogonal zum Radius steht.
Denn das Drehmoment errechnet sich mit $\vec{M}=\vec{r}\times\vec{F}$, durch 
eine orthogonale Stellung von $\vec{r}$ und $\vec{F}$ kann mit den Beträgen 
gerechnet werden und ergibt den Zusammenhang $M=rF$. 
Im Versuch wurden folgende Werte bestimmt:


\begin{table}
\centering
\caption{Messung der Kraft für eine Auslenkung}
\label{tab: winkelricht}
\renewcommand{\arraystretch}{1.2}
\begin{tabular}{lcr}
	\toprule
	Abstand in $\si{\meter}$ & Winkel in $\mathrm{rad}$ & Kraft in $\si{\newton}$ \\
	\midrule
	$\num{1.01e-1}$ & $\frac{\pi}{6}$ & $\num{1.40e-1}$ \\
	$\num{1.01e-1}$ & $\frac{\pi}{3}$ & $\num{2.40e-1}$ \\
	$\num{1.01e-1}$ & $\frac{\pi}{2}$ & $\num{3.20e-1}$ \\
	$\num{1.01e-1}$ & $\frac{2\pi}{3}$ & $\num{4.60e-1}$ \\
	$\num{1.58e-1}$ & $\frac{\pi}{6}$ & $\num{6e-2}$ \\
	$\num{1.58e-1}$ & $\frac{\pi}{3}$ & $\num{1.2e-1}$ \\
	$\num{1.58e-1}$ & $\frac{\pi}{2}$ & $\num{2.20e-1}$ \\
	$\num{1.58e-1}$ & $\frac{2\pi}{3}$ & $\num{3e-1}$ \\
	$\num{2.41e-1}$ & $\frac{\pi}{6}$ & $\num{2.00e-2}$ \\
	$\num{2.41e-1}$ & $\frac{\pi}{3}$ & $\num{8.00e-2}$ \\
	$\num{2.41e-1}$ & $\frac{\pi}{2}$ & $\num{1.2e-1}$ \\
	$\num{2.41e-1}$ & $\frac{2\pi}{3}$ & $\num{2.00e-2}$ \\
	\bottomrule
\end{tabular}
\end{table}

Mit dem Zusammenhang

\begin{equation*}
D=\frac{M}{\phi}=\frac{Fr}{\phi}
\end{equation*}

ergibt sich dann, für jede Einzelne der insgesamt $n$ Messungen ein Wert für die Winkelrichtgröße.
Diese wurden anschließend durch

\begin{equation}
\label{eq:mittel}
\bar{x}=\frac{1}{n}\sum_{i=1}x_i
\end{equation}

gemittelt, mit der dazugehörigen Abweichung

\begin{equation}
\label{eq:stand_ab}
\bar{\sigma}_{\bar{x}}=\sqrt{\frac{1}{n(n-1)}\sum_{i=1}^{n}(x_i-\bar{x})^2}.
\end{equation}

Es ergibt sich dann für die Winkelrichtgröße der Wert:

\begin{equation}
\label{eq:winkel_passiv}
D_{passiv}=\left(\num{2.03e-2} \pm \num{1.22e-3}\right)\si{\newton\meter}
\end{equation}

\subsubsection{Dynamische Methode}

Bei der dynamischen Methode wird die Schwingungsdauer von zwei kleinen Zylindern ($m_1$ und $m_2$) die symetrisch zum Mittelpunktes
eines Stabes befestigt werden gemmessen. Durch Verschiebung der Massen, ergibt sich eine Vielzahl von Messwerten, die im Anhang eingesehn
werden können.
Diese wurden anschließend gemittelt mit \eqref{eq:mittel} und \eqref{eq:stand_ab}.
Das Resultat ist in der Tabelle dargestellt:

\begin{table}
\centering
\caption{Gemittelte Schwingungsdauer für die dynamische Methode}
\label{tab: winkel_dynamisc}
%\renewcommand{\arraystretch}{1.2}
\begin{tabular}{lr}
	\toprule
	$\bar{T}$ in $\si{\per\second}$ &  $\sigma_{\bar{T}}$ in $\si{\per\second}$ \\
	\midrule
	\num{2.52} & \num{6.43e-3} \\
	\num{3.23} & \num{9.16e-3} \\
	\num{3.65} & \num{5.89e-3} \\
	\num{4.08} & \num{3.57e-3} \\
	\num{4.56} & \num{6.82e-3} \\
	\num{5.09} & \num{2.83e-2} \\
	\num{5.56} & \num{3.33e-2} \\
	\num{6.05} & \num{1.07e-2} \\
	\num{6.59} & \num{8.64e-3} \\
	\num{7.07} & \num{3.31e-3} \\
	\bottomrule
\end{tabular}
\end{table}

Mit den Schwingungsdauern und dem Gesetz

\begin{equation*}
T=2\pi\sqrt{\frac{I}{D}} \quad \Leftrightarrow \quad I=\frac{T^2 D}{4\pi^2}
\end{equation*}

kann auf die Winkelrichtgröße geschlossen werden.
Da die Zylinder von der Drehachse verschoben sind, macht man sich den Satz von Steiner \eqref{eq: steiner} zu nutze und erhält den Zusammenhang:

\begin{equation}
\label{eq:gerade}
\frac{T^2D}{4\pi^2}=I_s+ma^2\quad \Leftrightarrow \quad T^2=\frac{I_s 4\pi^2}{D}+\frac{4m_g\pi^2}{D}a^2
\end{equation}

Dabei sei $m_g=m_1+m_2=\num{222.51e-2}\si{\meter}+\num{223.46e-2}\si{\meter}$.
Offensichtlich ergibt sich eine Gleichung für eine Gerade, die mittels Linearer Regression angenährt werden kann.
Dazu verwendet man aus der Regressionsrechnung
die Annöherung um diu Steigung $m$ 

\begin{equation*}
m=\frac{\bar{xy}-\bar{x}\bar{y}}{\bar{x^2}-\bar{x}^2}
\end{equation*}

 zu bestimmen, mit dem dazugehörigen Fehler

\begin{equation*}
\sigma_m=\sqrt{\frac{\sigma^2}{N(\bar{x^2}-\bar{x}^2)}}.
\end{equation*}

Nach Ausführung der Regressionrechnung ergibt sich als Wert
\begin{equation}
\label{eq: steigung}
m=\left(\num{741.2}\pm\num{3.4}\right) \si{\kilogram\per\newton\meter}.
\end{equation}

Da es sich bei \eqref{eq:gerade} um eine Gerade handelt muss gelten:

\begin{equation*}
m=\frac{4m_g\pi^2}{D}
\end{equation*}

Damit kann wird Winkelrichtgröße dynamisch bestimmt,denn es gilt

\begin{equation*}
D=\frac{4m_g\pi^2}{m}.
\end{equation*}

Er ergibt sich:

\begin{equation}
\label{winkelrichtgroesse_dynamisch}
D_{dynam}=\left(\num{2.38e-2} \pm \num{1.10e-4}\right)\si{\newton\meter}
\end{equation}

Da die Abweichung von $D_{dynam}$ kleiner ist als von $D_{passiv}$ wird 
diese in folgenden Rechnungen genutzt.
Dabei sei ab sofort $D=D_{dynam}$.


\subsection{Bestimmung des Eigenträgheitsmoment $I_D$}

Auch bei der Bestimmung des Eigenträgheitsmoment machen wir uns die 
Geradengleichung \eqref{eq:gerade} und dessen Lineare Regression zunutzen.
Denn durch Betrachtung des $y$-Achsenabschnitt der Geraden kann auf das Eigenträgheitsmoment der Drillachse geschlossen werden.
Dazu

\begin{equation*}
b=\frac{4\pi^2}{D}\left(I_D+I_Z\right) \quad \Leftrightarrow \quad I_D=\frac{D}{4\pi^2}b-I_Z
\end{equation*}

Dabei sei $I_Z$ das Trägheitsmoment der kleinen Zylinder.%Soll ich die noch dahin schreiben ?

Aus der Regressionsrechnung ergibt sich der Zusammenhang

\begin{equation*}
b=\frac{\bar{x^2}\bar{y}-\bar{x}\bar{xy}}{\bar{x^2}-\bar{x}^2}
\end{equation*}

dür den $y$-Achsenabschnitt und der dazugehörigen Abweichung

\begin{equation*}
\sigma_b=\sqrt{\frac{\sigma^2}{N\left(\bar{x^2}-\bar{x}^2\right)}}.
\end{equation*}

Rechnerisch ergibt sich:

\begin{equation}
\label{eq:y_achsenabschnitt}
b=\left(\num{4.64}\pm\num{0.12}\right) \si{\meter}
\end{equation}

Mit dem theoretisch errechneten Wert für das Trägheitsmoment der kleinen Zylinder (mittels \eqref{eq:traeg_zylinder_schwer}) ergibt sich für das 
Eigenträgheitsmoment der Drillachse:

\begin{equation}
\label{eq:eigentraegheitsmoment}
I_D=\left(\num{0.003}\pm\num{0.000}\right) \si{\kilogram\meter\squared}
\end{equation}


Abschließend wird der schon benutzte lineare Zusammenhang zwischen $T^2$ und $a^2$ graphisch aufgezeigt.

\begin{figure}
  \centering
  \includegraphics[width=0.9\textwidth]{pics/lineare_regression.pdf}
  \caption{Zusammenhang zwischen $T^2$ und $a^2$}
  \label{fig:zusammenhang_a_T}
\end{figure}


\subsection{Experimentelle Bestimmung des Trägheitsmoment von Zylinder und Kugel}

Um das Trägheitsmoment zu bestimmen wird der Zusammenhang

\begin{equation*}
I_{körper}=\frac{T^2 D}{4\pi^2}-I_D
\end{equation*}
genutzt.

Am Ende jedes Unterkapitels soll ein Vergleich zwischen 
den Trägheitsmomenten die aus der Theorie folgen und den 
experiementellen Trägheitsmomenten folgen.

\subsubsection{Trägheitsmoment des Zylinder}

Die gemessenen Schwingungsdauer sind im Anahng nachzulesen.
Als gemittelter Wert ergibt sich:

\begin{equation*}
\bar{T}_{zylin}=\left(\num{1.17}\pm\num{0.00}\right) \si{\per\second}
\end{equation*}

Damit folgte für das Trägheitsmoment:

\begin{equation}
\label{eq:traeg_zylinder_grau_exp}
I_{zylin}=\left(\num{0.0008}\pm\num{0.0000}\right) \si{\kilogram\meter\squared}
\end{equation}

Für die theoretische Berechnung werden die Maße und Masse des Zylinders benötigt. 
Gemessen wurde:

\begin{align*}
\text{Maße} \quad &\\
d&=\num{3.49e-2}\si{\meter}\\
h&=\num{3.01e-2}\si{\meter}\\
\text{Masse} \quad &\\
m_{zyl}&=\num{1005.8}\si{\gram}
\end{align*}

Da es sich hier um eine Drehung um die Symetrieachse handelt wird zu
theoretischen Berechnung \eqref{eq:traeg_zylinde} genutzt.
Das resultierende Ergebnis lautet:

\begin{equation}
\label{eq:traeg_zylinder_theo}
I_{zylin \,theo}= \num{0.0006}\si{\kilogram\meter\squared}
\end{equation}

Die Abweichung vom experiementellen Resultat zur Theorie beläuft sich auf $\approx +33 \%$

\subsubsection{Trägheitsmoment der Kugel}

Bei der Bestimmung des Trägheitsmoment einer Kugel wird genauso Vorgegangen, wie bei 
der Bestimmung für den Zylinder. Auch hier sind die gemessenen Schwingungsdauer im Anhang einzusehen.
Als gemittelter Wert ergibt sich:

\begin{equation*}
\bar{T}_{kugel}=\left(\num{1.66}\pm\num{0.00}\right) \si{\per\second}
\end{equation*}

Daraus folgt für das Trägheitsmoment:

\begin{equation}
\label{eq:traeg_kugel_exp}
I_{kugel}=\left(\num{0.002}\pm\num{0.000}\right) \si{\kilogram\meter\squared}
\end{equation}

Um das theoretische Trägheitsmoment zu bestimmen werden die Abmessungen und das Gewicht der Kugel benötigt:

\begin{align*}
\text{Abmessung} \quad &\\
d&=\num{13.78e-2}\si{\meter}\\
\text{Gewicht} \quad &\\
m_{zyl}&=\num{812.40}\si{\gram}
\end{align*}

Als theoretische Grundformel wird \eqref{eq:traeg_kugel} genutzt.
Es resultiert 

\begin{equation}
\label{eq:traeg_kugel_theo}
I_{kugel \,theo}= \num{0.002}\si{\kilogram\meter\squared}
\end{equation}

als theoretisches Trägheitsmoment.

Abschließend gilt es noch zu erwähnen das es keinen Unterschied
zwischen theoretischen und experiementellen Trägheitsmoment gibt.

\subsection{Trägheitsmoment der Modellpuppe}

Wie auch bei dem voherigen Körpern wird zu Bestimmung des Trägheitsmoment
die Schwinugngsdauer der Puppe gemessen.

Der wesentliche Unterschied zu den vohrigen Körpern ist, der Vergleich mit den theoretischen Werte. Denn beid er Berrechnung muss die Form der Puppe 
durch bekannkte Körper (Zylinder und Kugeln) angenährt werden.
Da die einzelnen Körperteile nicht an jeder Stelle die gleichen Maße haben,
wurde mehere Werte gemessen (siehen Anhang) und dann gemittelt.
Es ergeben sich folgende Abmessungen für die Körperteile:

\begin{align*}
\text{Abmessung} \quad &\\
r_{kopf}&=\left(\num{0.027}\pm\num{0.002}\right)\si{\meter}\\
r_{arm}&=\left(\num{0.015}\pm\num{0.001}\right)\si{\meter}\\
r_{torso}&=\left(\num{0.035}\pm\num{0.004}\right)\si{\meter}\\
r_{bein}&=\left(\num{0.018}\pm\num{0.001}\right)\si{\meter}\\
\text{Gewicht} \quad &\\
m_{puppe}&=\num{161.90}\si{\gram}
\end{align*}

Für spätere Berechnung wird die Puppe, wie auf der folgenden Abbildung zusehen geometrisch angenährt:


\begin{figure}
  \centering
  \includegraphics[width=0.6\textwidth]{pics/puppe.pdf}
  \caption{Annäherung der Puppe}
  \label{fig:approx_puppe}
\end{figure}

\subsubsection{Position 1}

Die Haltung der Puppe in Position 1 ist in Abbildung \ref{fig:pup1} dargestellt.
Die gemessenen Schwingungsdauer sind im Anahng aufgelistet.
Als Mittelwert ergibt sich:

\begin{equation*}
\bar{T}_{puppe\, p1}=\left(\num{0.64}\pm\num{0.00}\right) \si{\per\second}
\end{equation*}

Daraus resultultiert für das Trägheitsmoment:

\begin{equation}
\label{eq:traeg_puppe_p1}
I_{puppe \,p1}= \left(\num{0.0002}\pm\num{0.0000}\right)\si{\kilogram\meter\squared}
\end{equation}

Bei der Berechnung des theoretischen Trägheitsmoment, macht man sich das additives Verhalten der Trägheitsmomente zu nutzen.
Dabei ist jeodch zu beachten, das bei den Armen und Beinen der Satz von Steiner zu berücksichtigen ist. Dabei sind die Verschiebungen zur Drehachse
\begin{align*}
a_{arm}&=\num{9.27e-2}\si{\meter}\\
a_{bein}&=\num{1.19e-2}\si{\meter}
\end{align*}

Desweiteren wird für den Satz von Steiner die Masse von Armen und Beinen benötigt.
Diese werden bestimmt indem man das Teilvolumen von Arm und Bein zum Gesamtvolumen bestimmt. 
Das Teilvolumen wird dann anschließend mit der Masse der Puppe multipliziert.

Für einen Arm ergibt sich
\begin{align}
\begin{aligned}
\label{eq:masse_arm}
\text{prozentualer} Anteil \quad &\left(10\pm\num{1}\right)\% \\
\text{Masse Arm} \quad &\left(\num{0.016}\pm\num{0.002}\right)\si{\kilogram}
\end{aligned}
\end{align}

und für ein Bein

\begin{align}
\begin{aligned}
\label{eq:masse_bein}
\text{prozentualer} Anteil \quad &\left(16\pm\num{2}\right)\% \\
\text{Masse Bein} \quad &\left(\num{0.026}\pm\num{0.003}\right)\si{\kilogram}.
\end{aligned}
\end{align}

Nach der Theorie hat die Puppe in Position $1$ ein Trägheitsmoment von

\begin{equation*}
\bar{T}_{puppe\, p1\,theo}=\left(\num{1.55e-5}\pm\num{0.27e-5}\right) \si{\kilogram\meter\squared}.
\end{equation*}

Der Unterschied zwischen Theorie und Praxis liegt bei $\approx +1200 \%$.

\subsubsection{Position 2}
Die Haltung der Puppe in Position 2 ist in Abbildung \ref{fig:pup2} zusehen.
Die gemessenen Schwingungen sind im Anhang abgedruckt.
Als gemittelte Schwingungsdauer errechnet sich:

\begin{equation*}
\bar{T}_{puppe\, p2}=\left(\num{0.91}\pm\num{0.00}\right) \si{\per\second}
\end{equation*}

Damit folgt für das Trägheitsmoment

\begin{equation}
\label{eq:traeg_puppe_p2}
I_{puppe \,p2}= \left(\num{0.0005}\pm\num{0.0000}\right)\si{\kilogram\meter\squared}
\end{equation}

Da bei dieser Position nur die Beine ihre Position geändert haben, ändert sich auch nur ihr Abstand zu Drehachse auf $a_{beine}=\num{6.77e-2}\si{\meter}$.
Sonst bleiben Massen und Abstände gleich.

Für das theoretische Trägheitsmoment folgt

\begin{equation*}
\bar{T}_{puppe\, p2\,theo}=\left(\num{0.0002}\pm\num{0.0000}\right) \si{\kilogram\meter\squared}.
\end{equation*}

Die Abweichung von dem experimentellen und theoretischen Ergebnis ist 
somit $\approx +250 \%$.


