\section*{Zielsetzung}
Im Versuch 302 sollen, unter Verwendung sogenannter Brückenschaltungen, die Wirk- und Blindwiderstände
 elektrischer Bauteile experimentell bestimmt werden.
\section{Theorie}
Zur Berechnung unbekannter Größen in einem elektrischen Stromkreis werden die Kirchhoffschen Regeln
benötigt. Die erste Kirchhoffsche Regel (Knotenregel) sagt aus, dass die Summe aller zu- und abfließenden
Ströme in einem Verzweigungspunkt verschwindet. Der Zusammenhang ergibt sich unmittelbar aus der Ladungserhaltung.
\begin{equation}
  \sum_k I_k = 0
  \label{eq: knoten}
\end{equation}

Die zweite Kirchhoffsche Regel besagt, dass in einem in sich geschlossenen Stromkreis
die Summe aller Quellspannungen der Summe aller abfallenden
Spannungen enspricht. Die Regel ist ein Resultat aus der Energieerhaltung.
\begin{equation}
  \sum_k U_{Q,k} = \sum_k I_k \cdot R_k
  \label{eq: maschen}
\end{equation}
Der grundsätzliche Aufbau einer Brückenschaltung ist in Abbildung x dargestellt. Regelt man die Brückenspannung
zwischen den Punkten A und B auf null, so ergeben sich mit \eqref{eq: knoten} und \eqref{eq: maschen}
für allgemeine komplexe Widerstände $Z_i = X_i + j \cdot Y_i$ die
beiden Bedingungen:
\begin{align}
  \begin{aligned}
    X_1 X_4 - Y_1 Y_4 &= X_2 X_3 - Y_2 Y_3 \\
    X_1 Y_4 + X_4 Y_1 &= X_2 Y_3 + X_3 Y_2
  \end{aligned}
  \label{eq: widerstandsbedingungen}
\end{align}
