\section{Diskussion}
In diesem Abschnitt soll die Aussagekraft der Ergebnisse, unter Berücksichtigung der  %die
Rahmbedingungen diskutiert werden. %stil

Hierbei sei direkt die Annahme der fehlerfreien Bauelemente zu erwähnen.
In der Realität ist dies nicht umsetzbar.
Eine nicht Betrachtung der Toleranzen sorgt demnach für ein
weniger repräsentatives Ergebnis.
Eine weitere Fehlerquelle ist das ungenaue Ablesen am Oszilloskop und Multimeter.
Insbesondere das Oszilloskop sorgt dabei für eine Ungenauigkeit.
Denn auf Grund eines Defektes war es nicht möglich die genauste Skala zu nutzen.
Dennoch zeigt der Versuch relativ gute Ergebnisse.
Die in Auswertung bestimmten Werte haben, bis auf einen (vgl. \eqref{eq:wert_r16_l_16}), eine geringe Abweichung. %arrrsch
Die Messung mit Brückenschaltung erlauben anscheinend relativ Genaues ausmessen von Bauelementen. %genaues Ausmessen. %sinn
Beim Vergleich der Ergebnise \eqref{eq:wert_r16_l_16} mit \eqref{eq:wert_lr_16_max}, %Ergebnisse
fällt der große Fehler der Kapazitätsmessbrücke bei der Berechnung des Widerstands
auf. Es empfiehlt sich also eher die Maxwell Brücke zu verwenden. %-,
Bei der Wien-Robinson-Brücke ist zu erwähnen, dass der in Abbildung \ref{fig: plot}
zu sehende Abfall (im Plot rechts) wahrscheinlich auf den Tiefpass zurückgeführt werden
kann.

Die abschließende Betrachtung der Ergebnisse zeigt,
dass Brückenschaltungen eine relativ gute Möglichkeit bieten um elektrische Bauelemente %bieten, stil
auszumessen.
