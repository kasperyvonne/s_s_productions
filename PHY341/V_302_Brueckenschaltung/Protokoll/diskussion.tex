\section{Diskussion}
In diesem Abschnitt soll die Aussagekraft der Ergebnise, bezogen auf den
Rahmbedingungen des Versuches verfolgen.

Hierbei sei direkt Annahme der fehlerfreien Messgrößen zu erwähnen.
Da eine fehlerfreie Produktion in der Realität nicht gewährleistet ist, 
sinkt die Qualität der Ergebnise.
Des Weiteren sind das ungenaue ablesen am Ozilliskop und Multimeter eine 
weitere Fehlerquelle.
Insbesondere das Ozilliskop beinträchtigt die Genauigkeit des Versuches.
Denn es war nicht möglich, aufgrund eines Defekts, auf die genauste 
Skala zu schalten.
Dennoch zeigt der Versuch relativ gute Ergebnise. 
Die in Auswertung bestimmten Werte haben, bis auf einen (vgl. \eqref{eq:wert_r16_l_16}), eine geringe Abweichung.
Die Messung mit Brückenschaltung erlauben anscheined relativ genaue Bestimmungen.
Innerhalb der Brückenschaltung gibt es weitere Unterschiede.
Beim Vergleich der Ergebnise \eqref{eq:wert_r16_l_16} mit \eqref{eq:wert_lr_16_max}, 
so fällt der große Fehler, der Kapazitätsmessbrücke, bei der Berechnung des Widerstands
auf. Es empfihelt sich also die Maxwell-Brücke zu bevorzugen.
Zu der Wien-Robinson-Brück bleibt zu sagen, dass der in Abbildung \ref{fig: plot}
zu sehende Abfall (im Plot rechts) wahrscheinlich auf den Tiefpass zurückgeführt werden
kann.

Die abschließende Betrachtung der Ergebnise zeigt, unter Berücksichtigung der Ungenauigkeit, 
dass Brückenschaltung eine gut Möglichkeit bieten um elektrische Bauelemente 
auszumessen.