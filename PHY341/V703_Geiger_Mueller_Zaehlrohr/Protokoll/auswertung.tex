\section{Auswertung}


\subsection{Untersuchung der Charakteristik des Zählrohres}
Die aufgenommen Messwerte für die Zählrate $N$ unter variabler Betriebspspannung $U$ sind
in Tabelle \ref{tab: zaelrate_strom} aufgeführt. Der angegebene Fehler wurde hierbei stets
auf eine ganze Zahl gerundet. Eine Darstellung der Messwerte befindet sich in Abbildung \ref{fig: zählrate_ges}.
Als obere und untere Schranke für die Betrachtung des Plateaus wurden die Werte zwischen $U = \SI{370}{\volt}$
und $U = \SI{600}{\volt}$ gewählt, da an den Grenzen dieses Intervalls ein signifikanter Unterschied zu einem
linearen Verlauf zu erkennen ist (vgl. die beiden benachbarten Werte in Abbildung \ref{fig: zählrate_ges}). Mittels linearer %\ref
Regressionsrechnung ergeben sich für die Steigung $m$ und den Ordinatenabschnitt $b$ folgende Werte
\begin{align}
  m &= \SI{0.041 \pm 0.009}{(\second\volt)^-1} \\
  b &= \SI{412 \pm 4}{\per\second}
\end{align}%.
Eine graphische Darstellung der Messwerte im gewählten Plateaubereich, sowie die Regressiongerade sind in
Abbildung \ref{fig: plateau} einzusehen.

\subsection{Bestimmung der Totzeit}
Mit Hilfe des Oszilloskops wurde die Totzeit fünf mal unter verschiedenen Betriebsspannungen ausgemessen. Die Messwerte
sind in Tabelle \ref{tab: totzeit_oszi} aufgetragen. Für den Mittelwert mit zugehöriger Standardabweichung ergibt sich
\begin{equation}
  T_1 = \SI{96 \pm 2}{\micro\second}.
\end{equation}
Für die Anwendung der Zwei-Quellen-Methode wurden jeweils mit der Betriebsspannung $U = \SI{480}{\volt}$ die
Impulszahlen $Z\ua{i}$ der Strahlung zweier einzelner Proben ($N_1,\, N_2$) und der Gesamtimpulszahl $Z\ua{1+2}$ aufgenommen.
\begin{align}
  Z_1 &= \num{2.57(2)e+04}\\
  Z_2 &= \num{1.11(3)e+03}\\
  Z_{1+2} &= \num{2.68(2)e+04}.
\end{align}%., da kann man auch num verwenden?
Mit dem Messintervall $\Delta t = \SI{60}{\second}$ ergeben sich somit für die Zählraten
\begin{align}
  N_1 &= \SI{428 \pm 3}{\per\second}\\
  N_2 &= \SI{19(0) }{\per\second}\\
  N_{1+2} &= \SI{446 \pm 3}{\per\second}.
\end{align}%.
Auch hier wurde wieder auf eine ganze Zahl gerundet. Im Rahmen der Fehlertoleranz wird die Ungleichung %Ungleichung
\eqref{eq:totzeit_summe} als erfüllt angenommen. Einsetzen der ungerundeten Werte in Gleichung \eqref{eq:totzeit} liefert damit die Totzeit %eqref{}
\begin{equation}
  T_2 = \SI{0.3(24)e+02}{\micro\second}.
  \label{eq: res_totzeit}
\end{equation}
\begin{table} 
\centering 
\caption{Mit dem Oszilloskop gemessene Totzeiten $T$ unter verschiedenen Beschleunigungsspannungen $U$.} 
\label{tab: totzeit_oszi} 
\begin{tabular}{S S } 
\toprule  
{$U/ \si{\volt}$} & {$T/ \si{\micro\second}$}  \\ 
\midrule  
 360  & 100\\ 
400  & 90\\ 
460  & 90\\ 
500  & 100\\ 
540  & 100\\ 
\bottomrule 
\end{tabular} 
\end{table}
\begin{table} 
\centering 
\caption{Mit dem Oszilloskop gemessene Erholungszeiten $T_E$ unter verschiedenen Beschleunigungsspannungen $U$.} 
\label{tab: erholungszeit} 
\begin{tabular}{S S } 
\toprule  
{$U/ \si{\volt}$} & {$T_E/ \si{\micro\second}$}  \\ 
\midrule  
 520  & 300\\ 
540  & 250\\ 
570  & 250\\ 
630  & 300\\ 
640  & 300\\ 
\bottomrule 
\end{tabular} 
\end{table}

\subsection{Bestimmung der Erholungszeit}
Für die Erholungszeit $T\ua{E}$ wurden die in Tabelle \ref{tab: erholungszeit} dokumentierten Werte mit dem Oszilloskop ausgemessen. Als %\ref{}
Mittelwert mit Standardabweichung berechnet sich
\begin{equation}
  T_E = \SI{2.8(1)e+02}{\micro\second}.
\end{equation}

\subsection{Bestimmung der pro Teilchen vom Zählrohr freigesetzten Ladungsmenge}
Die gemessenen mittleren Ströme $I$, sowie die gemäß \eqref{eq:lafung_pro_teilchen} berechneten Ladungsmengen $\Delta Q$, die pro %\eqref{}
Teilchen im Zählrohr freigesetzt werden, sind in Tabelle \ref{tab: zaelrate_strom} eingetragen. In Abbildung
\ref{fig: ladung} befindet sich eine Auftragung der Ladungsmenge gegen die Beschleunigungsspannung $U$.


\newpage
\begin{table} 
\centering 
\caption{Gemessene Impulszahlen $Z$ und Ionisationsströme $I$ unter verschiedenen Beschleunigungsspannungen $U$ und berechnete Zählraten $N$ und pro einfallendem Teilchen freigesetzte Ladungsmenge $Q$} 
\label{tab: zaelrate_strom} 
\begin{tabular}{S S S S S S S S } 
\toprule  
{$U/ \si{\volt}$} & x & y & {$I/\si{\micro\ampere}$}  \\ 
\midrule  
 310.00  & 24002.00  & 154.93  & 400.03  & 2.58  & 0.10  & 1.560  & 0.01\\ 
320.00  & 24776.00  & 157.40  & 412.93  & 2.62  & 0.20  & 3.023  & 0.02\\ 
330.00  & 24727.00  & 157.25  & 412.12  & 2.62  & 0.20  & 3.029  & 0.02\\ 
340.00  & 24888.00  & 157.76  & 414.80  & 2.63  & 0.20  & 3.009  & 0.02\\ 
350.00  & 25088.00  & 158.39  & 418.13  & 2.64  & 0.30  & 4.478  & 0.03\\ 
360.00  & 25247.00  & 158.89  & 420.78  & 2.65  & 0.39  & 5.785  & 0.04\\ 
370.00  & 25824.00  & 160.70  & 430.40  & 2.68  & 0.40  & 5.801  & 0.04\\ 
380.00  & 25645.00  & 160.14  & 427.42  & 2.67  & 0.40  & 5.841  & 0.04\\ 
390.00  & 25685.00  & 160.27  & 428.08  & 2.67  & 0.50  & 7.290  & 0.05\\ 
400.00  & 25624.00  & 160.07  & 427.07  & 2.67  & 0.60  & 8.769  & 0.05\\ 
410.00  & 25729.00  & 160.40  & 428.82  & 2.67  & 0.60  & 8.733  & 0.05\\ 
420.00  & 25997.00  & 161.24  & 433.28  & 2.69  & 0.60  & 8.643  & 0.05\\ 
430.00  & 25682.00  & 160.26  & 428.03  & 2.67  & 0.70  & 10.207  & 0.06\\ 
440.00  & 25899.00  & 160.93  & 431.65  & 2.68  & 0.70  & 10.122  & 0.06\\ 
450.00  & 26112.00  & 161.59  & 435.20  & 2.69  & 0.80  & 11.473  & 0.07\\ 
460.00  & 25651.00  & 160.16  & 427.52  & 2.67  & 0.80  & 11.680  & 0.07\\ 
470.00  & 25754.00  & 160.48  & 429.23  & 2.67  & 0.90  & 13.087  & 0.08\\ 
480.00  & 25575.00  & 159.92  & 426.25  & 2.67  & 1.00  & 14.643  & 0.09\\ 
490.00  & 25860.00  & 160.81  & 431.00  & 2.68  & 1.00  & 14.481  & 0.09\\ 
500.00  & 25857.00  & 160.80  & 430.95  & 2.68  & 1.10  & 15.931  & 0.10\\ 
510.00  & 26132.00  & 161.65  & 435.53  & 2.69  & 1.20  & 17.197  & 0.11\\ 
520.00  & 26011.00  & 161.28  & 433.52  & 2.69  & 1.20  & 17.277  & 0.11\\ 
530.00  & 25983.00  & 161.19  & 433.05  & 2.69  & 1.30  & 18.737  & 0.12\\ 
540.00  & 26080.00  & 161.49  & 434.67  & 2.69  & 1.40  & 20.103  & 0.12\\ 
550.00  & 25786.00  & 160.58  & 429.77  & 2.68  & 1.40  & 20.332  & 0.13\\ 
560.00  & 26138.00  & 161.67  & 435.63  & 2.69  & 1.40  & 20.058  & 0.12\\ 
570.00  & 26227.00  & 161.95  & 437.12  & 2.70  & 1.50  & 21.418  & 0.13\\ 
580.00  & 26214.00  & 161.91  & 436.90  & 2.70  & 1.60  & 22.857  & 0.14\\ 
590.00  & 26455.00  & 162.65  & 440.92  & 2.71  & 1.60  & 22.649  & 0.14\\ 
600.00  & 26306.00  & 162.19  & 438.43  & 2.70  & 1.60  & 22.777  & 0.14\\ 
610.00  & 26693.00  & 163.38  & 444.88  & 2.72  & 1.80  & 25.253  & 0.15\\ 
620.00  & 26900.00  & 164.01  & 448.33  & 2.73  & 1.80  & 25.059  & 0.15\\ 
630.00  & 26755.00  & 163.57  & 445.92  & 2.73  & 1.80  & 25.195  & 0.15\\ 
640.00  & 26951.00  & 164.17  & 449.18  & 2.74  & 1.90  & 26.401  & 0.16\\ 
650.00  & 27020.00  & 164.38  & 450.33  & 2.74  & 2.00  & 27.720  & 0.17\\ 
660.00  & 27350.00  & 165.38  & 455.83  & 2.76  & 2.00  & 27.385  & 0.17\\ 
670.00  & 27317.00  & 165.28  & 455.28  & 2.75  & 2.10  & 28.789  & 0.17\\ 
680.00  & 27883.00  & 166.98  & 464.72  & 2.78  & 2.10  & 28.205  & 0.17\\ 
690.00  & 28075.00  & 167.56  & 467.92  & 2.79  & 2.20  & 29.346  & 0.18\\ 
700.00  & 28383.00  & 168.47  & 473.05  & 2.81  & 2.30  & 30.347  & 0.18\\ 
\bottomrule 
\end{tabular} 
\end{table}
\begin{figure}
  \centering
  \includegraphics[width = \textwidth]{../Messdaten/plots/all_counts.pdf}
  \caption{Graphische Darstellung der Zählrohrcharakteristik, Abhängigkeit der Zählrate $N$ von der Betriebsspannung $U$.}
  \label{fig: zählrate_ges}
\end{figure}
\begin{figure}
  \centering
  \includegraphics[width = 0.8\textwidth]{../Messdaten/plots/plateau.pdf}
  \caption{Graphische Darstellung des Plateaubereichs.}
  \label{fig: plateau}
\end{figure}
\begin{figure}
  \centering
  \includegraphics[width = 0.8\textwidth]{../Messdaten/plots/ladung.pdf}
  \caption{Graphische Darstellung der Abhängigkeit zwischen Zählrohrspannung $U$ und pro einfallendem Teilchen ausgelöster Ladungsmenge $\Delta Q$.}
  \label{fig: ladung}
\end{figure}
