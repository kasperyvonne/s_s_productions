\section{Auswertung}


\subsection{Untersuchung der Charakteristik des Zählrohres}
Die aufgenommen Messwerte für die Zählrate $N$ unter variabler Betriebspspannung $U$ sind
in Tabelle \ref{tab: zaelrate_strom} aufgeführt. Der angegebene Fehler wurde hierbei stets
auf eine ganze Zahl gerundet. Eine Darstellung der Messwerte befindet sich in Abbildung \ref{fig: zählrate_ges}.
Als obere und untere Schranke für die Betrachtung des Plateaus wurden die Werte zwischen $U = \SI{370}{\volt}$
und $U = \SI{600}{\volt}$ gewählt, da an den Grenzen dieses Intervalls ein signifikanter Unterschied zu einem
linearen Verlauf zu erkennen ist (vgl. die beiden benachbarten Werte in Abbildung \ref{}). Mittels linearer
Regressionsrechnung ergeben sich für die Steigung $m$ und den Ordinatenabschnitt $b$ folgende Werte
\begin{align}
  m &= \SI{0.041 \pm 0.009}{\per\second\per\volt} \\
  b &= \SI{412 \pm 4}{\per\second}
\end{align}
Eine graphische Darstellung der Messwerte im gewählten Plateaubereich, sowie die Regressiongerade sind in
Abbildung \ref{fig: plateau} einzusehen.

\subsection{Bestimmung der Totzeit}
Mit Hilfe des Oszilloskops wurde die Totzeit fünf mal unter verschiedenen Betriebsspannungen ausgemessen. Die Messwerte
sind in Tabelle \ref{} aufgetragen. Für den Mittelwert mit zugehöriger Standardabweichung ergibt sich
\begin{equation}
  T_1 = \SI{96 \pm 2}{\micro\second}.
\end{equation}
\begin{table} 
\centering 
\caption{Mit dem Oszilloskop gemessene Totzeiten $T$ unter verschiedenen Beschleunigungsspannungen $U$.} 
\label{tab: totzeit_oszi} 
\begin{tabular}{S S } 
\toprule  
{$U/ \si{\volt}$} & {$T/ \si{\micro\second}$}  \\ 
\midrule  
 360  & 100\\ 
400  & 90\\ 
460  & 90\\ 
500  & 100\\ 
540  & 100\\ 
\bottomrule 
\end{tabular} 
\end{table}

\begin{table} 
\centering 
\caption{Mit dem Oszilloskop gemessene Erholungszeiten $T_E$ unter verschiedenen Beschleunigungsspannungen $U$.} 
\label{tab: erholungszeit} 
\begin{tabular}{S S } 
\toprule  
{$U/ \si{\volt}$} & {$T_E/ \si{\micro\second}$}  \\ 
\midrule  
 520  & 300\\ 
540  & 250\\ 
570  & 250\\ 
630  & 300\\ 
640  & 300\\ 
\bottomrule 
\end{tabular} 
\end{table}


\newpage
\begin{table} 
\centering 
\caption{Zählraten und Ionisationsströme unter verschiedenen Beschleunigungsspannungen $U$.} 
\label{tab: zaelrate_strom} 
\begin{tabular}{S S S S } 
\toprule  
{$U/ \si{\volt}$} & x & y & {$I/\si{\micro\ampere}$}  \\ 
\midrule  
 310  & 400  & 3  & 0.1\\ 
320  & 413  & 3  & 0.2\\ 
330  & 412  & 3  & 0.2\\ 
340  & 415  & 3  & 0.2\\ 
350  & 418  & 3  & 0.3\\ 
360  & 421  & 3  & 0.4\\ 
370  & 430  & 3  & 0.4\\ 
380  & 427  & 3  & 0.4\\ 
390  & 428  & 3  & 0.5\\ 
400  & 427  & 3  & 0.6\\ 
410  & 429  & 3  & 0.6\\ 
420  & 433  & 3  & 0.6\\ 
430  & 428  & 3  & 0.7\\ 
440  & 432  & 3  & 0.7\\ 
450  & 435  & 3  & 0.8\\ 
460  & 428  & 3  & 0.8\\ 
470  & 429  & 3  & 0.9\\ 
480  & 426  & 3  & 1.0\\ 
490  & 431  & 3  & 1.0\\ 
500  & 431  & 3  & 1.1\\ 
510  & 436  & 3  & 1.2\\ 
520  & 434  & 3  & 1.2\\ 
530  & 433  & 3  & 1.3\\ 
540  & 435  & 3  & 1.4\\ 
550  & 430  & 3  & 1.4\\ 
560  & 436  & 3  & 1.4\\ 
570  & 437  & 3  & 1.5\\ 
580  & 437  & 3  & 1.6\\ 
590  & 441  & 3  & 1.6\\ 
600  & 438  & 3  & 1.6\\ 
610  & 445  & 3  & 1.8\\ 
620  & 448  & 3  & 1.8\\ 
630  & 446  & 3  & 1.8\\ 
640  & 449  & 3  & 1.9\\ 
650  & 450  & 3  & 2.0\\ 
660  & 456  & 3  & 2.0\\ 
670  & 455  & 3  & 2.1\\ 
680  & 465  & 3  & 2.1\\ 
690  & 468  & 3  & 2.2\\ 
700  & 473  & 3  & 2.3\\ 
\bottomrule 
\end{tabular} 
\end{table}
\begin{figure}
  \centering
  \includegraphics[width = 0.8\textwidth]{../Messdaten/plots/all_counts.pdf}
  \caption{Graphische Darstellung der Zählrohrcharakteristik, Abhängigkeit der Zählrate $N$ von der Betriebsspannung $U$.}
  \label{fig: zählrate_ges}
\end{figure}
\begin{figure}
  \centering
  \includegraphics[width = 0.8\textwidth]{../Messdaten/plots/plateau.pdf}
  \caption{Graphische Darstellung des Plateaubereichs.}
  \label{fig: plateau}
\end{figure}
