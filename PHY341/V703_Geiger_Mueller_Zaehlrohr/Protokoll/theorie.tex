\setcounter{page}{1}
\section*{Zielsetzung}

In dem Versuch V703 sollen die Eigenschaften eines %würde den Begriff Charakteristika nicht benutzen, weil er in dem Zusammenhang schon reserviert ist
Geige-Müller-Zählrohrs untersucht werden. Zu diesen gehören die %Geiger-Müller-Zählrohrs, Punkt statt Komma
Totzeit und die Nachentladungen. %so klingt es, als sei die Untersuchung eine Charakteristik

\section{Theorie}

\subsection{Funktionsweise eines Geiger-Müller-Zählrohr}
Ein Geiger-Müller-Zähler besteht aus einem %grundsätzlich unpassend
Kathodenzylinder (negativ geladen) und einem Anodendraht (positiv geladen) (vgl. Abbildung \ref{fig: schematischer_aufbau}).
Zwischen Anode und Kathode befindet sich ein Gasgemisch (bestehend aus einem Edelgas
und Alkohldampf).
\begin{figure}
  \centering
  \includegraphics[width=0.6\textwidth]{bilder/geiger_.png}
  \caption{Schematische Darstellung eines Geiger-Müller-Zählrohrs \cite{anleitung703}.}
  \label{fig: schematischer_aufbau}
  \end{figure}
Durch Anlegen einer Spannung ($300-\SI{2000}{\volt}$) entsteht, im Inneren des Zylinders, ein elektrisches
Feld. Auf der Vorderseite des Geiger-Müller-Zählrohrs %im Inneren
befindet sich ein Eintrittsfenster aus Mylar. %Mylar ist nicht das material
Die Mylarfolie erleichter der Strahlung in das Zählrohr einzutreten.

Gelangt die  Strahlung in das Zählrohr, so ionisiert diese das Gasgemisch. %redundant
Die freigeschlagenen Elektronen können, je nach anliegender Spannung, weitere Atome ionisieren. %entweder keine Kommas oder noch ein nach können
Die dadurch entstehende Elektronenlawine (als Townsend-Lawine bezeichnet) gelangt zu der Anode und erzeugt so
einen Strom. Der Strom wird verstärkt und an ein %redundant; mittels eines
Zählgerät weitergeleitet.

Wie oben angesprochen hängt die Anzahl der registrierten
Elektronen von der anliegenden Spannung ab (vgl. Abbildung \ref{fig:teilchen_spannung}).
\begin{figure}
  \centering
  \includegraphics[width=0.6\textwidth]{bilder/diagramm.png}
  \caption{Anzahl der gemessen Teilchen pro Zeit bei unterschiedlichen Spannungen \cite{anleitung703}.} %Spannungen
  \label{fig:teilchen_spannung}
\end{figure}
Bei einer zu geringen Spannung gehen die meisten Elektronen durch Rekombination verloren. Vergrößert man die Spannung,
erreicht man den \emph{Arbeitsbereich einer Ionisationskammer}, dieser funktioniert nur bei
Strahlungen mit hoher Intensität. Durch eine weitere Erhöhung der Spannung gelangt man
in den sogenannten \emph{Propotionalbereich}. In diesem besitzen die Elektronen
genügend Energie, um ihrerseits weitere Atome ionisieren zu können (auch Stoßionisation genannt).
Der Vorteil des Propotionalbereiches liegt darin, dass hier neben der Teilchenanzahl auch die Energie %darin
einfallender Teilchen gemessen werden kann. Bei zunehmender Spannung %wiederholter
gelangt man in den \emph{Auslösebereich} bzw. den Arbeitsbereich des
Geiger-Müller-Zählers. Die hier auftretenden Elektronenlawinen enthalten %Leerzeichen nach punkt
eine große Anzahl an UV-Photonen. Da sich die Photonen frei im elektrischen Feld bewegen können,
lösen sie im gesamten Zählrohr Townsend-Lawinen aus. %lawinen, überarbeite die Formulierungen in diesem Absatz

Das Auftreten der Elektronlawinen im ganzen Rohr hat eine \emph{Totzeit} zu folge, %kein Komma
in dieser ist es dem Zählrohr nicht möglich neu eindringende Teilchen %bei dieser falsch
zu messen. Die Ursache für die Unempfindlichkeit sind die durch die Elektronenlawine
entstehenden positiven Ionen.
Da die Ionen eine größere Masse als Elektronen besitzen,
bewegen sie sich wesentlich langsamer zur Kathode, als die Elektronen
zur Anode. Hierdurch entsteht ein Gegenfeld, welches das
elektrische Feld von Kathode und Anode abschwächt.
Erst nachdem die Ionen die Kathode erreicht haben, ist eine
Registrierung wieder möglich. Nach der Totzeit schließt sich
eine \emph{Erholungszeit} an. In dieser Zeit erzeugen
die an der Kathode eintreffenden Ionen neue Elektronen.
Die freigesetzen Elektronen sind der Auslöser für schwächere Elektronenlawine (Nachentladung genannt).
Beide Zeiten sind %hervorrufen; genannt
in Abbildung \ref{fig: tot_und_erholungszeit} grafisch dargestellt.
\begin{figure}
  \centering
  \includegraphics[width=0.6\textwidth]{bilder/totzeit.png}
  \caption{Grafische Darstellung von Tot- und Erholungszeit \cite{anleitung703}.}
  \label{fig: tot_und_erholungszeit}
  \end{figure}

Der im Gasgemisch befindliche Alkoholdampf, soll die Erholungszeit verringern. Dies liegt zum einen
daran, dass die Edelgasionen die Alkoholmoleküle ionisieren. Die Alkoholmoleküle %Erklärbar, damit klingt doof
besitzen eine geringere Ionisierungsenergie und können so keine Elektronen aus
der Kathode schlagen. Zum Andern besitzt sie mehr Schwingungsmoden %moden (kein modus)
mit unterschiedlichen Frequenzen. %moden (kein modus).
Dadurch verliert das Alkoholmolekül %Alkohol
weiter Energie. %wegen doof

\subsection{Charakteristika des Zählrohrs}
In dem Arbeitsbereich des Geiger-Müller-Zählers erhält man einen
charakteristischen Verlauf für die regestrierten Teilchen $N$ in Abhängigkeit zur Spannung. %regestrierten, in der Anleitung und in der Auswertung ist N die Zählrate
Diese \emph{Charakteristik} ist in Abbildung \ref{fig: plateau} dargestellt.
\begin{figure}
  \centering
  \includegraphics[width=0.6\textwidth]{bilder/pleateu.png}
  \caption{Vergrößerung des Geige-Müller-Bereichs \cite{anleitung703}.}
  \label{fig: plateau}
\end{figure}
Ab der Spannung $U\ua{E}$ erreicht man den idealen Spannungsbereich des Zählrohrs. %rohrs
Für Spannungen größer als $U\ua{E}$ ist ein
linearer Zusammenhang zwischen $N$ und $V$ erkennbar, dieser Bereich wird auch als
\emph{Plateau} bezeichnet. Bei einem idealen Zählrohr besitzt der
lineare Zusammenhang eine Steigung von null. In der Realität jedoch besitzt
die Gerade eine geringe Steigung. Erklärbar ist die Steigung mit den oben angesprochenen Nachentladungen. %Gerade
Am Ende des Plateau nimmt die
Zahl der Nachentladungen drastisch zu. Wird die Spannung hier noch weiter erhöht, %gewaltig umgangssprachlich
ist eine Zerstörung des Zählrohrs möglich.

%Überarbeite die Theorie nochmal, viele Sätze sind unschön/unpräzise formuliert
