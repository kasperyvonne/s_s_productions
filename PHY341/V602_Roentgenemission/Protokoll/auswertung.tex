\section{Auswertung}
Die in der Auswertung bestimmten Ausgleichsrechnungen werden mit
dem Python Paket \emph{scipy.optimize}\cite{scipy} durchgeführt.
Des Weiteren werden die Fehler und insbesondere die Fehlerfortpflanzungen
mit dem Python Paket \emph{uncertainties}\cite{uncertainties} berechnet.
Zusätzlich werden die benötigten Naturkonstanten dem Python Paket \emph{scipy.constants}\cite{scipy}
entnommen.
\FloatBarrier
\subsection{Untersuchung der Bragg-Bedingung}
Eine Liste mit den gemessenen Werten ist in Tabelle \ref{tab: bragg_test} zu finden und in Abbildung \ref{fig: bragg_plot} abgebildet.
\begin{table} 
\centering 
\caption{Messwerte bei der Untersuchung der Bragg Bedingung.} 
\label{tab: bragg_test} 
\begin{tabular}{S S S S } 
\toprule  
{$\alpha_{\mathrm{Gl}} \, / \, \si{\degree}$} & {$I \, / \, \mathrm{Imp}/\mathrm{s}$} & {$\alpha_{\mathrm{Gl}} \, / \, \si{\degree}$} & {$I \, / \, \mathrm{Imp}/\mathrm{s}$}  \\ 
\midrule  
 26.0  & 41  & 28.1  & 156\\ 
26.1  & 41  & 28.2  & 158\\ 
26.2  & 44  & 28.3  & 165\\ 
26.3  & 43  & 28.4  & 168\\ 
26.4  & 47  & 28.5  & 169\\ 
26.5  & 47  & 28.6  & 156\\ 
26.6  & 53  & 28.7  & 148\\ 
26.7  & 56  & 28.8  & 137\\ 
26.8  & 68  & 28.9  & 125\\ 
26.9  & 75  & 29.0  & 119\\ 
27.0  & 73  & 29.1  & 110\\ 
27.1  & 83  & 29.2  & 105\\ 
27.2  & 91  & 29.3  & 91\\ 
27.3  & 102  & 29.4  & 86\\ 
27.4  & 110  & 29.5  & 69\\ 
27.5  & 122  & 29.6  & 66\\ 
27.6  & 118  & 29.7  & 53\\ 
27.7  & 133  & 29.8  & 43\\ 
27.8  & 140  & 29.9  & 40\\ 
27.9  & 150  & 30.0  & 40\\ 
\bottomrule 
\end{tabular} 
\end{table}
\begin{figure}
  \centering
  \includegraphics[width=0.8 \textwidth]{../Messdaten/bragbed.pdf}
  \caption{Messwerte zur Untersuchung der Bragg-Bedingung.}
  \label{fig: bragg_plot}
\end{figure}
Das Maximum der Messwerte ist gleich dem zweifachen Bragg-Winkel:
\begin{equation}
  \label{eq:bestimmt_bragg_winkel}
  2\theta\ua{bragg}=\SI{28.5}{\degree} \quad \Leftrightarrow \quad \theta\ua{bragg}=\SI{14.25}{\degree}.
\end{equation}
\FloatBarrier

\FloatBarrier
\subsection{Untersuchung des Emissionspektrum von $\ce{Cu}$}
Die aufgenommen Werte sind in Tabelle \ref{tab: emi_cu} aufgelistet und in Abbildung \ref{fig: emission_cu} dargestellt.
Mit Gleichung \eqref{eq: E_beugungswinkel} werden die abgelesenen Winkel in die entsprechende Energie umgerechnet.
\begin{table} 
\centering 
\caption{Messwerte bei der Untersuchung des Emmissionspektrums von $\ce{Cu}$.} 
\label{tab: emi_cu} 
\begin{tabular}{S S S S S S } 
\toprule  
{$\theta \, / \, \si{\degree}$} & {$I \, / \, \mathrm{Imp}/\mathrm{s}$} & {$\theta \, / \, \si{\degree}$} & {$I \, / \, \mathrm{Imp}/\mathrm{s}$} &{$\theta \, / \, \si{\degree}$} & {$I \, / \, \mathrm{Imp}/\mathrm{s}$}  \\ 
\midrule  
 4.00  & 30  & 11.40  & 215  & 18.75  & 119\\ 
4.20  & 27  & 11.60  & 218  & 19.00  & 101\\ 
4.40  & 29  & 11.80  & 201  & 19.20  & 110\\ 
4.60  & 31  & 12.00  & 225  & 19.40  & 121\\ 
4.80  & 34  & 12.20  & 220  & 19.60  & 464\\ 
5.00  & 43  & 12.40  & 212  & 19.75  & 1194\\ 
5.20  & 51  & 12.60  & 220  & 20.00  & 761\\ 
5.40  & 61  & 12.80  & 195  & 20.20  & 169\\ 
5.60  & 64  & 13.00  & 175  & 20.40  & 147\\ 
5.80  & 67  & 13.20  & 166  & 20.60  & 123\\ 
6.00  & 83  & 13.40  & 176  & 20.75  & 120\\ 
6.20  & 87  & 13.60  & 163  & 21.00  & 116\\ 
6.40  & 94  & 13.80  & 162  & 21.20  & 113\\ 
6.60  & 104  & 14.00  & 163  & 21.40  & 122\\ 
6.80  & 118  & 14.20  & 161  & 21.60  & 165\\ 
7.00  & 128  & 14.40  & 154  & 21.75  & 316\\ 
7.20  & 127  & 14.60  & 153  & 22.00  & 4140\\ 
7.40  & 145  & 14.80  & 151  & 22.20  & 3604\\ 
7.60  & 145  & 15.00  & 150  & 22.40  & 333\\ 
7.80  & 159  & 15.20  & 145  & 22.60  & 144\\ 
8.00  & 156  & 15.40  & 146  & 22.75  & 110\\ 
8.20  & 169  & 15.60  & 136  & 23.00  & 92\\ 
8.35  & 168  & 15.80  & 137  & 23.20  & 86\\ 
8.60  & 178  & 16.00  & 140  & 23.40  & 83\\ 
8.80  & 181  & 16.20  & 130  & 23.60  & 67\\ 
9.00  & 188  & 16.40  & 130  & 23.80  & 75\\ 
9.20  & 185  & 16.60  & 117  & 24.00  & 69\\ 
9.35  & 204  & 16.75  & 122  & 24.20  & 67\\ 
9.60  & 204  & 17.00  & 126  & 24.40  & 64\\ 
9.80  & 198  & 17.20  & 114  & 24.60  & 61\\ 
10.00  & 215  & 17.40  & 118  & 24.80  & 60\\ 
10.20  & 211  & 17.60  & 105  & 25.00  & 58\\ 
10.40  & 225  & 17.75  & 119  & 25.20  & 0\\ 
10.60  & 231  & 18.00  & 98  & 25.40  & 54\\ 
10.80  & 211  & 18.20  & 110  & 25.60  & 55\\ 
11.00  & 235  & 18.40  & 104  & 25.80  & 48\\ 
11.20  & 221  & 18.60  & 112  & 26.00  & 45\\ 
\bottomrule 
\end{tabular} 
\end{table}
\begin{figure}
  \centering
  \includegraphics[width=1 \textwidth]{../Messdaten/emission_cu.pdf}
  \caption{Messsung des Emmissionspektrums von $\ce{Cu}$.} %Hingegen passt nicht
  \label{fig: emission_cu}
\end{figure}
In der Abbildung sind die $K_\alpha$ und $K_\beta$ Kanten mit eingezeichnet.
Aus dem Experiment resultieren die folgenden Energien:
\begin{equation}
  \label{eq:k_alpha,k_beta}
  K_\alpha=\SI{8.2}{\kilo\eV} \qquad   K_\beta=\SI{9.1}{\kilo\eV}.
\end{equation}
Die maximale Energie bzw. die minimale Wellenlänge kann durch Umrechnung des minimalen
Winkels $\theta\ua{min}=\SI{4}{\degree}$ angegeben werden:
\begin{equation}
  \label{eq: maximale_energie}
  E\ua{max}=\SI{44.1}{\kilo\eV} \quad \overset{\eqref{eq: photonenenergie}}{\Leftrightarrow} \quad \lambda\ua{min}=\SI{2.81e-11}{\meter}.
\end{equation}
Mit Hilfe der Energien \eqref{eq:k_alpha,k_beta}, können die Abschirmkonstanten
für die einzelnen K- Linien bestimmt werden.
Aus den Formeln \eqref{eq: sigma_1} und \eqref{eq: sigma_2} errechnet sich:
\begin{equation}
   \label{eq:abschirm}
   \sigma_1=3.13 \qquad \sigma_2=12.8.
\end{equation}
Abschließend soll noch die Halbwertsbreite der beiden Peaks untersucht werden.
Dazu wird der Hochpunkt des jeweiligen Peaks mit einem naheliegenden Punkt $\theta'$
durch eine Gerade $g(x)=mx+b$ verbunden. Mit dieser Geraden wird dann der Punkt $\theta\ua{half}$
bestimmt, wo die Maximalwert des Peaks gerade halbiert ist. Damit kann dann die Halbwertsbreite
berrechnet werden:
\begin{equation}
  \label{eq:halbwertsbreite}
  \sigma\theta=\theta\ua{max}-\theta\ua{half} \qquad \theta\ua{half}=\frac{0.5*\map{Imp}\ua{max}-b}{m}.
\end{equation}
Die resultierenden Geraden sind in Abbildung \ref{fig: halbwert} zu erkennen.
\begin{figure}
  \centering
  \includegraphics[width=0.8\textwidth]{../Messdaten/emission_cu_zoom.pdf}
  \caption{Bestimmung der Halbwertsbreite von $K_\alpha$ und $K_\beta$ mit Hilfe von Geraden.} %Hingegen passt nicht
  \label{fig: halbwert}
\end{figure}
Die Ergebnisse sind in Tabelle \ref{tab: Halbwertsbreite} aufgelistet.
\begin{table}
  \centering
  \caption{Untersuchte Elemente und deren Grenzwinkel $\theta\ua{K,lit}$\cite{k_kante}.}
  \label{tab: Halbwertsbreite}
  \begin{tabular}{S S S S S S S S}
    \toprule
    {Kante} & {$\theta\ua{max}$ in $\si{\degree}$} & {$\map{Imp}\ua{max}$}& {$\theta'$ in $\si{\degree}$} & {$\map{Imp}'$} & {$m$ in $\si{\degree}^{-1}$} & {$b$}& {$\sigma_\theta$ in $\si{\degree}^{-1}$} \\
    \midrule
    $K_{\alpha}$ & 22.00 & 4140 & 21.75 & 316 & 15296 & -332372 & 0.14 \\
    $K_{\beta}$ & 19.75 & 1194 & 19.60 & 464 & 4867 & -94922 & 0.12\\
    \bottomrule
  \end{tabular}
\end{table}
Eine qualitative Diskussion der Halbwertsbreite erfolgt in der Diskussion.
\FloatBarrier

\FloatBarrier
\subsection{Analyse von Absorptionsspektren}
Die im $2:1$ Kopplungsmodus betriebene Versuchsapperatur wurde wie folgt eingestellt:
\begin{equation*}
  \theta\in\left[\theta\ua{K}-2,\theta\ua{K}+2\right]\,\si{\degree},\quad \Delta\theta=\SI{0.1}{\degree}, \quad \Delta t=\SI{20}{\second}.
\end{equation*}
Bei dem Winkel $\theta\ua{K}$ erwartet man für die jeweilige Probe das Impulsmaximum.
Die Liste mit allen zu untersuchenden Materialen und den zugehörogen Winkel ist in Tabelle
\ref{tab: theta_k} zu finden.
\begin{table}
  \centering
  \caption{Untersuchte Elemente und deren Grenzwinkel $\theta\ua{K,lit}$\cite{k_kante}.}
  \label{tab: theta_k}
  \begin{tabular}{S S S}
    \toprule
    {Element}& {$Z$} & {$\theta\ua{K,lit}$} \\
    \midrule
    $\ce{Zr}$&40 & 9.85 \\
    $\ce{Zn}$& 30  & 18.6 \\
    $\ce{Ge}$&32 & 17.3 \\
    $\ce{Br}$&35 & 13.2 \\
    $\ce{Sr}$&38 & 11.0 \\
    \bottomrule
  \end{tabular}
\end{table}
\FloatBarrier
\FloatBarrier
\subsubsection{Untersuchung von Zink $\ce{Zn}$}\label{sec: zink}
Die bei Zink gemessenen Werte sind in Tabelle \ref{tab: zink} aufgeführt.
\begin{table}
\centering
\caption{Messwerte bei der Untersuchung des Absorptionsspektrums von $\ce{Zn}$.}
\label{tab: zink}
\begin{tabular}{S S S S }
\toprule
{$\theta \, / \, \si{\degree}$} & {$I \, / \, \mathrm{Imp}/\mathrm{s}$} & {$\theta \, / \, \si{\degree}$} & {$I \, / \, \mathrm{Imp}/\mathrm{s}$} \\
\midrule
 17.00  & 43& 19.10  & 69\\
17.10  & 46 & 19.20  & 71\\
17.20  & 45 & 19.25  & 71\\
17.25  & 47 & 19.40  & 80\\
17.40  & 45 & 19.50  & 115\\
17.50  & 44 & 19.60  & 396\\
17.60  & 43 & 19.70  & 747\\
17.70  & 43 & 19.75  & 794\\
17.75  & 40 & 19.90  & 753\\
17.90  & 44 & 20.00  & 457\\
18.00  & 41 & 20.10  & 138\\
18.10  & 53 & 20.20  & 104\\
18.20  & 59 & 20.25  & 96\\
18.25  & 68 & 20.40  & 89\\
18.40  & 74 & 20.50  & 81\\
18.50  & 70 & 20.60  & 76\\
18.60  & 72 & 20.70  & 77\\
18.70  & 74 & 20.75  & 76\\
18.75  & 77 & 20.90  & 72\\
18.90  & 68 & 21.00  & 71\\
19.00  & 70 & \,\,\text{-}  & \,\,\text{-}\\
\bottomrule
\end{tabular}
\end{table}

In der Abbildung \ref{fig: absotp_zink} befindet sich die graphische Abbildung der Messdaten.
\begin{figure}
  \centering
  \includegraphics[width=1\textwidth]{../Messdaten/zink.pdf}
  \caption{Gemessenes Absorptionsspektrum für Zink. Zusätzlich ist in dem Plot eine Vergrößerung der K-Kante zu sehen.} %Hingegen passt nicht
  \label{fig: absotp_zink}
\end{figure}
In der Abbildung sind zusätzlich auch die beiden Winkel $\theta\ua{min}$ und $\theta\ua{max}$
eingezeichnet. Mit Hilfe dieser beiden Winkel soll der Winkel, für die K-Kante genährt werden zu
\begin{equation}
  \label{eq:theta_k}
  \theta\ua{K}=\theta\ua{min}+\left(\theta\ua{max}-\theta\ua{min}\right).
\end{equation}
Die Energie $E\ua{K}$ ergibt sich dann aus der Umrechnung des Winkels mittels \eqref{eq: E_beugungswinkel}.
Für Zink ergibt sich als Energie der K-Kante
\begin{equation*}
  E\ua{K,zink}=\SI{9.9}{\kilo\eV} \quad \text{mit}\quad \theta\ua{min}=\SI{18.0}{\degree},\, \,   \theta\ua{max}=\SI{18.4}{\degree}
\end{equation*}
Mit der Gleichung \eqref{eq: sigma_1} kann aus der Energien $E\ua{K}$ allgemein die Abschirmzahl
bestimmt werden.
Für Zink ergibt sich
\begin{equation}
  \label{eq: abschirm_zink}
  \sigma\ua{K}=3.1.
\end{equation}
\FloatBarrier
\FloatBarrier
\subsubsection{Untersuchung von $\ce{Ge}$,$\ce{Zr}$,$\ce{Br}$,$\ce{Sr}$}
Die jeweils gemessenen Spektren sind in Abbildung \ref{fig: spektrum_brom_germanium} und
\ref{fig: spektrum_strontium_zirconium} dargestellt. Die zugehörogen Messwerte befinden sich in den Tabellen \ref{tab: zr}- \ref{tab: strom}.
\begin{figure}
  \centering
  \begin{subfigure}{0.48\textwidth}
    \centering
    \includegraphics[width=1 \textwidth]{../Messdaten/brom.pdf}
    \caption{Gemessenes Spektrum für Brom $\ce{Br}$. Die zugehörigen Messwerte sind in Tabelle \ref{tab: brom} zu finden.} %der, geraden
    \label{fig: brom_spektrum}
  \end{subfigure}
  \begin{subfigure}{0.48\textwidth}
    \centering
    \includegraphics[width=1 \textwidth]{../Messdaten/germanium.pdf}
    \caption{Gemessenes Spektrum für Brom $\ce{Ge}$. Die zugehörigen Messwerte sind in Tabelle \ref{tab: germanium} zu finden.} %s.o.
    \label{fig: germaium_spektrum}
  \end{subfigure}
  \caption{Gemessene Absorptionsspektrum für Brom und Germanium.}
  \label{fig: spektrum_brom_germanium}
\end{figure}
\begin{figure}
  \centering
  \begin{subfigure}{0.48\textwidth}
    \centering
    \includegraphics[width=1 \textwidth]{../Messdaten/strom.pdf}
    \caption{Gemessenes Spektrum für Strontium $\ce{Sr}$. Die zugehörigen Messwerte sind in Tabelle \ref{tab: strom} zu finden.}
    \label{fig: frank_hertz}
  \end{subfigure}
  \begin{subfigure}{0.48\textwidth}
    \centering
    \includegraphics[width=1 \textwidth]{../Messdaten/zr.pdf}
    \caption{Gemessenes Spektrum für Zirconium $\ce{Zr}$. Die zugehörigen Messwerte sind in Tabelle \ref{tab: zr} zu finden.  } %s.o.
    \label{fig: enrgie_hot}
  \end{subfigure}
  \caption{Gemessene Absorptionsspektrum für Strontium und Zirconium.}
  \label{fig: spektrum_strontium_zirconium}
\end{figure}
Analog zu Kapitel \ref{sec: zink} werden die Enegerien $E\ua{K}$ und Abschirmkonstanten $\sigma\ua{K}$
bestimmt. Die Ergebnisse werden in Tabelle \ref{tab: messergebnisse} aufgeführt.
\begin{table}
  \centering
  \caption{Messergebnisse}
  \label{tab: messergebnisse}
  \begin{tabular}{S S S S S S}
    \toprule
    {Element}&   {$\theta\ua{min}$ in $\si{\degree}$ } & {$\theta\ua{max}$ in $\si{\degree}$} & {$E\ua{K}$ in $\si{\kilo\eV}$} &{$\sigma\ua{K}$}  \\
    \midrule
    $\ce{Zr}$ & 9.2 & 9.7 & 18.8 & 2.88\\
    $\ce{Ge}$  & 15.5 & 15.9 & 11.4 & 3.09\\
    $\ce{Br}$&  12.5 & 13.0 & 13.9 & 2.98 \\
    $\ce{Sr}$ & 10.4 & 10.9 & 16.7 & 3.01 \\
    \bottomrule
  \end{tabular}
\end{table}
  \begin{table}
\centering
\caption{Messwerte bei der Untersuchung des Absorptionsspektrums von $\ce{Zr}$.}
\label{tab: zr}
\begin{tabular}{S S S S}
\toprule
{$\theta \, / \, \si{\degree}$} & {$I \, / \, \mathrm{Imp}/\mathrm{s}$} & {$\theta \, / \, \si{\degree}$} & {$I \, / \, \mathrm{Imp}/\mathrm{s}$}  \\
\midrule
8.00  & 60 & 10.10  & 141\\
8.10  & 68 & 10.20  & 145\\
8.20  & 65 & 10.30  & 141\\
8.30  & 72 & 10.40  & 146\\
8.35  & 70 & 10.50  & 146\\
8.50  & 74 & 10.60  & 146\\
8.60  & 71 & 10.70  & 144\\
8.70  & 73 & 10.80  & 142\\
8.80  & 69 & 10.90  & 144\\
8.85  & 66 & 11.00  & 150\\
9.00  & 68 & 11.10  & 145\\
9.10  & 66 & 11.20  & 144\\
9.20  & 67 & 11.30  & 139\\
9.30  & 75 & 11.40  & 140\\
9.35  & 84 & 11.50  & 140\\
9.50  & 105 & 11.60  & 138\\
9.60  & 126 &11.70  & 131\\
9.70  & 138 &11.80  & 137\\
9.80  & 140 & 11.90  & 136\\
9.85  & 142 & 12.00  & 135\\
10.00  & 141 & \,\,\text{-}  & \,\,\text{-} \\






















\bottomrule
\end{tabular}
\end{table}

\begin{table}
\centering
\caption{Messwerte bei der Untersuchung des Emmissionspektrum von $\ce{Br}$.} 
\label{tab: brom}
\begin{tabular}{S S }
\toprule
{$\theta \, / \, \si{\degree}$} & {$I \, / \, \mathrm{Imp}/\mathrm{s}$}  \\
\midrule
 11.0  & 15.0\\
11.1  & 14.0\\
11.2  & 17.0\\
11.3  & 16.0\\
11.4  & 17.0\\
11.5  & 17.0\\
11.6  & 17.0\\
11.7  & 16.0\\
11.8  & 16.0\\
11.9  & 16.0\\
12.0  & 13.0\\
12.1  & 15.0\\
12.2  & 13.0\\
12.3  & 14.0\\
12.4  & 15.0\\
12.5  & 14.0\\
12.6  & 16.0\\
12.7  & 19.0\\
12.8  & 22.0\\
12.9  & 27.0\\
13.0  & 30.0\\
13.1  & 29.0\\
13.2  & 31.0\\
13.3  & 28.0\\
13.4  & 29.0\\
13.5  & 27.0\\
13.6  & 27.0\\
13.7  & 26.0\\
13.8  & 23.0\\
13.9  & 23.0\\
14.0  & 22.0\\
14.1  & 22.0\\
14.2  & 21.0\\
14.3  & 23.0\\
14.4  & 20.0\\
14.5  & 18.0\\
14.6  & 19.0\\
14.7  & 17.0\\
14.8  & 17.0\\
14.9  & 17.0\\
15.0  & 17.0\\
\bottomrule
\end{tabular}
\end{table}

\begin{table}
\centering
\caption{Messwerte bei der Untersuchung des Absorptionsspektrums von $\ce{Ge}$.}
\label{tab: germanium}
\begin{tabular}{S S S S }
\toprule
{$\theta \, / \, \si{\degree}$} & {$I \, / \, \mathrm{Imp}/\mathrm{s}$} & {$\theta \, / \, \si{\degree}$} & {$I \, / \, \mathrm{Imp}/\mathrm{s}$}  \\
\midrule
15.10  & 22 & 17.10  & 32\\
15.20  & 19 & 17.20  & 32\\
15.30  & 20 & 17.25  & 31\\
15.40  & 22 & 17.40  & 31\\
15.50  & 22 & 17.50  & 26\\
15.60  & 24 & 17.60  & 28\\
15.70  & 29 & 17.70  & 26\\
15.80  & 37 & 17.75  & 26\\
15.90  & 40 & 17.90  & 25\\
16.00  & 39 & 18.00  & 26\\
16.10  & 42 & 18.10  & 23\\
16.20  & 42 & 18.20  & 28\\
16.25  & 40 & 18.25  & 24\\
16.40  & 37 & 18.40  & 22\\
16.50  & 36 & 18.50  & 25\\
16.60  & 35 & 18.60  & 22\\
16.75  & 35 & 18.70  & 22\\
16.80  & 35 & 18.75 & 21\\
16.90  & 33 & 18.90  & 21\\
15.00  & 21 & 19.00  & 21\\
17.00  & 33 & \,\,\text{-}  & \,\,\text{-} \\
\bottomrule
\end{tabular}
\end{table}

\begin{table}
\centering
\caption{Messwerte bei der Untersuchung des Emmissionspektrum von $\ce{Sr}$.}
\label{tab: strom}
\begin{tabular}{S S S S}
\toprule
{$\theta \, / \, \si{\degree}$} & {$I \, / \, \mathrm{Imp}/\mathrm{s}$} & {$\theta \, / \, \si{\degree}$} & {$I \, / \, \mathrm{Imp}/\mathrm{s}$}  \\
\midrule
9.00  & 35 & 11.10  & 104\\
9.10  & 36 & 11.20  & 101\\
9.20  & 35 & 11.30  & 97\\
9.30  & 33 & 11.40  & 98\\
9.35  & 37 & 11.50  & 94\\
9.50  & 34 & 11.60  & 94\\
9.60  & 31 & 11.70  & 90\\
9.70  & 31 & 11.80  & 93\\
9.80  & 33 & 11.90  & 90\\
9.85  & 34 & 12.00  & 88\\
10.00  & 32 & 12.10  & 88\\
10.10  & 33 &12.20  & 84\\
10.20  & 33 & 12.30  & 80\\
10.30  & 32 & 12.40  & 82\\
10.40  & 33 & 12.50  & 79\\
10.50  & 43 & 12.60  & 75\\
10.60  & 63 & 12.70  & 71\\
10.70  & 82 & 12.80  & 69\\
10.80  & 96 & 12.90  & 63\\
10.90  & 109 &13.00  & 61\\
11.00  & 103 & \,\,\text{-}  & \,\,\text{-} \\
\bottomrule
\end{tabular}
\end{table}

\FloatBarrier
\FloatBarrier
\subsubsection{Untersuchung des Absorptionsspektrum von Gold $\ce{Au}$}
Bei der Untersuchung der beiden L-Kanten ($L_2$ und $L_3$) wird die Apparatur, wie folgt
konfiguriert:
\begin{equation*}
  \theta\in\left[\theta\ua{{L_3}}-2,\theta\ua{{L_2}}+2\right]\,\si{\degree},\quad \Delta\theta=\SI{0.1}{\degree}, \quad \Delta t=\SI{20}{\second}.
\end{equation*}
Dabei wurden die Winkel
\begin{equation*}
  \ua{\theta\ua{L_2}}=\SI{12.95}{\degree} \qquad  \ua{\theta\ua{L_3}}=\SI{14.95}{\degree}
\end{equation*}
aus der Literatur \cite{l_kante} notiert.
Die aufgenommenen Messwerte sind in Tabelle \ref{tab: gold} aufgelistet und in Abbildung
\ref{fig: absotp_gold} aufgetragen. In dieser ist die abgelesene $L_2$ und $L_3$ Kante mit eingezeichnet.

\begin{figure}
  \centering
  \includegraphics[width=0.8\textwidth]{../Messdaten/gold.pdf}
  \caption{Gemessenes Absorptionsspektrum für Gold. In der Graphik sind die abgelesenen $L_2$ und $L_2$ Kanten farblich markiert.} %Hingegen passt nicht
  \label{fig: absotp_gold}
\end{figure}
Die abgelesenen Winkel werden mit Gleichung \eqref{eq: E_beugungswinkel} in Energien umgerechnet:
\begin{equation}
  \label{eq:l_kanten_gold}
  E\ua{L_2}=\SI{14.0}{\kilo\eV}  \qquad   E\ua{L_3}=\SI{12.0}{\kilo\eV}.
\end{equation}
Mit den Energiewerten und der Kernladungszahl $Z\ua{Au}=79$ kann die Abschirmkonstante
von Gold bestimmt werden. Unter Verwendung von Gleichung \eqref{eq: sigma_L} ergibt sich
\begin{equation*}
    \sigma\ua{L}=1.7.
\end{equation*}
\input{../Messdaten/gold_final.tex}
\FloatBarrier
\FloatBarrier
\subsection{Bestimmung der Rydbergenergie}
Für die Bestimmung Rydbergenerigie $R_\infty$ wird durch die zuvor
bestimmten Energien $E\ua{K}$ eine Regressionsgerade der Form
\begin{equation*}
  g(x)=mx+b
\end{equation*}
gelegt. Mit der Ausgleichsrechnung sollen die Parameter der Gleichung \eqref{eq: E_niveaus} ($n=1$)
bestimmt werden. Beim Koeffizientenvergleich ergibt sich das die Rydbergenergie
gerade
\begin{equation*}
  R_\infty=m^2
\end{equation*}
ist.
Die in den vorherigen Kapiteln bestimmten Energien sind in Tabelle \ref{tab: ener_ryd} gelistet
und in der Abbildung \ref{fig: ryd_ener} gezeichnet. In der Abbildung ist
die resultierende Regressionsgerade mit aufgetragen.
\begin{table} 
\centering 
\caption{Experimentell bestimmten Energie $E_{\mathrm{K}}$} 
\label{tab: ener_ryd} 
\begin{tabular}{S S S} 
\toprule  
{Element} & {Z} & {$E_{\mathrm{K}}$}  \\ 
\midrule  
$\ce{Zr}$& 40.0  & 136.9\\ 
$\ce{Ge}$ & 32.0  & 106.7\\ 
$\ce{Br}$ &35.0  & 118.1\\ 
$\ce{Sr}$ &38.0  & 129.1\\ 
 $\ce{Zn}$ &30.0  & 99.3\\ 
\bottomrule 
\end{tabular} 
\end{table}


\begin{figure}
  \centering
  \includegraphics[width=0.7\textwidth]{../Messdaten/energie_z.pdf}
  \caption{Auftagung der experimentell berechneten Energien $E\ua{K}$ gegen die Kernladungszahl $Z$. Zusätzlich ist in der Abbildung die zugehöroge Ausgleichsgerade zu sehen.}
  \label{fig: ryd_ener}
\end{figure}
Die Ausgleichsrechnung liefert folgende Parameter:
\begin{equation*}
m=\num{3.76\pm0.02}\,\sqrt{\si{\eV}} \qquad b=\num{-13.5\pm0.7}\,\sqrt{\si{\eV}}.
\end{equation*}
Damit wird die Rydbergenergie bestimmt zu:
\begin{equation}
  R_\infty=\SI{14.11\pm0.15}{\eV}
\end{equation}
\FloatBarrier
