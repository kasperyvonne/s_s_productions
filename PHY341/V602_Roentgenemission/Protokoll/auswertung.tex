\section{Auswertung}
Die in der Auswertung bestimmten Ausgleichsrechnungen werden mit
dem Python Paket \emph{scipy.optimize}\cite{scipy} durchgeführt.
Des Weiteren werden die Fehler und insbesondere die Fehlerfortpflanzungen
mit dem Python Paket \emph{uncertainties}\cite{uncertainties} berechnet.
Zusätzlich werden die benötigten Naturkonstanten dem Python Paket \emph{scipy.constants}\cite{scipy.constants}
entommen.

\subsection{Untersuchung der Bragg-Bedingung}
Die an der Messapperatur eingestellten Konfiguration lauten:
\begin{equation*}
  \theta=\SI{14}{\degree}, \quad \alpha\ua{GM}\in\left[26,30\right]\,\si{\degree},\quad \Delta\alpha\ua{GM}=\SI{0.1}{\degree}, \quad \Delta t=\SI{10}{\second}.
\end{equation*}
Eine Liste mit den gemessenen Werten ist in Tabelle \ref{} zu finden und in Abbildung \ref{fig: bragg_plot} abgebildet.
\begin{table} 
\centering 
\caption{Messwerte bei der Untersuchung der Bragg Bedingung.} 
\label{tab: bragg_test} 
\begin{tabular}{S S S S } 
\toprule  
{$\alpha_{\mathrm{Gl}} \, / \, \si{\degree}$} & {$I \, / \, \mathrm{Imp}/\mathrm{s}$} & {$\alpha_{\mathrm{Gl}} \, / \, \si{\degree}$} & {$I \, / \, \mathrm{Imp}/\mathrm{s}$}  \\ 
\midrule  
 26.0  & 41  & 28.1  & 156\\ 
26.1  & 41  & 28.2  & 158\\ 
26.2  & 44  & 28.3  & 165\\ 
26.3  & 43  & 28.4  & 168\\ 
26.4  & 47  & 28.5  & 169\\ 
26.5  & 47  & 28.6  & 156\\ 
26.6  & 53  & 28.7  & 148\\ 
26.7  & 56  & 28.8  & 137\\ 
26.8  & 68  & 28.9  & 125\\ 
26.9  & 75  & 29.0  & 119\\ 
27.0  & 73  & 29.1  & 110\\ 
27.1  & 83  & 29.2  & 105\\ 
27.2  & 91  & 29.3  & 91\\ 
27.3  & 102  & 29.4  & 86\\ 
27.4  & 110  & 29.5  & 69\\ 
27.5  & 122  & 29.6  & 66\\ 
27.6  & 118  & 29.7  & 53\\ 
27.7  & 133  & 29.8  & 43\\ 
27.8  & 140  & 29.9  & 40\\ 
27.9  & 150  & 30.0  & 40\\ 
\bottomrule 
\end{tabular} 
\end{table}
\begin{figure}
  \centering
  \includegraphics[width=0.8 \textwidth]{../Messdaten/bragbed.pdf}
  \caption{Messwerte zur Untersuchung der Bragg-Bedingung.} %Hingegen passt nicht
  \label{fig: bragg_plot}
\end{figure}
Das Maximum der Messwerte ist der bestimmte Bragg-Winkel:
\begin{equation}
  \label{eq:bestimmt_bragg_winkel}
  2\theta\ua{bragg}=\SI{28.5}{\degree} \quad \Leftrightarrow \quad \theta\ua{bragg}=\SI{14.25}{\degree}.
\end{equation}

\subsection{Untersuchung des Emissionspektrum von $\ce{Cu}$}
Zu Beginn wird der $2:1$ Kopplungsmodus der Apperatur gewählt und folgende Parameter eingestellt:
\begin{equation*}
  \theta\in\left[4,26\right]\,\si{\degree},\quad \Delta\theta=\SI{0.2}{\degree}, \quad \Delta t=\SI{5}{\second}.
\end{equation*}
Die aufgenommen Werte sind in Tabelle \ref{} aufgelistet und in Abbildung \ref{fig: emission_cu} dargestellt.
\begin{table} 
\centering 
\caption{Messwerte bei der Untersuchung des Emmissionspektrums von $\ce{Cu}$.} 
\label{tab: emi_cu} 
\begin{tabular}{S S S S S S } 
\toprule  
{$\theta \, / \, \si{\degree}$} & {$I \, / \, \mathrm{Imp}/\mathrm{s}$} & {$\theta \, / \, \si{\degree}$} & {$I \, / \, \mathrm{Imp}/\mathrm{s}$} &{$\theta \, / \, \si{\degree}$} & {$I \, / \, \mathrm{Imp}/\mathrm{s}$}  \\ 
\midrule  
 4.00  & 30  & 11.40  & 215  & 18.75  & 119\\ 
4.20  & 27  & 11.60  & 218  & 19.00  & 101\\ 
4.40  & 29  & 11.80  & 201  & 19.20  & 110\\ 
4.60  & 31  & 12.00  & 225  & 19.40  & 121\\ 
4.80  & 34  & 12.20  & 220  & 19.60  & 464\\ 
5.00  & 43  & 12.40  & 212  & 19.75  & 1194\\ 
5.20  & 51  & 12.60  & 220  & 20.00  & 761\\ 
5.40  & 61  & 12.80  & 195  & 20.20  & 169\\ 
5.60  & 64  & 13.00  & 175  & 20.40  & 147\\ 
5.80  & 67  & 13.20  & 166  & 20.60  & 123\\ 
6.00  & 83  & 13.40  & 176  & 20.75  & 120\\ 
6.20  & 87  & 13.60  & 163  & 21.00  & 116\\ 
6.40  & 94  & 13.80  & 162  & 21.20  & 113\\ 
6.60  & 104  & 14.00  & 163  & 21.40  & 122\\ 
6.80  & 118  & 14.20  & 161  & 21.60  & 165\\ 
7.00  & 128  & 14.40  & 154  & 21.75  & 316\\ 
7.20  & 127  & 14.60  & 153  & 22.00  & 4140\\ 
7.40  & 145  & 14.80  & 151  & 22.20  & 3604\\ 
7.60  & 145  & 15.00  & 150  & 22.40  & 333\\ 
7.80  & 159  & 15.20  & 145  & 22.60  & 144\\ 
8.00  & 156  & 15.40  & 146  & 22.75  & 110\\ 
8.20  & 169  & 15.60  & 136  & 23.00  & 92\\ 
8.35  & 168  & 15.80  & 137  & 23.20  & 86\\ 
8.60  & 178  & 16.00  & 140  & 23.40  & 83\\ 
8.80  & 181  & 16.20  & 130  & 23.60  & 67\\ 
9.00  & 188  & 16.40  & 130  & 23.80  & 75\\ 
9.20  & 185  & 16.60  & 117  & 24.00  & 69\\ 
9.35  & 204  & 16.75  & 122  & 24.20  & 67\\ 
9.60  & 204  & 17.00  & 126  & 24.40  & 64\\ 
9.80  & 198  & 17.20  & 114  & 24.60  & 61\\ 
10.00  & 215  & 17.40  & 118  & 24.80  & 60\\ 
10.20  & 211  & 17.60  & 105  & 25.00  & 58\\ 
10.40  & 225  & 17.75  & 119  & 25.20  & 0\\ 
10.60  & 231  & 18.00  & 98  & 25.40  & 54\\ 
10.80  & 211  & 18.20  & 110  & 25.60  & 55\\ 
11.00  & 235  & 18.40  & 104  & 25.80  & 48\\ 
11.20  & 221  & 18.60  & 112  & 26.00  & 45\\ 
\bottomrule 
\end{tabular} 
\end{table}
\begin{figure}
  \centering
  \includegraphics[width=0.8 \textwidth]{../Messdaten/emission_cu.pdf}
  \caption{Messsung des Emmissionspektrums von $\ce{Cu}$.} %Hingegen passt nicht
  \label{fig: emission_cu}
\end{figure}
In der Abbildung sind die $K_\alpha$ und $K_\beta$ Kante mit eingezeichnet.
Aus dem Experiment ergebens sich die folgenden Werte:
\begin{equation}
  \label{eq:k_alpha,k_beta}
  K_\alpha=\SI{8.2}{\kilo\eV} \qquad   K_\beta=\SI{9.1}{\kilo\eV}.
\end{equation}
Die maximale Energie beläuft sich auf:
\begin{equation}
  \label{eq: maximale_energie}
  E\ua{max}=\SI{44.1}{\kilo\eV}
\end{equation}
Die Winkel werden dabei mit Gleichung \eqref{} umgerechnet.
Mit Hilfe der Energien \eqref{eq:k_alpha,k_beta}, können die Abschirmkonstanten
für die einzelnen K- Linien bestimmt werden.
Aus den Formeln \eqref{} errechnet sich:
\begin{equation}
   \label{eq:abschirm}
   \sigma_1=3.13 \qquad \sigma_2=20.9.
\end{equation}

\textbf{Es fehlt noch die Halbwertsbreite}

\subsection{Analyse von Absorptionsspektren}

Die im $2:1$ Kopplungsmodus betriebene Versuchsapperatur wurde wie folgt eingestellt:
\begin{equation*}
  \theta\in\left[\theta\ua{K}-2,\theta\ua{K}+2\right]\,\si{\degree},\quad \Delta\theta=\SI{0.1}{\degree}, \quad \Delta t=\SI{20}{\second}.
\end{equation*}
Der Winkel $\theta\ua{K}$ ist der Winkel an für die Probe das Maximum erwartet
wird. Die Liste mit allen zu untersuchenden Materialen und Winkel ist in Tabelle
\ref{tab: theta_k} zu finden.
\begin{table}
  \centering
  \caption{Untersuchte Elemente und deren Grenzwinkel $\theta\ua{K}$}
  \label{tab: theta_k}
  \begin{tabular}{S S S}
    \toprule
    {Element}& {$Z$} & {$\theta\ua{K,lit}$} \\
    \midrule
    $\ce{Zn}$& 30  & 9.65 \\
    $\ce{Zr}$&40 & 9.85 \\
    $\ce{Ge}$&32 & 17.3 \\
    $\ce{Br}$&35 & 13.2 \\
    $\ce{Sr}$&38 & 11.0 \\
    \bottomrule
  \end{tabular}
\end{table}
\subsubsection{Untersuchung von Zink $\ce{Zn}$}\label{sec: zink}
Die bei Zink gemessenen Werte sind in Tabelle \ref{} aufgeführt.
\begin{table} 
\centering 
\caption{Messwerte bei der Untersuchung des Emmissionspektrum von $\ce{Cu}$.} 
\label{tab: zink} 
\begin{tabular}{S S } 
\toprule  
{$\theta \, / \, \si{\degree}$} & {$I \, / \, \mathmrm{Imp}/\mathrm{s}$}  \\ 
\midrule  
 17.0  & 43.0\\ 
17.1  & 46.0\\ 
17.2  & 45.0\\ 
17.2  & 47.0\\ 
17.4  & 45.0\\ 
17.5  & 44.0\\ 
17.6  & 43.0\\ 
17.7  & 43.0\\ 
17.8  & 40.0\\ 
17.9  & 44.0\\ 
18.0  & 41.0\\ 
18.1  & 53.0\\ 
18.2  & 59.0\\ 
18.2  & 68.0\\ 
18.4  & 74.0\\ 
18.5  & 70.0\\ 
18.6  & 72.0\\ 
18.7  & 74.0\\ 
18.8  & 77.0\\ 
18.9  & 68.0\\ 
19.0  & 70.0\\ 
19.1  & 69.0\\ 
19.2  & 71.0\\ 
19.2  & 71.0\\ 
19.4  & 80.0\\ 
19.5  & 115.0\\ 
19.6  & 396.0\\ 
19.7  & 747.0\\ 
19.8  & 794.0\\ 
19.9  & 753.0\\ 
20.0  & 457.0\\ 
20.1  & 138.0\\ 
20.2  & 104.0\\ 
20.2  & 96.0\\ 
20.4  & 89.0\\ 
20.5  & 81.0\\ 
20.6  & 76.0\\ 
20.7  & 77.0\\ 
20.8  & 76.0\\ 
20.9  & 72.0\\ 
21.0  & 71.0\\ 
\bottomrule 
\end{tabular} 
\end{table}
In der Abbildung \ref{fig: absotp_zink} befindet sich die graphische Abbildung der Messdaten.
\begin{figure}
  \centering
  \includegraphics[width=1.2\textwidth]{../Messdaten/zink.pdf}
  \caption{Gemessenes Absorptionsspektrum für Zink. Zusätzlich ist in dem Plot eine Vergrößerung der K-Kante zu sehen.} %Hingegen passt nicht
  \label{fig: absotp_zink}
\end{figure}
In der Abbildung ist zusätzlich auch die beiden Winkel $\theta\ua{min}$ und $\theta\ua{max}$ die die K-Kante einschränken
eingezeichnet. Um die Energie der K-Kante zu bestimmen wird der Mittelwert der beiden Winekl bestimmt
\begin{equation}
  \label{eq:theta_k}
  \theta\ua{K}=\theta\ua{min}+\left(\theta\ua{max}-\theta\ua{min}\right)
\end{equation}
und anschließend mit Gleichung \eqref{} in die entsprechende Energie umgerechnet.
Für Zink ergibt sich als Energie der K-Kante
\begin{equation*}
  E\ua{K,zink}=\SI{9.9}{\kilo\eV} \quad \text{mit}\quad \theta\ua{min}=\SI{18.0}{\degree},\,\theta\ua{max}=\SI{18.4}{\degree}
\end{equation*}
Mit der Gleichung \eqref{} kann aus der Energien $E\ua{K}$ allgemein die Abschirmzahl
bestimmt werden.
Für Zink ergibt sich
\begin{equation}
  \label{eq: abschirm_zink}
  \sigma\ua{K}=3.1.
\end{equation}
\subsubsection{Untersuchung von $\ce{Ge}$,$\ce{Zr}$,$\ce{Br}$,$\ce{Sr}$}
Die für die einzelenen Elemente gemessenen Spektren sind in Abbildung \ref{fig: spektrum_brom_germanium} und
\ref{fig: spektrum_strontium_zirconium} dargestellt. Die zugehörogen Messwerte befinden sich in den Tabellen \ref{}- \ref{}.
\begin{figure}
  \centering
  \begin{subfigure}{0.48\textwidth}
    \centering
    \includegraphics[width=1 \textwidth]{../Messdaten/brom.pdf}
    \caption{Gemessenes Spektrum für Brom $\ce{Br}$. Die zugehörigen Messwerte sind in Tabelle \ref{} zu finden.} %der, geraden
    \label{fig: brom_spektrum}
  \end{subfigure}
  \begin{subfigure}{0.48\textwidth}
    \centering
    \includegraphics[width=1 \textwidth]{../Messdaten/germanium.pdf}
    \caption{Gemessenes Spektrum für Brom $\ce{Ge}$. Die zugehörigen Messwerte sind in Tabelle \ref{} zu finden.} %s.o.
    \label{fig: germaium_spektrum}
  \end{subfigure}
  \caption{Gemessene Absorptionsspektrum für Brom und Germanium.}
  \label{fig: spektrum_brom_germanium}
\end{figure}
\begin{figure}
  \centering
  \begin{subfigure}{0.48\textwidth}
    \centering
    \includegraphics[width=1 \textwidth]{../Messdaten/strom.pdf}
    \caption{Gemessenes Spektrum für Strontium $\ce{Sr}$. Die zugehörigen Messwerte sind in Tabelle \ref{} zu finden.}
    \label{fig: frank_hertz}
  \end{subfigure}
  \begin{subfigure}{0.48\textwidth}
    \centering
    \includegraphics[width=1 \textwidth]{../Messdaten/zr.pdf}
    \caption{Gemessenes Spektrum für Zirconium $\ce{Zr}$. Die zugehörigen Messwerte sind in Tabelle \ref{} zu finden.  } %s.o.
    \label{fig: enrgie_hot}
  \end{subfigure}
  \caption{Gemessene Absorptionsspektrum für Strontium und Zirconium.}
  \label{fig: spektrum_strontium_zirconium}
\end{figure}
Analog zu Kapitel \ref{sec: zink} werden die Enegerie $E\ua{K}$ und $\sigma\ua{K}$
bestimmt. Die sich Ergebnisse werden in Tabelle \ref{tab: messergebnisse} aufgeführt.
\begin{table}
  \centering
  \caption{Messergebnisse}
  \label{tab: messergebnisse}
  \begin{tabular}{S S S S S S}
    \toprule
    {Element}&   {$\theta\ua{min}$ in $\si{\degree}$ } & {$\theta\ua{max}$ in $\si{\degree}$} & {$E\ua{K}$ in $\si{\kilo\eV}$} &{$\sigma\ua{K}$}  \\
    \midrule
    $\ce{Zr}$ & 9.2 & 9.7 & 18.8 & 2.88\\
    $\ce{Ge}$  & 15.5 & 15.9 & 11.4 & 3.09\\
    $\ce{Br}$&  12.5 & 13.0 & 13.9 & 2.98 \\
    $\ce{Sr}$ & 10.4 & 10.9 & 16.7 & 3.01 \\
    \bottomrule
  \end{tabular}
\end{table}
