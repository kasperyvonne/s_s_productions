\def\ifundefined#1{\expandafter\ifx\csname#1\endcsname\relax} \ifundefined{preambleloaded}
\typeout{PRECOMILED PREAMBLE NOT LOADED}\documentclass[parskip=half, bibliography=totoc, captions=tableheading, titlepage=firstiscover]{scrartcl} %Ich habe [parskip=half] hinzugefügt

%\usepackage[calc]{picture}
\usepackage{fixltx2e}
\usepackage{longtable}
\usepackage{tcolorbox}

%\pagestyle{headings}
\usepackage{scrpage2}
\clearscrheadings

\addtolength{\footskip}{1cm}
\ohead{}
\cfoot[\pagemark]{\pagemark}
\ofoot[]{}
\ifoot[]{}
\pagestyle{scrheadings}

\usepackage{polyglossia}
\setmainlanguage{german}
\usepackage{caption}
\usepackage{amsmath}
\usepackage{amssymb}
\usepackage{mathtools}

\usepackage{fontspec}
\defaultfontfeatures{Ligatures=TeX}

\usepackage[
  math-style=ISO,
  bold-style=ISO,
  sans-style=italic,
  nabla=upright,
]{unicode-math}

\setmathfont{Latin Modern Math}
%\setmathfont[range={\mathscr, \mathbfscr}]{XITS Math}
%\setmathfont[range=\coloneq]{XITS Math}
%\setmathfont[range=\propto]{XITS Math}

\usepackage[autostyle]{csquotes}

\usepackage[
  locale=DE,                   % deutsche Einstellungen
  separate-uncertainty=true,   % Immer Fehler mit \pm
  per-mode=symbol-or-fraction, % m/s im Text, sonst Brüche
]{siunitx}
%\sisetup{math-stylemicro=\text{µ},text-micro=µ}

\usepackage{xfrac}

\usepackage[section, below]{placeins}
\usepackage[
  labelfont=bf,
  font=small,
  width=0.9\textwidth,
]{caption}

\usepackage{subcaption}

\usepackage{graphicx}

\usepackage{float}
\floatplacement{figure}{h}
\floatplacement{table}{h}

\usepackage{booktabs}



\usepackage{bookmark}

\usepackage[shortcuts]{extdash}

\usepackage[math]{blindtext}

\usepackage{microtype}

\usepackage[
  backend=biber,
]{biblatex}
% Quellendatenbank
\addbibresource{lit.bib}

\usepackage{hyperref}

\usepackage{color} % Das ist Geschmacksfrage

\usepackage{makeidx} %Ich habe makeidx hinzugefügt + makeindex
\makeindex


\usepackage[version=3]{mhchem} % für Thermodynamik-chemische Elemente
\usepackage{enumitem} %Ich habe enumitem hinzugefügt
\usepackage{expl3}
\usepackage{xparse}
%\ExplSyntaxOn

\NewDocumentCommand \dif {m}
{
\mathinner{\symup{d} #1}
}
\usepackage{subcaption}



%\usepackage{showframe}
\newcommand{\versuch}{Hall-Effekt und Elektrizitätsleitung bei Metallen}
\newcommand{\vnr}{311}
\newcommand{\vd}{Tag der Durchführung: 20.12.16}
\newcommand{\va}{Tag der Abgabe: dd.mm.yy}
\newcommand{\map}[1]{\mathup{#1}}

\author{Steven Becker \\
steven.becker@tu-dortmund.de \\
und \\
Stefan Grisard \\
stefan.grisard@tu-dortmund.de}

\title{\versuch}
\subtitle{Versuch \vnr}

\date{\vd \\
\va}
 \else
\typeout{\preambleloaded}
\fi

\begin{document}

\maketitle
\newpage

\section{Theorie}

\subsection{Trägheitsmoment}
Für das Trägheitsmoment eines starren Körpers gilt:
\begin{equation}
   I = \int r^2 \dif{m}
\end{equation}

Hierbei entspricht $r$ dem Abstand eines Massenelements $dm$ zur Rotationsachse. Dies
lässt sich mit Hilfe der Dichte des Körpers $\rho(\vec{r}) = \dif{m}/\dif{V}$ umformulieren zu:
\begin{equation}
  I = \int \rho(\vec{r})\cdot r^2 \dif{V}
\end{equation}

Das Trägheitsmoment ist stets bezüglich einer Achse definiert und ist eine
additive Größe. Der so genannte \textit{Satz von Steiner} macht eine Aussage über die Änderung des Trägheitsmoments
bei paralleler Verschiebung jener Achsen, die durch den Schwerpunkt verlaufen. Ist das
Trägheitsmoment $I_S$ eines Körpers der Masse $m$ bezüglich einer solchen Achse bekannt, so ergibt sich das
Trägheitsmoment bezüglich einer um die Länge $a$ parallel verschobenen Achse zu:

\begin{equation}
    I = I_S + m \cdot a^2
    \label{eq: steiner}
\end{equation}
Konkret werden im Versuch die folgenden Formeln zur Berechnung von Trägheitsmomenten benötigt:
\begin{enumerate}
  \item homogene Kugel mit Radius R, Gesamtmasse M, Rotation um Symmetrieachse:
  \begin{equation}
    I = \frac{2}{5}M R^2
  \end{equation}

  \item homogener Zylinder mit Radius R, Höhe h, Gesamtmasse M, Rotation um Symmetrieachse:
  \begin{equation}
    I = \frac{1}{2} M R^2
  \end{equation}

  \item homogener Zylinder mit Radius R, Höhe h, Gesamtmasse M, Rotation um Achse durch den Schwerpunkt senkrecht zur Symmetrieachse:
  \begin{equation}
    I = M (\frac{R^2}{4} + \frac{h^2}{12})
  \end{equation}
\end{enumerate}

\subsection{Drehbewegung}
Für den Drehimpuls eines rotierenden Körpers gilt:
\begin{equation}
 \vec{L} = I \cdot \vec{\omega}
\end{equation}

Bei konstantem Trägheitsmoment folgt mit dem Drehwinkel $\phi$ für die zeitliche Änderung, die dem Drehmoment
entspricht:
\begin{equation}
  \dot{\vec{L}} = \vec{M} = I \cdot \frac{d}{dt}\vec{\omega} =
  I \cdot \frac{d^2}{dt^2} \phi \cdot \frac{\vec{\omega}}{|\vec{\omega}|}
  \label{eq: drehmoment}
\end{equation}
In einem schwingunsfähigen Drehsystem wirkt z.B. durch eine Torsionsfeder ein
rücktreibendes Moment $M_{r}$, dessen Stärke durch eine Winkelrichtgröße $D$ gewichtet
werden kann.
\begin{equation}
  M_{r} = - D \cdot \phi
\end{equation}
Mit \eqref{eq: drehmoment} ergibt sich die homogene Differentialgleichung:
\begin{equation}
  \ddot{\phi} + \frac{D}{I}\phi = 0
\end{equation}
Hiebei handelt sich um die charakteristische Differentialgleichung des harmonischen
Oszillators, deren Allgemeine Lösung mit $\omega = \sqrt{\frac{D}{I}}$ folgendermaßen angegeben werden kann:
\begin{equation}
  \phi(t) = A\cdot \cos{\omega t} + B \cdot \sin{\omega t}
\end{equation}
Zwischen Drehfrequenz und Schwingungsdauer $T$ besteht der bekannte Zusammenhang
$T =\omega / 2\pi$. \\

\section{Versuchsdurchführung}
Mit den gefundenen Zusammenhängen aus dem voran gegangenen Abschnitt ist es
möglich das Trägheitsmoment verschiedener Körper experimentell zu bestimmen.
Des Weiteren soll der \textit{Satz von Steiner} \refeq{eq: steiner} überprüft
werden.

\section{Auswertung}
\subsection{Eichung des Thermoelements}
Der Zusammenhang zwischen dem Kontaktpotential des Kalorimeters und der Temperatur soll im späteren Verlauf
durch einen linearen Zusammenhang angenähert werden. Hierzu wurde eine Kontakstelle in Eiswasser($0 \si{\celsius}$)
und eine in kochendes Wasser ($100 \si{\celsius}$) gehalten. Folgender Wert wurde hierbei am Kalorimeter abgelesen:
\begin{equation}
  U_{eich} = 3.98 \si{\milli\volt}
\end{equation}
Die Steigung der Geraden ergibt sich zu:
\begin{equation}
  m = \frac{100 \si{\kelvin}}{U_{eich}} = 25.13 \si{\kelvin \per \milli\volt}
\end{equation}
Hiermit lässt sich die Temperatur in Kelvin über den Zusammenhang
\begin{equation}
  T(U) = m \cdot U + 275.13 \si{\kelvin}
\end{equation}
berechnen. Im späteren Verlauf werden die Spannungen direkt durch die zugeordnete Temperatur ersetzt.

\subsection{Bestimmung der Wärmekapazität des Kalorimeters}
Zur Bestimmung der spezifischen Wärmekapazitäten wird zunächst die Wärmekapazität des Kalorimeters über den Zusammenhang
(Formel x) berechnet. Die Massen $m_x$ und $m_y$ ergeben sich mittels der Dichte von Wasser bei $40\si{\celsius}$: $\rho_w = ...\si{\gram\meter^3}$ (quelle).
Der Wert $c_w = ...$ wurde der Praktikumsanleitung entnommen. Die gemessenen Werte, sowie das Ergebnis für $c_g m_g$ sind in $tab$ aufgetragen.
\begin{table}
  \centering
  \begin{tabular}{S S S S S}
      \toprule
    {$m_x$ in $\si{\gram}$} & {$m_y$ in $\si{\gram}$} & {$T_x$ in $\si{\kelvin}$} & {$T_y$ in $\si{\kelvin}$} & $c_g m_g$ in  \\
    \midrule
  \end{tabular}
  \caption{Massen, sowie Temperaturen der beiden Wassermengen zur Bestimmung der spezifischen Wärmekapazität des Kalorimeters}
  \label{tab: cgmg}
\end{table}

\subsection{Bestimmung der Wärmekapazitäten von Blei, Graphit und Aluminium}
Die berechneten Temperaturen aus den gemessenen Kontaktpotentialen zur Bestimmung der Wärmekapazitäten befinden sich in den Tabellen (...).
\begin{table}

\section{Diskussion}
Im Folgenden sollen die Messergebnis im Bezug auf die Messgenauigkeit des
Vesuchsaufbaus diskutiert werden.
Zusammenfassent findet man in Tabelle \ref{tab:resultate} die Resultate.

\begin{table}
\centering
\caption{Messergebnisse.}
\label{tab:resultate}
\begin{tabular}{S S S}
\toprule
{Stoff} & {Bestimmte Halbwertszeit $T$ in $\si{\second}$} & {Theoretischehalbwertszeit $T\ua{theo}$ in $\si{\second}$} \\
\text{Indium} \, $\ce{^{116}_{49}In}$  & 3.53\pm0.12 e+3  & 3.24 e+3 \\
\text{Rhodium}\, $\ce{^{104i}_{45} Rh}$  & 3.4\pm2.6 e+2  & 2.6 e+2 \\
\text{Rhodium} $\ce{^{104}_{45} Rh}$  & 54.3\pm2.0 & 42.3 \\
\bottomrule
\end{tabular}
\end{table}

Die Unterschiede zu der Theorie, insbesondere bei $\ce{^{104i}_{45} Rh}$ sind
erstmal nicht mit dem Versuchaufbau zu erklären. Die in der Abbildung \ref{fig: plot_rhodium_lang}
zu erkenneden großen Abweichung, sind wohmöglich auf den verkürtzen Messbereich
zurück zuführen. Eine Reduzierung der Ungenauigkeit lässt sich durch erneute Messungen
verringer.

Abschließend ist zu sagen, dass der Versuch bei den Stoffen $\ce{^{116}_{49}In}$ und $\ce{^{104}_{45} Rh}$
gute Ergebnise liefert. Lediglich bei $\ce{^{104i}_{45} Rh}$ sollte eine neue
Messung durchgeführt werden.


\printbibliography
\newpage
\section{Anhang}
\begin{figure}
  \centering
  \includegraphics[width = \textwidth]{pics/schokolade.pdf}
  \caption{Fit an Daten des Röngenemissionsspektrums. Hierbei wird das kontinuierliche Bremsspektrum durch ein Polynom
  dritten Grades dargestellt und die beiden Peaks des charakteristischen Spektrums durch eine summierte Gaußfunktion. Das Gesamte
  Spektrum ergibt sich aus der Summe der beiden Spektren.}
  \label{fig: fit_emissionsspektrum}
\end{figure}
Für den linken Peak ergeben sich Mittelwert $x\ua{0, 1}$ und Standardabweichung $\sigma\ua{1}$ zu
\begin{align}
  x\ua{0, 1} &= \SI{9.073(2)}{\kilo\eV} & \sigma\ua{1} &= \SI{66(2)}{\eV}.
\intertext{Entsprechend für den rechten Peak}
  x\ua{0, 2} &= \SI{8.1861(4)}{\kilo\eV} & \sigma\ua{2} &= \SI{47.8(7)}{\eV}.
\end{align}

\end{document}
