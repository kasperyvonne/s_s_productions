\section{Diskussion}
Im Folgenden sollen die Messergebnisse in Bezug auf die Messgenauigkeit des %Messergebnisse in Bezug
Vesuchsaufbaus diskutiert werden.
Zusammenfassend findet man in Tabelle \ref{tab:resultate} die Resultate. %Zusammenfassend
\begin{table}
\centering
\caption{Messergebnisse.}
\label{tab:resultate}
\begin{tabular}{S S S}
\toprule
{Stoff} & {Bestimmte Halbwertszeit $T$ in $\si{\second}$} & {Theoretischehalbwertszeit $T\ua{theo}$ in $\si{\second}$} \\
\text{Indium} \, $\ce{^{116}_{49}In}$  & 3.53\pm0.12 e+3  & 3.24 e+3 \\
\text{Rhodium}\, $\ce{^{104i}_{45} Rh}$  & 3.4\pm2.6 e+2  & 2.6 e+2 \\
\text{Rhodium} $\ce{^{104}_{45} Rh}$  &  54.3\pm2.0 & 42.3 \\
\bottomrule
\end{tabular}
\end{table}
Die Unterschiede zu der Theorie, insbesondere bei $\ce{^{104i}_{45} Rh}$ sind %Satz umgangssprachlich formuliert... 'erstmal nicht'
grundsätzlich nicht mit dem Versuchaufbau zu erklären. Die in der Abbildung \ref{fig: plot_rhodium_lang}
zu erkennbaren großen Abweichung, sind wahrscheinlich auf statistische Fehler zurückzuführbar.
Eine Reduzierung der Ungenauigkeit lässt sich durch erneute Messungen %zurückzuführen oder zurück zu führen
verringern. %verringern

Abschließend ist zu sagen, dass der Versuch bei den Stoffen $\ce{^{116}_{49}In}$ und $\ce{^{104}_{45} Rh}$
gute Ergebnisse liefert. Lediglich bei $\ce{^{104i}_{45} Rh}$ sollte eine neue %Ergebnisse
Messung durchgeführt werden.
