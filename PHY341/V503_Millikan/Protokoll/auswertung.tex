\section{Auswertung}
Die gemessenen Thermowiderstände $R$, Fallzeiten $t$ und Spannungen $U$ sind in Tabelle \ref{} einzusehen.
Die aus der Anleitung entnommenen Werte Paare $(R, T)$ (siehe Tabelle x) wurden benutzt um den Zusammenhang
zwischen dem Thermowiderstand und und der Temperatur mittels Polynominterpolation zu approximieren. Die graphische
Darstellung des gefundenen Polynoms befindet sich in Abbildung \ref{}. Die mit dem Interpolationspolynom berechneten
Temperaturen sind ebenfalls in Tabelle \ref{} eingefügt. Der Zusammenhang zwischen der Viskosität von Luft und
der Temperatur wurde der Anleitung \cite{} entsprechend als linear angenommen. Mit den Wertepaaren
\begin{align}
  \eta_1 &= \SI{1.85e-5}{\newton\second\meter^{-2}}, \quad T_1 = \SI{16}{\celsius} \\
  \eta_2 &= \SI{1.88e-5}{\newton\second\meter^{-2}}, \quad T_2 = \SI{32}{\celsius},
\end{align}
die dem Graphen aus der Anleitung entnommen wurden, wird eine Geradengleichung bestimmt, die den bestimmten Temperaturen
$T$ in $\si{\celsius}$ die Viskositäten $\eta$ in $\si{\newton\second\meter^{-2}}$ zuordnet.
\begin{equation}
  \eta(T) = \SI{0.001875}{\newton\second\meter^{-2} \celsius^{-1} } * T  + \SI{1.82}{\newton\second\meter^{-2}}.
\end{equation}
Die somit berechneten Viskositäten sind in Tabelle \ref{} aufgetragen.
Die Fallgeschwindigkeiten der Tröpfchen berechnen sich gemäß \eqref{} und der Falldistanz $s = \SI{0.5}{\milli\meter}$.
Mit Gleichung \eqref{} ergeben sich somit schließlich die Tröpfchenradien und die die Ladungen $q$. Hierbei wurde
direkt die Korrektur \eqref{} angewandt. Alle Ergebnisse sind in Tabelle \ref{} eingetragen. \\
Um nun 



\begin{table}[H]
\centering
\caption{Gemessene und brechnete Größen für einzelne beobachtete Tropfen. Thermowiderstand $R$, Temperatur $T$, Luftviskosität $\eta$, Tröpfchenradius $r$ (korrigiert), Fallzeit $t$, Fallgeschwindigkeit $v_0$, Schwebespannung $U$ und Ladung $q$.}
\label{tab: data}
\begin{tabular}{S S S S S S S S }
\toprule
{$R/\si{\mega\ohm}$} & {$T/\si{\celsius}$} & {$\eta/10^{-5}\si{\newton\second\meter^{-2}}$} & {$r/\si{\micro\meter}$} & {$t/\si{\second}$} &
 {$v_0/\si{\centi\meter\second^{-1}}$} & {$U/\si{\volt}$}  & {$q/10^{-19}\si{\coulomb}$}  \\
\midrule
 1.89  & 27.26  & 1.87  & 0.6  & 10.525  & 0.005  & 79.60  & 9.10\\
1.88  & 27.48  & 1.87  & 0.6  & 11.498  & 0.004  & 68.80  & 9.15\\
1.85  & 28.17  & 1.87  & 0.6  & 10.807  & 0.005  & 61.50  & 11.31\\
1.83  & 28.63  & 1.87  & 0.7  & 9.504  & 0.005  & 144.00  & 5.93\\
1.81  & 29.11  & 1.87  & 0.7  & 9.251  & 0.005  & 143.60  & 6.21\\
1.81  & 29.11  & 1.87  & 0.6  & 12.489  & 0.004  & 97.90  & 5.65\\
1.81  & 29.11  & 1.87  & 0.9  & 5.147  & 0.010  & 86.60  & 25.89\\
1.80  & 29.36  & 1.88  & 0.4  & 28.793  & 0.002  & 22.30  & 6.41\\
1.79  & 29.60  & 1.88  & 0.5  & 16.228  & 0.003  & 130.40  & $\textcolor{red}{\num{2.79}}$\\
1.79  & 29.60  & 1.88  & 0.5  & 15.588  & 0.003  & 70.20  & 5.53\\
1.78  & 29.85  & 1.88  & 1.0  & 4.940  & 0.010  & 184.40  & 12.98\\
1.77  & 30.11  & 1.88  & 0.4  & 23.773  & 0.002  & 26.70  & 7.34\\
1.77  & 30.11  & 1.88  & 0.5  & 16.099  & 0.003  & 280.00  & $\textcolor{red}{\num{1.32}}$\\
1.76  & 30.36  & 1.88  & 0.7  & 10.014  & 0.005  & 98.60  & 7.99\\
1.76  & 30.36  & 1.88  & 0.4  & 21.332  & 0.002  & 207.00  & $\textcolor{red}{\num{1.13}}$\\
1.76  & 30.36  & 1.88  & 0.6  & 10.561  & 0.005  & 79.70  & 9.09\\
1.75  & 30.62  & 1.88  & 0.6  & 13.247  & 0.004  & 53.80  & 9.38\\
1.75  & 30.62  & 1.88  & 0.6  & 13.929  & 0.004  & 104.40  & 4.46\\
1.74  & 30.89  & 1.88  & 0.7  & 7.828  & 0.006  & 197.00  & 5.91\\
1.74  & 30.89  & 1.88  & 0.6  & 13.804  & 0.004  & 301.00  & $\textcolor{red}{\num{1.57}}$\\
1.74  & 30.89  & 1.88  & 0.5  & 15.460  & 0.003  & 35.00  & 11.26\\
1.74  & 30.89  & 1.88  & 0.7  & 9.605  & 0.005  & 83.40  & 10.10\\
1.73  & 31.16  & 1.88  & 0.5  & 16.007  & 0.003  & 60.60  & 6.15\\
1.73  & 31.16  & 1.88  & 0.5  & 14.304  & 0.003  & 97.10  & 4.60\\
1.73  & 31.16  & 1.88  & 0.6  & 13.946  & 0.004  & 107.90  & $\textcolor{red}{\num{4.31}}$\\
\bottomrule
\end{tabular}
\end{table}

