\section{Diskussion}
Im Folgenden sollen die Ergebnisse der Messungen interpretiert und in Beziehung zur Präzision des verwendeten Aufbaus gestellt werden. \\
Am Graphen \ref{fig: amplitude}, der die Einhüllende der gedämpften Schwingung darstellt, kann der in der Theorie beschriebene exponentiell abklingende
Verlauf deutlich erkannt werden. Der aus dieser Kurve bestimmte effektive Widerstand weicht in plausibler Weise vom verbauten Widerstand ab.\\
Die experimentell bestimmten Werte aus der Messung der Resonanzamplitude, etwa die Resonanzfrequenz (proz. Abweichung $\SI{-3.19}{\percent}$),
stimmen in guter Näherung mit den theoretischen Werten überein. Dies ist auf die hohe Messpräzision des digitalen Oszilloskops zurückzuführen.
Die berechneten Abweichungen lassen sich durch nicht betrachtete Innenwiderstände und sonstige systematische Fehler erklären. Ähnliche Resultate
wurden mittels der Beobachtung der Frequenzabhängigkeit der Phasendifferenz erzielt. Insbesondere die Resonanzfrequenz (proz. Abweichung $\SI{-0.26}{\percent}$) konnte
sehr genau bestimmt werden. Dennoch ist die diese Methode als weniger präzise einzustufen, was sich in der Abweichung der Resonanzbreite
zeigt ($\SI{-18.96}{\percent}$). Gerade bei hohen Frequenzen ist die Ausmessung der zeitlichen Abstände mit Hilfe der Cursor-Einstellung des Oszilloskops
nur sehr ungenau möglich. Die präzise Bestimmung der Resonanzfrequenz ist hier also eher als zufällig einzustufen. \\
Eine größere Präzision bei der Messung des Widerstandes $R\ua{AP}$ wäre zu erzielen, wenn man sich an dieser Stelle nicht auf die Skala verlassen, sondern
die Größe mit einem seperaten Messgerät aufnehmen würde.\\
Insgesamt konnte die Phänomenologie gedämpfter und erzwungener Schwingungen im Rahmen der angesprochenen Messpräzision sehr gut beobachtet und untersucht werden.
