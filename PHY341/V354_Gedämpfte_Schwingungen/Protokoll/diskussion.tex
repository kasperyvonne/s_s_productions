\section{Diskussion}
Die gewonnen Ergebnisse, sowie berechnete Theoriewerte und mittlere prozentuale Abweichungen sind in Tabelle \ref{tab: results} aufgeführt. \\
Der gefundenen Wert für den effektiven Widerstand der Schaltung weicht um etwa $\SI{51}{\ohm}$ von dem verbauten Widerstand ($R =  \SI{48}{\ohm}$) ab. Dies kann
auf nicht betrachtete Innenwiderstände des Generators bzw. anderer Bauteile zurückgeführt werden und stellt
gleichzeitig eine Erklärung für die deutliche prozentuale Abweichung der charakteristischen Abklingzeit $T\ua{ex}$ vom Theoriewert dar.\\
Die experimentell bestimmten Werte aus der Messung der Resonanzamplitude, etwa die Resonanzfrequenz (proz. Abweichung $\SI{-3.19}{\percent}$),
stimmen in guter Näherung mit den theoretischen Werten überein. Dies ist auf die hohe Messpräzision des digitalen Oszilloskops zurückzuführen.
Die berechneten Abweichungen lassen sich durch nicht betrachtete Innenwiderstände und sonstige systematische Fehler erklären. Des Weiteren zeigt
sich in der graphischen Darstellung der gesamten Messreihe \ref{fig: spannungsverlauf_U_C_log} der charakteristische Verlauf einer Resonanzerscheinung.
Für kleine Frequenzen unterscheiden sich die Spannung am Kondensator $U\ua{C}$ und die Errergerspannung $U$ nur geringfügig, in der nähe
der Resonanzfrequenz kommt es dann zu einem starken Anstieg.\\
Ähnliche Resultate wurden mittels der Beobachtung der Frequenzabhängigkeit der Phasendifferenz erzielt.
Insbesondere die Resonanzfrequenz (proz. Abweichung $\SI{-0.26}{\percent}$) konnte
sehr genau bestimmt werden. Dennoch ist die diese Methode als weniger präzise einzustufen, was sich in der Abweichung der Resonanzbreite
zeigt ($\SI{-18.96}{\percent}$). Gerade bei hohen Frequenzen ist die Ausmessung der zeitlichen Abstände mit Hilfe der Cursor-Einstellung des Oszilloskops
nur sehr ungenau möglich. Die präzise Bestimmung der Resonanzfrequenz ist hier also eher als zufällig einzustufen. Dennoch zeigt sich im Graph \ref{fig: phasenverlauf}
der erwartete Verlauf der Phasendifferenz. Für kleine Frequenzen besteht keine Phasendifferenz zwischen Erreger und Resonator (siehe Abb. \ref{fig: phasenverlauf_log}).
Die Funktion hat einen Nulldurchgang bei der Resonanzfrequenz und übersteigt den Wert $\pi$ nicht.\\
Die große Abweichung für die Größe $R\ua{AP}$ lässt sich durch die ungenaue Messung des Widerstandes erklären. Einerseits kann der gesuchte Punkt,
an dem es zu keinen Überschwingern mehr kommt, nur sehr schwer erkannt werden. Darüber hinaus stellt die Skala, an der der Widerstand
letztendlich abgelesen wurde eine Quelle für systematische Fehler dar. Eine größere Präzision bei der Messung des Widerstandes $R\ua{AP}$ wäre zu erzielen,
wenn ein gesondertes Messgerät zur Bestimmung des Widerstandes verwendet würde.
Insgesamt konnte die Phänomenologie gedämpfter und erzwungener Schwingungen im Rahmen der angesprochenen Messpräzision sehr gut beobachtet und untersucht werden.
g\ua{r, \sigma} = \SI{+1.044970215792965(0)}{}

