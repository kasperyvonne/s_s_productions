\section{Auswertung}
Zur qualtitativen Diskussion der Messergebnisse sollen im folgenden Abschnitt relevante Rechnungen mit den gefundenen
Werten durchgeführt werden. Fehlerfortpflanzungen, sowie Ausgleichsrechnungen werden mit
Hilfe der Python Bibliotheken \emph{uncertainties}\cite{uncertainties} und \emph{scipy}\cite{scipy} bestimmt.
\subsection{Untersuchung der Spannungsamplitude}
Die Messwerte zur Untersuchung des zeitlichen Verlaufs der Kondensatorspannungsamplitude $A(t)$ sind in Tabelle \ref{tab: amplitude} aufgetragen. Wie in der Theorie erwähnt,
lässt sich der Verlauf durch eine Funktion der Form
\begin{equation}
  A(t) = A_0 \, \exp[-B\cdot t]
\end{equation}
darstellen. Durch Anwenden des Logarithmus erhält man einen linearen Zusammenhang der Form:
\begin{equation}
  log \left( \frac{A(t)}{A_0} \right) = - B\cdot t
\end{equation}
Der Wert $A_0$ wurde der Messreihe entnommen.
\begin{equation}
  A_0 = \SI{9.84}{\volt}.
\end{equation}
Mittels einer linearen Regressionsrechnung erhält man für den Parameter $B$:
\begin{equation}
  B = \SI{4.9(27)e3}{\second^{-1}}.
\end{equation}
Die Fit-Funktion, sowie die Messwerte sind in Abbildung \ref{fig: amplitude} dargestellt.
Mit dem Parameter $B$ lässt sich der effektive Widerstand $R\ua{eff}$ bestimmen.
\begin{equation}
  B = \frac{R\ua{eff}}{2 L} \quad \Rightarrow \quad R\ua{eff} =  \SI{99(5)}{\ohm}.
  \label{eq: Fitparameter}
\end{equation}
Des Weiteren ergibt sich aus der Ausgleichsrechnung für die Abklingzeit $T\ua{ex}$:
\begin{equation}
  B = \frac{1}{T\ua{ex}} \quad \Rightarrow \quad T\ua{ex} =  \SI{2.04(11)e-4}{\second}.
\end{equation}
\begin{table} 
\centering 
\caption{Zeitlicher Verlauf der Amplitude des gedämpften Schwingkreises.} 
\label{tab: amplitude} 
\begin{tabular}{S S } 
\toprule  
{$t$ / $10^{-5}\si{\second}$} & {$A$ / $\si{\volt}$}  \\ 
\midrule  
 0.0  & 9.84\\ 
3.0  & 9.04\\ 
6.0  & 8.40\\ 
9.0  & 7.92\\ 
12.0  & 7.52\\ 
15.0  & 7.12\\ 
18.0  & 6.80\\ 
21.0  & 6.56\\ 
24.0  & 6.16\\ 
27.0  & 5.92\\ 
30.0  & 5.76\\ 
33.0  & 5.68\\ 
36.0  & 5.52\\ 
\bottomrule 
\end{tabular} 
\end{table}
\begin{figure}
  \centering
  \includegraphics[width = \textwidth]{../Messdaten/amplitude.pdf}
  \caption{Zeitlicher Verlauf der Kondensatorspannungsamplitude des gedämpften Schwingkreises und Fit-Funktion.}
  \label{fig: amplitude}
\end{figure}

\subsection{Frequenzabhängigkeit der Kondensatorspannung}
Die gemessenen Werte der Generator- und Kondensatorspannung unter variabler Frequenz sind in Tabelle \ref{tab: frequenzabhängigkeit} aufgeführt.
Die gesamte Messreihe ist in Abbildung \ref{fig: spannungsverlauf_U_C_log} mit logarithmischer $x$-Achse graphisch dargestellt. Eine lineare Darstellung
der Werte um die Resonanzfrequenz ist in Abbildung \ref{fig: spannungsverlauf_U_C}
einzusehen. Für die Resonanzfrequenz ergibt sich:
\begin{equation}
  \nu\ua{0, exp} = \SI{33}{\kilo\hertz}.
  \label{eq: exp_resonanzfrequenz}
\end{equation}
Des Weiteren werden aus der graphischen Darstellung die Größen $\nu_+$ und $\nu_-$ entnommen, aus denen die Breite der Resonanzkurve bestimmt werden kann.
\begin{align}
  \begin{aligned}
  \nu_{+} &= \SI{3.745e4}{\hertz} \\
  \nu_{-} &= \SI{2.860e4}{\hertz} \\
  (\nu_{+}-\nu_{-})\ua{exp} &= \SI{8.85e3}{\hertz}.
\end{aligned}
\label{eq: breite_exp}
\end{align}
\begin{figure}
  \centering
  \includegraphics[width = \textwidth]{../Messdaten/U_f_linear.pdf}
  \caption{Verlauf der normierten Spannung am Kondensator unter variabler Frequenz, sowie eingezeichnete Hilfslinien zum Ablesen der relevanten Größen.}
  \label{fig: spannungsverlauf_U_C}
\end{figure}

\begin{figure}
  \centering
  \includegraphics[width = \textwidth]{../Messdaten/U_f_log.pdf}
  \caption{Logarithmische Darstellung: Verlauf der normierten Spannung am Kondensator unter variabler Frequenz, sowie eingezeichnete Hilfslinien zum Ablesen der relevanten Größen.}
  \label{fig: spannungsverlauf_U_C_log}
\end{figure}

Für die Resonanzüberhöhung $q$ wird folgender Wert aus den Ergebnissen entnommen:
\begin{equation}
    q\ua{exp} = \SI{3.94}.
\end{equation}

\subsection{Frequenzabhängigkeit der Phasenverschiebung}
Die mit dem Oszilloskop bestimmten Werte für die Phasenlänge der Generatorspannung $b$ und den zeitlichen Versatz $a$ zur Kondensatorspannung, sowie die gemäß \eqref{eq:phasen_unterschied}
errechnete Phasenverschiebung $\varphi$ sind ebenfalls in Tabelle \ref{tab: frequenzabhängigkeit} aufgetragen. Auch hier wurde der Gesamte
Messbereich logarithmisch dargestellt (siehe Abbildung \ref{fig: phasenverlauf_log}). Eine graphische Darstellung der Werte bei hohen Frequenzen befindet sich in
Abbildung \ref{fig: phasenverlauf}. Aus der Abbildung bzw. den Daten wird folgender Wert, bei dem die Phasendifferenz $\frac{\pi}{2}$ beträgt, entnommen:
\begin{equation}
  \nu_0 = \SI{34}{\kilo\hertz}.
\end{equation}
Weiter werden als Werte für $\nu_+$ und $\nu_-$ jene Frequenzen abgelesen, bei denen die Phasenverschiebung
gerade $\frac{3\pi}{4}$ bzw. $\frac{\pi}{4}$ beträgt. Hieraus kann erneut die Breite bestimmt werden:
\begin{align}
  \begin{aligned}
  \nu_{+} &= \SI{30.0}{\kilo\hertz} \\
  \nu_{-} &= \SI{36.5}{\kilo\hertz} \\
  (\nu_{+}-\nu_{-})\ua{exp} &= \SI{6.5e3}{\hertz} \\
\end{aligned}
\end{align}
\begin{table}
\centering
\caption{Messwerte der Zeitparameter a und b, Kondensator- und Generatorspannung, sowie die errechnete Phasendifferenz in Abhängigkeit von der Frequenz.}
\label{tab: frequenzabhängigkeit}
\begin{tabular}{S S S S S S }
\toprule
{$\nu$ / \si{\hertz}} & {$U$ / $\si{\volt}$} & {$U_C$ / $\si{\volt}$} & {$a$ / $\si{\micro\second}$} & {$b$ / $\si{\micro\second}$} & {$\phi$ / rad}  \\
\midrule
 10  & 4.64  & 4.64  & 0.00  & \num{1.00e5}  & 0.00\\
25  & 4.64  & 4.64  & 0.00  & \num{4.00e4}  & 0.00\\
500  & 4.64  & 4.60  & 0.00  & \num{2.00e3}  & 0.00\\
1000  & 4.64  & 4.64  & 0.00  & \num{1.00e3}  & 0.00\\
5000  & 4.68  & 4.80  & 0.00  & 200.00  & 0.00\\
5500  & 4.72  & 4.84  & 0.00  & 180.00  & 0.00\\
6000  & 4.80  & 4.84  & 1.10  & 167.00  & 0.04\\
6500  & 4.80  & 4.88  & 1.10  & 153.00  & 0.05\\
7000  & 4.72  & 4.92  & 1.20  & 142.00  & 0.05\\
25000  & 4.56  & 9.20  & 3.16  & 40.00  & 0.50\\
26000  & 4.60  & 9.88  & 3.44  & 38.48  & 0.56\\
27000  & 4.60  & 10.80  & 3.76  & 36.92  & 0.64\\
28000  & 4.48  & 11.80  & 4.18  & 35.75  & 0.73\\
29000  & 4.44  & 12.90  & 3.56  & 34.48  & 0.65\\
30000  & 4.48  & 14.10  & 4.06  & 33.30  & 0.77\\
31000  & 4.44  & 15.30  & 6.71  & 32.20  & 1.31\\
32000  & 4.46  & 16.40  & 6.71  & 31.26  & 1.35\\
33000  & 4.32  & 17.00  & 6.40  & 30.27  & 1.33\\
34000  & 4.32  & 16.90  & 7.36  & 29.46  & 1.57\\
35000  & 4.32  & 15.90  & 9.32  & 28.50  & 2.05\\
36000  & 4.36  & 14.60  & 9.80  & 27.85  & 2.21\\
37000  & 4.40  & 13.00  & 10.30  & 26.95  & 2.40\\
38000  & 4.48  & 11.50  & 10.50  & 26.32  & 2.51\\
39000  & 4.48  & 10.30  & 9.72  & 25.64  & 2.38\\
40000  & 4.52  & 9.04  & 9.80  & 24.92  & 2.47\\
41000  & 4.56  & 8.08  & 10.40  & 24.38  & 2.68\\
42000  & 4.52  & 7.32  & 10.30  & 23.92  & 2.71\\
43000  & 4.56  & 6.60  & 9.70  & 23.24  & 2.62\\
44000  & 4.56  & 6.00  & 9.40  & 22.74  & 2.60\\
45000  & 4.60  & 5.48  & 9.60  & 22.27  & 2.71\\
\bottomrule
\end{tabular}
\end{table}


\begin{figure}
  \centering
  \includegraphics[width = \textwidth]{../Messdaten/phase_f_linear.pdf}
  \caption{Verlauf der Phasendifferenz zwischen Erreger- und Kondensatorspannung unter variabler Frequenz.}
  \label{fig: phasenverlauf}
\end{figure}
\begin{figure}
  \centering
  \includegraphics[width = \textwidth]{../Messdaten/phase_f_log.pdf}
  \caption{Logaritmische Darstellung: Verlauf der Phasendifferenz zwischen Erreger- und Kondensatorspannung unter variabler Frequenz.}
  \label{fig: phasenverlauf_log}
\end{figure}

\subsection{Bestimmung des Widerstandes $R\ua{AP}$}
Aus der Messung wurde folgender Wert für den Widerstand $R\ua{AP}$ gewonnen:
\begin{equation}
  R\ua{AP, exp} = \SI{28.0}{\kilo\ohm}.
\end{equation}
