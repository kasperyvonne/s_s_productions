\section{Versuchsaufbau/-durchführung}
Bei den Teilversuchen \emph{Zeitabhängigkeit der Schwingungsamplitude}
und \emph{Widerstandsbestimmung für den ap. Grenzfall} wird der selbe
Versuchsaufbau \ref{fig:aufbau_eins} verwendet. Lediglich der Widerstand ist bei dem
zweiten Teilversuch variabel einstellbar.
\begin{figure}
  \centering
  \includegraphics[width = 0.5\textwidth]{bilder/erster_versuchsaufbau.png}
  \caption{Versuchsaufbau für die Untersuchung der Schwingungsamplitude und des ap. Grenzfalles \cite{anleitung354}. }
  \label{fig:aufbau_eins}
\end{figure}
\subsection{Zeitabhängigkeit der Schwingungsamplitude}
Am Anfang muss die Frequenz des Generators %Generators
so eingestellt werden, dass die Kondensatorspannung um den Faktor $3$ bis $8$
abgenommen hat. Mit Hilfe der 'Cursor'-Funktion des Oszilloskops, wird dann %Mit Hilfe, beides ist möglich
die Höhe von allen oberen Schwingungsbäuchen gemessen.
\subsection{Widerstandsbestimmung für den ap. Grenzfall}
Um den Wiederstand zu bestimmen, an dem sich der aperiodische Grenzfall
einstellt. Wird zunächst das Potentionmeter auf seinen Maximalwert justiert.
Der Widerstand wird nun kontinuierlich verringert, dabei wird das %verringert
Ozsilloskop beobachtet. Man dreht den Widerstand so weit runter, bis %so weit
ein Überschwinger ($\frac{\map{d}U\ua{C}}{\map{d}t}>0$) entdeckt wird.
Der Wiederstand wird nun soweit erhöht, bis der Überschwinger gerade verschwindet.
An dieser Stelle befindet sich der aperiodische Grenzfall und somit $R\ua{ap}$. %befindet, aperiodische
\subsection{Frequenzabhängigkeit der Kondensatorspannung und Phase}
\begin{figure}
  \centering
  \includegraphics[width = 0.5\textwidth]{bilder/aufbau_zwei.png}
  \caption{Versuchsaufbau für die Untersuchung der Frequenzabhängigkeit der Kondensatorspannung und Phase \cite{anleitung354}. }
  \label{fig:aufbau_zwei}
\end{figure}
Der Versuch wird nach Abbildung \ref{fig:aufbau_zwei} aufgebaut.
Die am Generator anliegende Frequenz wird im Intervall $f\in\left[25,45\right]\,\si{\kilo\hertz}$
varriiert. Mit Hilfe der 'Measure'-Funktion des Ozsilloskops %Mit Hilfe, Oszilloskops
kann Generator- und Kondensatorspannung bei verschiedenen Frequenzen %verschiedenen
abgelesen werden. Zusätzlich wird mit ihr die Phasenlänge $b$ der Generatorspannung
gemessen. Abschließend bietet die 'Cursor'-Funktion eine Möglichkeit %ohne um
die zeitliche Differenz $a$ zwischen Generator- und Kondensatorspannung zu
bestimmen.
Mit den Größen $a$ und $b$ kann dann der Phasenunterschied zwischen
Generator- und Kondensatorspannung errechnet werden:
\begin{equation}
  \label{eq:phasen_unterschied}
  \varphi=\frac{2a}{b}\pi.
\end{equation}
%schau mal in die pdf, an vielen Stellen sind Zeilen eingerückt
