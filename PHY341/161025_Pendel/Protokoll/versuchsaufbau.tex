\section{Versuchsdurchführung}
Für den Versuch werden zwei Pendel und eine Feder benötigt.
Die Pendel besitzen Massen mit verstellbare Höhen.
Somit kann die Länge eines Pendels verändert werden.
Ein Pendel wird reibungsarm mit einer Spitzenlagerung
montiert. Diese Lagerung zeichnet sich dadurch aus, dass zwei Nadelspitzen jeweils 
in einer Nut sitzen, damit wird eine harmonische Schwnigung ermöglicht.

Zunächst müssen die Schwingungsfrequenzen jedes einzelnen Pendels 
gemessen werden. Dazu wird es ausgelenkt und die Zeit für $5$ Schwingungen gemessen.
Nachdem $10$ Messreihen erfasst sind, wird eine Feder benutzt, um beide 
Pendel miteinander zu koppeln.
Anschließend werden die beiden einzelnen Pendel des gekoppelten Systems entweder 
in die gleiche oder in die entgegengestzte Richtung ausgelenkt.
Genau wie bei der Einzelmessung werden $10$ mal $5$ Schwingungen gemessen.

Am Ende des Experiment wird ein Pendel ausgelenkt während das Andere 
in der Ruhelage stillsteht. Nachdem das Pendel losgelassen wird kann eine Schwebung 
beobachtet werden. Gemessen wird bei diesem Versuch die Schwebungsfrequenz und die Schwingungsdauer bestehend aus
$5$ Schwingungen.

Wenn all diese Messsung mit einer Pendellänge durchgeführt wurden, verändert man durch Verschiebung des 
Gewichtes die Pendellänge und misst alle Schwingungen nochmal.