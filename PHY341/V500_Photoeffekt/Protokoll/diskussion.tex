\section{Diskussion}
Im Folgenden sollen die Messergebnisse in Bezug auf die Messgenauigkeit des
Versuchsaufbaus diskutiert werden. %Floskel

Als eine mögliche Fehlerquelle des Versuchsaufbaues ist die Empfindlichkeit der
Strommessung zu nennen. Da im $\si{\nano\ampere}$ Bereich gemessen wurde, sorgten schon kleinste %Punkt statt Komma vor da
Veränderung der Lichtintensität (z.\,B. hervorgerufen durch Erschütterungen) zu einer Veränderung der gemessenen Stromstärken. %Plural
Eine weitere Unsicherheit entsteht durch das Spektrum der Hg-Lampe,  %jedes Spektrum ist nicht perfekt diskret... dass die Farben so verwischt waren lag eher an dem optischen Aufbau. Erwähn lieber noch, dass auch Farben enthalten waren, die eigentlich gar nicht dazu gehören. Der Typ meinte doch, dass da iwelche Lufteinschlüsse drin sind oder so.
weil es teilweise zu Überlagerung von den Spektralfarben kam, war eine genaue Justierung der Photozelle erschwert.

Die in der Abbildung \ref{fig:gelb_all} zu sehende Sättigung entsteht dadurch, dass
ab einer bestimmten Beschleunigungsspannung alle Elektronen die Anode erreichen.
Lediglich eine Erhöhung der Lichtintensität würde eine Vergrößerung der Stromstärke
zur Folge haben. %zur Folge
Das asymptotische Verhalten der Stromstärke ist auf die Anode
bzw. das Streuverhalten der Elektronen zurückzuführen, denn bei der verwendeten %denn
Anode ist nicht sichergestellt, dass alle Elektronen registriert werden. %registriert
Das Problem kann zum Beispiel mit einer Anode gelöst werden, die die Photokathode
einhüllt.

Die Stromstärke fällt nicht schlagartig gegen null, da die Elektronen in der Photokathode
eine Energieverteilung besitzen. Auf Grund dessen besitzt nicht jedes ausgelöstes
Elektron dieselbe Energie. Einige Elektronen können dadurch, bis zu einem bestimmten Spannungsbereich,
die Bremsspannung überwinden.

Bei dem Photoeffekt ist auch ein positiver Photostrom beobachtbar, das
Phänomen ist erklärbar damit, dass auch bei der Anode ein lichtelektrischer Effekt
auftritt. Jedoch besitzt die Anode eine höhere Austrittsarbeit, dies sorgt
dafür dass nur einige wenige Elektronen ausgelöst werden.
Verstärkt wird der positive Photostrom durch die Kathoden selber, %das Kathoden?
denn das Kathodenmaterial verdampft schon für Temperaturen um
$T=\SI{20}{\celsius}$. Das verdampfte Kathodenmaterial kann sich auf die
Anode legen und so die Austrittsarbeit der Anode herabsetzen, folglich steigt der
positive Photostrom.

Beim Vergleich des Messergebnisses für die Konstante $\frac{\map{h}}{\map{e}}$ mit der Theorie
zeigt sich jedoch, dass die oben beschriebenden Mängel kaum ins Gewicht fallen.
Bei einer Gegenüberstellung mit dem Theoriewert\cite{scipy} %Denn
\begin{equation}
  \label{eq:abweichung}
  \frac{\map{h}}{\map{e}}\ua{exp}=\left(\num{3.8\pm 0.7}\right)\,\num{e-15}\,\si{\eV}, \quad  \frac{\map{h}}{\map{e}}\ua{theo}=\num{4.1 e-15}\,\si{\eV}, \quad \Delta\approx -8\%
\end{equation}
zeigt sich eine geringe Abweichung zwischen Experiment und Theorie.
Dennoch könnte eine Verbesserung des Versuchsaufbaus
für eine Verringerung der Fehler in \eqref{eq:messergebnisse} sorgen.

Zusammenfassend liefert der Versuch eine gute Möglichkeit den Photoeffekt zu untersuchen, %zu
durch eine Verbesserung des Versuchsaufbaus wäre eine genaue Bestimmung von $\frac{\map{h}}{\map{e}}$
und der Austrittsarbeit denkbar.
