to\section{Auswertung}

Die im Folgenden durchgeführten Regressionsrechnungen werden mit dem
Python Paket \emph{scipy.optimize}\cite{scipy} durchgeführt. %iwas klappt mit dem zitieren nich

\subsection{Untersuchung der Gegenspannunng für verschieden farbiges Licht}
\FloatBarrier
Die aufgenommenen Messwerte sind in den Tabellen \ref{tab: gelb} - \ref{tab: uv_zwei} aufgelistet.
\begin{table} 
\centering 
\caption{Gemessener Photostrom bei gelbem Licht} 
\label{tab: gelb} 
\begin{tabular}{S S S } 
\toprule  
{Bremsspannung $U$ in $\si{\volt}$} & {Photostrom $I_{\map{p}}$ in $\si{\nano\ampere}$} & {Photostrom $\sqrt{I_{\map{p}}}$ in $\sqrt{\si{\nano\ampere}}$}  \\ 
\midrule  
 0.001  & 0.14  & 0.37\\ 
0.020  & 0.12  & 0.35\\ 
0.041  & 0.10  & 0.32\\ 
0.060  & 0.10  & 0.32\\ 
0.080  & 0.08  & 0.28\\ 
0.100  & 0.07  & 0.26\\ 
0.120  & 0.06  & 0.24\\ 
0.140  & 0.05  & 0.22\\ 
0.160  & 0.04  & 0.20\\ 
0.181  & 0.03  & 0.17\\ 
0.200  & 0.03  & 0.17\\ 
0.220  & 0.02  & 0.14\\ 
0.240  & 0.02  & 0.13\\ 
0.260  & 0.01  & 0.12\\ 
0.280  & 0.01  & 0.10\\ 
0.300  & 0.01  & 0.09\\ 
0.320  & 0.01  & 0.08\\ 
\bottomrule 
\end{tabular} 
\end{table}
\begin{table} 
\centering 
\caption{Gemessener Photostrom bei grünem licht} 
\label{tab: gruen} 
\begin{tabular}{S S S } 
\toprule  
{Bremsspannung $U$ in $\si{\volt}$} & {Photostrom $I_{\map{p}}$ in $\si{\nano\ampere}$} & {Photostrom $\sqrt{I_{\map{p}}}$ in $\sqrt{\si{\nano\ampere}}$}  \\ 
\midrule  
 0.00  & 0.32  & 0.57\\ 
0.02  & 0.30  & 0.55\\ 
0.04  & 0.30  & 0.55\\ 
0.06  & 0.28  & 0.53\\ 
0.08  & 0.24  & 0.49\\ 
0.10  & 0.20  & 0.45\\ 
0.12  & 0.20  & 0.45\\ 
0.14  & 0.18  & 0.42\\ 
0.16  & 0.16  & 0.40\\ 
0.18  & 0.12  & 0.35\\ 
0.20  & 0.10  & 0.32\\ 
0.25  & 0.08  & 0.28\\ 
0.30  & 0.04  & 0.20\\ 
0.35  & 0.02  & 0.14\\ 
0.40  & 0.00  & 0.00\\ 
\bottomrule 
\end{tabular} 
\end{table}
\begin{table} 
\centering 
\caption{Gemessener Photostrom bei grün-blauem licht} 
\label{tab: gb} 
\begin{tabular}{S S S } 
\toprule  
Jo  \\ 
\midrule  
 0.001  & 0.03  & 0.18\\ 
0.050  & 0.03  & 0.17\\ 
0.102  & 0.03  & 0.16\\ 
0.150  & 0.02  & 0.15\\ 
0.200  & 0.02  & 0.13\\ 
0.250  & 0.01  & 0.12\\ 
0.300  & 0.01  & 0.11\\ 
0.350  & 0.01  & 0.10\\ 
0.400  & 0.01  & 0.09\\ 
0.451  & 0.01  & 0.08\\ 
0.500  & 0.00  & 0.06\\ 
0.551  & 0.00  & 0.04\\ 
0.602  & 0.00  & 0.04\\ 
0.653  & 0.00  & 0.03\\ 
0.700  & 0.00  & 0.00\\ 
\bottomrule 
\end{tabular} 
\end{table}
\begin{table} 
\centering 
\caption{Gemessener Photostrom bei violettem licht} 
\label{tab: violett} 
\begin{tabular}{S S S } 
\toprule  
Jo  \\ 
\midrule  
 0.001  & 0.58  & 0.76\\ 
0.102  & 0.48  & 0.69\\ 
0.200  & 0.40  & 0.63\\ 
0.301  & 0.32  & 0.57\\ 
0.401  & 0.24  & 0.49\\ 
0.503  & 0.16  & 0.40\\ 
0.604  & 0.10  & 0.32\\ 
0.702  & 0.06  & 0.24\\ 
0.801  & 0.04  & 0.20\\ 
0.902  & 0.02  & 0.14\\ 
1.001  & 0.01  & 0.10\\ 
1.103  & 0.00  & 0.00\\ 
\bottomrule 
\end{tabular} 
\end{table}
\begin{table} 
\centering 
\caption{Gemessener Photostrom beim ersten ultravioletten licht} 
\label{tab: uv_eins} 
\begin{tabular}{S S S } 
\toprule  
Jo  \\ 
\midrule  
 0.001  & 0.14  & 0.37\\ 
0.101  & 0.12  & 0.35\\ 
0.200  & 0.10  & 0.32\\ 
0.300  & 0.08  & 0.28\\ 
0.401  & 0.08  & 0.28\\ 
0.500  & 0.06  & 0.24\\ 
0.602  & 0.04  & 0.20\\ 
0.701  & 0.03  & 0.17\\ 
0.802  & 0.02  & 0.14\\ 
0.900  & 0.01  & 0.10\\ 
1.001  & 0.01  & 0.10\\ 
1.101  & 0.00  & 0.00\\ 
\bottomrule 
\end{tabular} 
\end{table}
\begin{table} 
\centering 
\caption{Gemessener Photostrom beim zweiten ultravioletten licht} 
\label{tab: uv_zwei} 
\begin{tabular}{S S S } 
\toprule  
Jo  \\ 
\midrule  
 0.020  & 1.00  & 1.00\\ 
0.100  & 0.90  & 0.95\\ 
0.200  & 0.90  & 0.95\\ 
0.400  & 0.60  & 0.77\\ 
0.600  & 0.50  & 0.71\\ 
0.800  & 0.30  & 0.55\\ 
1.000  & 0.20  & 0.45\\ 
1.200  & 0.10  & 0.32\\ 
1.400  & 0.04  & 0.20\\ 
1.600  & 0.01  & 0.10\\ 
1.800  & 0.00  & 0.00\\ 
\bottomrule 
\end{tabular} 
\end{table}
An die gemessenen Spannungen $U$ und an die Wurzel des Photostroms $I\ua{p}$ wird eine
Regressionsgerade
\begin{equation}
  \label{eq:reg} %dann nenn x doch auch U und y wurzel I
  g(x)=mx+b \quad \Leftrightarrow \quad \sqrt{I\ua{p}}=mU+b
\end{equation}
gefittet.
In den Darstellungen \ref{fig: darstellung_1} bis \ref{fig: darstellung_3} werden die Ergebnisse der Regressionsrechung
illustriert.
\begin{figure}
  \centering
  \begin{subfigure}{0.48\textwidth}
    \centering
    \includegraphics[width=1 \textwidth]{../Messdaten/gelbem.pdf}
    \caption{Gelbes Licht.}
    \label{fig: gelb}
  \end{subfigure}
  \begin{subfigure}{0.48\textwidth}
    \centering
    \includegraphics[width=1 \textwidth]{../Messdaten/grünem.pdf}
    \caption{Grünes Licht.}
    \label{fig: grün}
  \end{subfigure}
  \caption{Darstellung der linearen Abhängigkeit von $\sqrt{I\ua{p}}$ zur Bremsspanung $U$.}
  \label{fig: darstellung_1}
\end{figure}
\begin{figure}
  \centering
  \begin{subfigure}{0.48\textwidth}
    \centering
    \includegraphics[width=1 \textwidth]{../Messdaten/grün-blauem.pdf}
    \caption{Grün-blaues Licht.}
    \label{fig: grün-blau}
  \end{subfigure}
  \begin{subfigure}{0.48\textwidth}
    \centering
    \includegraphics[width=1 \textwidth]{../Messdaten/violettem.pdf} %benenn die konsistent
    \caption{Violettes Licht.}
    \label{fig: violett}
  \end{subfigure}
  \caption{Darstellung der linearen Abhängigkeit von $\sqrt{I\ua{p}}$ zur Bremsspannung $U$.}
  \label{fig: darstellung_2}
\end{figure}
\begin{figure}
  \centering
  \begin{subfigure}{0.48\textwidth}
    \centering
    \includegraphics[width=1 \textwidth]{../Messdaten/uv_eins.pdf}
    \caption{UV 1 Licht.}
    \label{fig: uv_eins}
  \end{subfigure}
  \begin{subfigure}{0.48\textwidth}
    \centering
    \includegraphics[width=1 \textwidth]{../Messdaten/uv_zwei.pdf}
    \caption{UV 2 Licht.}
    \label{fig: uv_zwei}
  \end{subfigure}
  \caption{Darstellung der linearen Abhängigkeit von $\sqrt{I\ua{p}}$ zur Bremsspannung $U$.}
  \label{fig: darstellung_3}
\end{figure}
Die Parameter derAusgleichsrechnung sind in Tabelle %doppelt aus;
\ref{tab: messergebnisse} aufgelistet. Die Grenzspannung $U\ua{g}$ wird
wie folgt berechnet:
\begin{equation}
  \label{eq:ug}
  U\ua{g}\overset{\ref{eq:reg}}{=}-\frac{b}{m}.
\end{equation}
\begin{table}
\centering
\caption{Messergebnisse für die verschiedenen Wellenlängen}
\label{tab: messergebnisse}
\begin{tabular}{ S S S S S S S }
\toprule
 {Farben} & {$m$ in $\si{\ampere\per\volt}$} & { $\sigma_{\map{m}}$ in $\si{\ampere\per\volt}$ } & { $b$ in $\si{\ampere}$} & { $\sigma_{\map{b}}$ in $\si{\ampere}$} & {$U_{\map{g}}$ in $\si{\volt}$ } & { $\sigma_{U_{\map{g}}}$ in $\si{\volt}$}     \\
\midrule
\text{gelb}       & -0.94  & 0.02  & 0.360  & 0.005  & 0.38  & 0.01\\
\text{grün}       &  -1.36  & 0.05  & 0.597  & 0.010  & 0.44  & 0.02\\
\text{blau-grün}                   &  -0.24  & 0.01  & 0.185  & 0.003  & 0.75  & 0.02\\
\text{violett}                  &  -0.69  & 0.01  & 0.760  & 0.010  & 1.10  & 0.03\\
\text{uv 1}                  &  -0.94  & 0.02  & 0.360  & 0.005  & 1.24  & 0.01\\
\text{uv 2}                   &  -0.58  & 0.01  & 1.023  & 0.011  & 1.77  & 0.04\\                      
\bottomrule
\end{tabular}
\end{table}

\FloatBarrier
\subsection{Bestimmung von $\frac{\map{h}}{\map{e}}$ und der Austrittsarbeit}
\FloatBarrier
Die oben bestimmten Grenzspanunngen werden im %vorangegangenden was?, benutz nicht so oft wörter die auf enden enden, klingt blöd
Folgenden genutzt, um die Konstante $\frac{\map{h}}{\map{e}}$ und die Austrittsarbeit des Kathodenmaterials
experimentell zu bestimmen.
Es wird dazu eine Regressionrechnung für $U\ua{g}(f)$ durchgeführt.
Aus der Versuchsanleitung\cite{anleitung500} werden die Wellenlängen der verschieden Farben der $\map{Hg}$-Lampe (vgl. Tabelle \ref{tab: wellen})
entnommen.
\begin{table}
\centering
\caption{Untersuchtes Lichtspektrum der $\map{HG}$-Lampe }
\label{tab: wellen}
\begin{tabular}{S S S }
\toprule
{Farbe} & {$\lambda$ in \si{\nano\meter}} & { $f$ in $\si{\THz}$ } \\
\midrule
\text{gelb} & 577.0  & 519.6\\
\text{grün} & 546.0  & 549.1\\
\text{blau-grün} & 492.0  & 609.3\\
\text{violett}  & 434.0  & 690.8\\
\text{uv 1}  & 365.0  & 821.3\\
\text{uv 2} & 366.0  & 819.1\\
\bottomrule
\end{tabular}
\end{table}

Die Umrechnung von der Wellenlänge $\lambda$ in die Frequenz $f$ erfolgt mit
\begin{equation*}
  \label{eq:umrech}
  f=\frac{\map{c}}{\lambda}.
\end{equation*}
Hierbei ist$\map{c}$ die Vaakumlichtgeschwindigkeit\cite{scipy}. %nicht sei, ist
Um $\frac{\map{h}}{\map{e}}$ und die Austrittsarbeit $A\ua{k}$ zu bestimmen, wird zunächst die
Gleichung \eqref{eq:regress} umgestellt zu:
\begin{equation*}
  \Leftrightarrow \qquad U\ua{g}=\frac{\map{h}}{\map{e}} f - \frac{A\ua{k}}{\map{e}}.
\end{equation*}
An diese kann eine Regressionsgerade der Form \eqref{eq:reg} gefittet werden.
Der Zusammenhang zwischen Fitparamtern und den gesuchten Größen lautet:
\begin{align*}
  \frac{\map{h}}{\map{e}}&=m\\
  A\ua{k}&=-b \quad \left(\frac{1}{\map{e}}\,\text{liefert die Einheit} \, \si{\eV}\right)
\end{align*}
\begin{figure}
    \centering
    \includegraphics[width=1 \textwidth]{../Messdaten/wellenlaenge_gegen.pdf}
    \caption{Experimentell bestimmten Grenzfrequenzen.}
    \label{fig:grenz}
  \end{figure}
Aus der Regressionsrechnung folgt für die Austrittsarbeit $A\ua{k}$ und die
Konstante $\frac{\map{h}}{\map{e}}$:
\begin{align}
  \label{eq:messergebnisse}
  \begin{aligned}
  \frac{\map{h}}{\map{e}}&= \left(\num{3.8\pm 0.7}\right)\,\num{e-15}\,\si{\eV} \\
  A\ua{k}&=\SI{1.6\pm 0.5}{\eV}.
  \end{aligned}
\end{align}
Die Messwerte und die Regressionsgerade sind in Abbildung \ref{fig:grenz} dargestellt.
\FloatBarrier
\subsection{Untersuchung des Photostroms bei gelbem Licht}
\FloatBarrier
Für die Untersuchung des Photostrom $I\ua{p}$ wurde die
Stromstärke für Spannungen $U$ im Bereich $U\in\left(-20,20\right)\,\si{\volt}$
notiert (vgl. Tab. \ref{tab: gelb_all}).
\begin{table} 
\centering 
\caption{Gemessener Photostrom bei gelbem Licht} 
\label{tab: gelb_all} 
\begin{tabular}{S S } 
\toprule  
{Spanung $U$ in $\si{\volt}$} & {Photostrom $I_{\map{p}}$ in $\si{\nano\ampere}$}  \\ 
\midrule  
 18.970  & 2.10\\ 
18.020  & 2.00\\ 
17.000  & 2.00\\ 
16.010  & 2.10\\ 
15.010  & 2.00\\ 
14.030  & 2.00\\ 
13.000  & 2.00\\ 
12.070  & 2.00\\ 
11.030  & 1.70\\ 
10.070  & 1.70\\ 
9.020  & 1.60\\ 
8.010  & 1.60\\ 
7.000  & 1.50\\ 
6.010  & 1.60\\ 
5.040  & 1.50\\ 
4.000  & 1.40\\ 
3.070  & 1.00\\ 
2.040  & 0.90\\ 
1.010  & 0.60\\ 
0.020  & 0.10\\ 
-0.001  & 0.14\\ 
-0.020  & 0.12\\ 
-0.041  & 0.10\\ 
-0.060  & 0.10\\ 
-0.080  & 0.08\\ 
-0.100  & 0.07\\ 
-0.120  & 0.06\\ 
-0.140  & 0.05\\ 
-0.160  & 0.04\\ 
-0.181  & 0.03\\ 
-0.200  & 0.03\\ 
-0.220  & 0.02\\ 
-0.240  & 0.02\\ 
-0.260  & 0.01\\ 
-0.280  & 0.01\\ 
-0.300  & 0.01\\ 
-0.320  & 0.01\\ 
-0.340  & 0.00\\ 
\bottomrule 
\end{tabular} 
\end{table}
In Abbildung \ref{fig:gelb_all} ist der Stromverlauf dargestellt.
\begin{figure}
    \centering
    \includegraphics[width=1 \textwidth]{../Messdaten/gelb.pdf}
    \caption{Abhängigkeit des Photostroms von der Spannung.}
    \label{fig:gelb_all}
  \end{figure}

Die in der Abbildung \ref{fig:gelb_all} zu sehende Sättigung entsteht dadurch, dass
ab einer bestimmten Beschleunigungsspannung alle Elektronen die Anode erreichen.
Lediglich eine Erhöhung der Lichtintensität würde eine Vergrößerung der Stromstärke
zur Folge haben. %zur Folge
Das asymptotische Verhalten der Stromstärke ist auf die Anode
bzw. das Streuverhalten der Elektronen zurückzuführen, denn bei der verwendeten %denn
Anode ist nicht sichergestellt, dass alle Elektronen registriert werden. %registriert
Das Problem kann zum Beispiel mit einer Anode gelöst werden, die die Photokathode
einhüllt.

Die Stromstärke fällt nicht schlagartig gegen null, da die Elektronen in der Photokathode
eine Energieverteilung besitzen. Auf Grund dessen besitzt nicht jedes ausgelöstes
Elektron dieselbe Energie. Einige Elektronen können dadurch, bis zu einem bestimmten Spannungsbereich,
die Bremsspannung überwinden.
\FloatBarrier
%Ist das ab hier nicht alles Diskussion?
