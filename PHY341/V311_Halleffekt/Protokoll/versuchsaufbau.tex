\section{Versuchsaufbau/-durchführung}
In dem Versuch sollen zwei Proben (Kupfer und Zinn) auf den \emph{Hall-Effekt} untersucht werden.

\subsection{Hysteresemessung}
Die Hysteresemessung dient dazu, die Abhängigkeit der Magnetfeldstärke 
zur Stromstärke zu bestimmen. Gleichzeitig aber auch den Effekt der Hysterese
zu bestimmen. Der Versuch wird wie in \ref{fig: auf_hall} zu sehen aufgebaut.
Mittels eines Teslameters wird nun für zehn Stromstärken das Magnetfeld bestimmt.
Die ersten $10$ Messungen werden beim Hochfahren des Magneten getätigt.
Die anderen zehn beim Herunterfahren.

\begin{figure}
  \centering
  \includegraphics[width=0.7\textwidth]{pics/Halleffekt_aufbau.png}
  \caption{Versuchsaufbau für die Messung der Hallspannung}
  \label{fig: auf_hall}
\end{figure}

\begin{center}
Die im Folgenden beschriebenen Messungen, werden immer zwei Mal durchgeführt.
Mit dem Unterschied, dass beim zweiten Durchgang die Spannung umgepolt wird.
Dies dient zur Vermeidung von systematischen Fehlern (z. B. Störspannung).
\end{center}

\subsection{Bestimmung des Widerstands}
Zur Widerstandsbestimmung werden ein Voltmeter und 
eine Spannungsquelle benötigt.
Nachdem beide an die Probe angeschlossen sind, 
wird die Spannungsquelle eingeschaltet und die Stromstärke variiert. 
Für jede Probe wird zu zehn verschiedenen Stromstärken 
die Spannung am Voltmeter notiert.
Mithilfe des ohmschen Gesetzes kann dann auf den Widerstand geschlossen werden.

\subsection{Messung der Hallspannung}
Der grundlegende Versuchsaufbau ist in Abbildung \ref{fig: auf_hall} dargestellt.
Zunächst werden die Abmessungen (Breite, Höhe und Länge) der Probe gemessen.
Anschließend folgt die Verkabelung nach dem in \ref{fig: kabel_hall} abgebildeten Schaltbild.
Die verkabelte Probe wird nun in eine Haltevorrichtung gegeben. Diese sorgt dadfür, dass 
der Stromfluss durch die Probe orthogonal zum Magnetfeld steht.
Für die Messung der \emph{Hall-Spannung} gibt es nun zwei Verfahren. 
Zum einen kann die Stromstärke, die an der Probe anliegt konstant gelassen werden und
die Magnetfeldstärke variiert werden. Zum Andern kann 
aber auch die Magnetfeldstärke konstant gelassen werden und die Stromstärke verändert
werden. Für beide Verfahren sollten zehn Messpunkte gewählt werden.
Bei der Regelung des Elektromagneten ist darauf zu achten, dass der 
erzeugende Strom langsam heruntergefahren wird.
Die von dem Hall-Effekt erezeugt Spannugn kann dann wie folgt berechnet werden:

\begin{align}
\label{eq:hall_spann}
\begin{aligned}
U\ua{ges+}=U\ua{H}+U\ua{stör}\\
U\ua{ges-}=-U\ua{H}+U\ua{stör}\\
\end{aligned}
\end{align}
Dabei sei $U\ua{stör}$ die oben angesprochende Störspannung.

\begin{figure}
  \centering
  \includegraphics[width=0.7\textwidth]{pics/verkabelung_hall.png}
  \caption{Verkabelung für die Messung der Hallspannung}
  \label{fig: kabel_hall}
\end{figure}

