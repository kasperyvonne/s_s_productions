\section{Auswertung}
Die in der Auswertung bestimmten Ausgleichsrechnungen werden mit
dem Python Paket \emph{scipy.optimize} berechnet.

\subsection{Bestimmung von RC mithilfe der Entladekurve}

Das für die Berechnung benötigte Ozsilloskopenbild ist in Abbbildung \ref{fig:entladekurve} zusehen.

\begin{figure}
  \includegraphics[width=0.8\textwidth]{pics/bilda_508hz.png}
  \caption{Vom Ozsillopskop gespeicherte Entladekurve.}
  \label{fig:entladekurve}
\end{figure}

Von dem Graphen wurden $9$ Punkte vermessen.
Diese sind in Tabelle \ref{} eingetragen.
\begin{table} 
\centering 
\caption{Aus der Grafik \ref{fig:entladekurve} bestimmte Messwerte} 
\label{tab:teil_a_spannungen} 
\begin{tabular}{S S } 
\toprule  
{$t$ in $\si{\milli\second}$} & {$U\ua{ein}$ in $\si{\volt}$}  \\ 
\midrule  
 0.0  & 6.0\\ 
0.1  & 5.2\\ 
0.2  & 4.4\\ 
0.3  & 3.7\\ 
0.4  & 3.0\\ 
0.5  & 2.3\\ 
0.6  & 1.8\\ 
0.7  & 1.3\\ 
0.8  & 0.8\\ 
\bottomrule 
\end{tabular} 
\end{table}


An die Messwerte wird ein Fit der Form

\begin{equation*}
  U(t)=a+\map{e}^{bt}+c.
\end{equation*}

gelegt.
Als Parameter ergeben sich:
\begin{equation}
  \label{eq:parameter_teila}
a=\SI{11.3\pm0.7}{\volt} \quad b=\num{-780.5\pm67.7} \quad c=\SI{-5.3\pm0.7}{\volt}.
\end{equation}
Aus der Formel  \eqref{eq:entlade_kurve} folgt weiter
\begin{equation}
  \frac{1}{b}=RC=\SI{1.12\pm0.1}{\milli\second}.
\end{equation}
Die Messwerte und der Fit sind in Abbildung \ref{fig:plot_teila} einmal grapisch
dargestellt.

\begin{figure}
  \includegraphics [width=0.8\textwidth]{pics/teil_a_entladung.pdf}
  \caption{Messwerte und Fit der Entladekurve}
  \label{fig:plot_teila}
\end{figure}

\subsection{Bestimmung von RC mithilfe der Kondensatorspannung}
Im Versuch wurden die in \ref{tab:teil_b_spannungen} dargestellten Werte gemesen.
\begin{table} 
\centering 
\caption{Gemessene Generator- und Kondensatorspannungen bei unterschiedlichen Frequenzen } 
\label{tab:teil_b_spannungen} 
\begin{tabular}{S S S S } 
\toprule  
{$f$ in $\si{\hertz}$} & {$U\ua{g}$ in $\si{\volt}$}& {$U\ua{C}$ in $\si{\volt}$} & {$\frac{U\ua{C}}{U\ua{g}}$}  \\ 
\midrule  
 11  & 10.29  & 10.29  & 1.00\\ 
25  & 10.29  & 9.90  & 0.96\\ 
50  & 10.29  & 9.21  & 0.89\\ 
75  & 10.29  & 8.41  & 0.82\\ 
100  & 10.29  & 7.37  & 0.72\\ 
200  & 10.39  & 4.95  & 0.48\\ 
401  & 10.29  & 2.81  & 0.27\\ 
600  & 10.29  & 1.94  & 0.19\\ 
800  & 10.29  & 1.38  & 0.13\\ 
1000  & 10.29  & 1.11  & 0.11\\ 
1200  & 10.29  & 0.93  & 0.09\\ 
1400  & 10.29  & 0.77  & 0.07\\ 
1600  & 10.29  & 0.71  & 0.07\\ 
1800  & 10.29  & 0.64  & 0.06\\ 
2000  & 10.29  & 0.57  & 0.06\\ 
2500  & 10.29  & 0.47  & 0.05\\ 
3000  & 10.29  & 0.40  & 0.04\\ 
3500  & 10.29  & 0.32  & 0.03\\ 
4000  & 10.29  & 0.28  & 0.03\\ 
\bottomrule 
\end{tabular} 
\end{table}

In diesem ist zugleich die Nomierung $\frac{U\ua{c}}{U\ua{g}}$
mitangegben.
Mit der Formel \eqref{eq: amplitude_theorie} kann an die Messwerte eine Regressionsgerade
errechnet werden. Hierbei wurde der Zusammenhang $\omega=2\pi f$ genutzt.
Aus der Regerssionsrechung folgt für den Wert von $RC$:
\begin{equation}
  \label{eq:rc_teilb}
  RC=\SI{1.49\pm0.02}{\milli\second}
\end{equation}
In der Abbildung \ref{fig:teilb} sind Messwerte und Regressionsgerade abgebildet.
\begin{figure}
  \includegraphics[width=0.8\textwidth]{pics/u_cdurchu_g.pdf}
  \caption{Normierte Amplitude in Abhängigkeit von der Frequenz.}
  \label{fig:teilb}
\end{figure}

\subsection{Bestimmung von RC mithilfe der Phasenverschiebung}
Die im Versuch gemessenen Abstände $a$ und $b$ in Tabelle \ref{}
dargestellt. Mithilfe von Formel \eqref{eq:phasenverschiebung} wurde die
Phasenverschiebung bestimmt.
\input{table/abstaende.tex}
Es fällt auf das die Phase $\varphi$ größer als $\frac{\pi}{2}$ wird.
Dies ist nach der Theorie nicht möglich.
Es lässt sich also schon jetzt ein systematischer Fehler nicht ausschließen.
Deshalb werden für die Berechnung von $RC$ die Messwerte weggelassen, an den
$\varphi$ größer als $\frac{\pi}{2}$ ist.
Mit der Gleichung \eqref{eq: phase_theorie} kann eine Regressionskurve für die
Phase bestimmt werden.
In Abbildung \ref{fig:plot_teil_c} sind Messwerte und Regressionskurve aufgetragen.
Dabei wird einmal die Regressionskurve mit und einmal ohne Korrekur abgebildet.
\begin{figure}
  \includegraphics[width=0.8\textwidth]{pics/frequenz_phase.pdf}
  \caption{Die von der RC-Kombination erzeugte Phasenverschiebung.}
  \label{fig:plot_teil_c}
\end{figure}
Aus der Regressionsrechung ergibt sich als Wert für $RC$

\begin{align}
  \label{eq:teil_c_rc}
  \begin{aligned}
    RC&=\SI{1.7\pm0.5}{\milli\second} \quad \text{ohne Korrektur} \\
    RC& =\SI{1.49\pm0.07}{\milli\second} \quad \text{mit Korrektur}
\end{aligned}
\end{align}
