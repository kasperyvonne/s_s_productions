\section{Diskussion}
Im folgenden Abschnitt soll die Aussagekraft der Ergebnisse im Bezug auf
die Rahmenbedingungen des Versuches diskutiert werden.
Es sei zunächst die Annahme von fehlerfreien Bauelementen (vgl. \eqref{eq: bauteile})
zu erwähnen. Die Annahme stimmt mit der Realität keines Falls überein. %bemängelt passt nicht
Die Qualität der Ergebnisse wird auf diese Weise verringert.
Hiermit lässt sich teils begründen, wieso die Abweichung zwischen Theorie und Praxis vereinzelt
so groß ($<30\%$) sind (vgl. dazu die ersten Werte in den Tabellen \ref{tab:teilb_schwingungen_prak_theo} und \ref{tab:teilc_schwingungen_prak_theo}).
Die in den Tabellen \ref{tab:teilb_schwingungen_prak_theo} und \ref{tab:teilc_schwingungen_prak_theo} gewählten Fehler, sollen
mögliche ablese Fehler berücksichtigt werden. %ungünstig formuliert
Des Weiteren ist die schlechte Skala bzw. Auflösung des Oszilloskops anzumerken.
Hierdurch war ein genaues Ablesen der Messwerte erschwert.
Die dadurch entstandene Ungenauigkeit wird durch den Fehler in Tabelle \ref{tab:teila_n_ck} berücksichtigt.
Es ist nicht auszuschließen, dass das Oszilloskop andere Teilergebnisse negativ beeinflusst.

Der Versuch liefert, bei Betrachtung der relativen Fehler, zweigeteilte  Ergebnisse.
Bei einer Betrachtung von Tab. \ref{tab:teilb_schwingungen_prak_theo} und Tab. \ref{tab:teilc_schwingungen_prak_theo} erkennt man,  %Sätze nicht mit denn anfangen lassen
dass die Abweichung zur Theorie teils unter $1\%$ liegen, manchmal aber auch Werte von $30\%$ erreichen können.
Es scheint, als ob eine Erhöhung der Kapazität $C\ua{k}$ eine höhere Genauigkeit des Versuches bewirkt (vgl. Abb. \ref{fig: plot}).
