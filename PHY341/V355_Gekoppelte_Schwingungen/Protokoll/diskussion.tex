\section{Diskussion}
Im folgenden Abschnitt soll die Aussagekfaft der Ergebnisse im Bezug auf
die Rahmenbedingungen des Versuches diskutiert werden.
Es sei zunächst die Annahme von fehlerfreien Bauelementen (vgl. \eqref{eq: bauteile})
zu erwähnen werden. Die Annahme stimmt mit der Realität keines Falls überein. %bemängelt passt nicht
Die Qualität der Ergebnisse wird auf diese Weise verringert.
Hiermit lässt sich begründen, wieso die Abweichung zwischen Theorie und Praxis teilweise
relativ groß ist (vgl. dazu die ersten Werte in den Tabellen \ref{fig:teilb_schwingungen_prak_theo} und \ref{fig:teilc_schwingungen_prak_theo}).
Die in den Tabellen \ref{fig:teilb_schwingungen_prak_theo} und \ref{fig:teilc_schwingungen_prak_theo} gewählten Fehler sind darauf zu
Mit der Abweichung sollen, mögliche ablese Fehler berücksichtigt werden. %ungünstig formuliert
Des Weiteren ist die schlechte Skala bzw. Auflösung des Oszilloskops anzumerken.
Hierdurch war ein genaues Ablesen der Messwerte erschwert.
Die dadurch entstandene Ungenauigkeit wird durch den Fehler in Tabelle \ref{fig:teila_n_ck} berücksichtigt.
Es ist nicht auszuschließen, dass das Oszilloskop andere Teilergebnisse negativ beeinflusst.

Dennoch liefert der Versuch eine relativ hohe Übereinstimmung mit den Theoriewerten.
Bei einer Betrachtung von Tab. \ref{fig:teilb_schwingungen_prak_theo} und Tab. \ref{fig:teilc_schwingungen_prak_theo} erkennt man,  %Sätze nicht mit denn anfangen lassen
dass in der berücksichtigten Signifikanz, kein Unterschied zwischen
Theorie und Experiment feststellbar ist.
Es scheint, als ob eine Erhöhung der Kapazität $C\ua{k}$ eine höhere Genauigkeit des Versuches bewirkt (vgl. Abb. \ref{fig: plot}).
