\section{Versuchsaufbau/-durchführung}
Die Brennweite einer Linse wird im Versuch $V408$ auf drei
verschiedene Weisen bestimmt.
Die Verfahren werden im Folgenden kurz erklärt.

Als Lichtquelle wird eine Halogenlampe verwendet und als
Gegenstand ein \emph{Perl L}. Mit Hilfe eines Schrimes wird
das gebrochene Licht sichtbar gemacht werden.

\subsection{Bestimmung der Brennweite mit der Linsengleichung}

Der Schirm, die Linse, das Perl L und die Lampe werden auf einer
Linie platziert.
Man verändert die Bildweite $b$ so lang, bis ein scharfes Bild auf dem
Schirm zu erkennen ist. Das Wertepaar ($b$,$g$) werden notiert.
Danach wird die Position der Linse verändert und stellt den Schirm neu ein.
Das Wertepaar wird wieder vermerkt. Der Vorgang wird so lang wiederholt, bis
zehn Wertepaare erfasst worden sind.
Mit Hilfe der Linsengleichung \eqref{eq: linsengleichung} und den Wertepaaren kann, dann die
Brennweite $f$ bestimmt werden.

\subsection{Bestimmung der Brennweite mit der Methode von Bessel}
Bei Methode von Bessel wird die Entfernung zwischen
Gegenstand und Bild festgehalten. Die Linse wird so lang verschoben, bis zwei
Positionen gefunden worden sind, an den ein scharfes Bild auf dem Schirm erkennbar ist.
Es ergeben sich folglich zwei Wertepaare ($b_1$,$g_1$) und ($b_2$,$g_2$).
Mit Hilfe der Hilfsgrößen $e=g_1+b_1=g_2=b_2$ und $d=g_1-b_1=g_2-b_2$ und der Formel
\begin{equation}
  \label{eq: bessel_methode}
  f=\frac{e^2-d^2}{4e}
\end{equation}
ist eine Berechnung der Brennweite möglich.
Nach der Aufnahme der Wertepaare wird die Linse neu positioniert.
Der Vorgang wird zehn Mal wiederholt.

Zusätzlich soll die chromatische Abberation der Linse untersucht werden,
dazu werden je fünf Wertepaare für rot und blau gefiltertes Filter
aufgenommen.
\subsection{Bestimmung der Brennweite mit der Methode von Abbe}
Die Methode von Abbe wird verwendet, um die Brennweite und die Lage der Hauptebenen
eines Linsensystems zu bestimmen. Hierzu wird der Abbildungsmaßstab $V$,
die Gegenstands- und Bildweite verwendet. Da die Hauptebenen nicht bekannt sind,
werden die Weiten $g'$ und $b'$ zu einem gewählten Punkt $A$ gemessen.
Es gilt der Zusammenhang
\begin{align}
    g'=g+h=f\left(1+\frac{1}{V}\right)+h \label{eq: abstaende_abbe_g} \\
    b'=b+h'=f\left(1+V\right)+h' \label{eq: abstaende_abbe_b}.
\end{align}
Als Linsensystem wird eine Zerstreungs- und Sammellinse verwendet (vgl. Abbildung \ref{fig: linsensystem}).
\begin{figure}
    \centering
    \includegraphics[width=0.6\textwidth]{./pics/linsensystem.png}
    \caption{Linsensystem \cite{anleitung408}.}
    \label{fig: linsensystem}
\end{figure}
Die relativen Bild- und Gegenstandsweiten werden von der Mittelebene
der Sammellinse gemessen.
Es wird der Schirm, bei fester Linsenposition, solang verschoben, bis ein scharfes
Bild auf dem Schirm erkennbar ist. Nun werden $g'$, $b'$ und $B$ notiert.
Mit Hilfe von \eqref{eq: abbildungsgesetz_gross} bzw. \eqref{eq: abbildungsgesetz_klein}
kann $V$ bestimmt werden.
Für zehn verschiedene Linsenpositionen, werden die Werte aufgeschrieben.
