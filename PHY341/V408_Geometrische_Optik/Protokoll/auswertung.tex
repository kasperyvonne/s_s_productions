\section{Auswertung}
...
Für die gemessenen Strecken wurde jeweils ein Fehler von $\SI{1}{\milli\meter}$ angenommen.

\subsection{Überprüfung der Linsen-/Abbildungsgleichung}
Für die Messungen zur Verifizierung der Gleichungen \eqref{} und \eqref{} wird eine Linse
der bekannten Brennweite
\begin{equation}
  f\ua{t} = \SI{10}{\centi\meter}
\end{equation}
verwendet. In Tabelle \ref{tab: methode_1} sind die gemessenen Wertepaare der Gegenstands- bzw. Bildweite $(g\ua{i}, b\ua{i})$,
sowie die zugehörigen Bildgrößen $B$ aufgeführt. Für die Berechnung der Vergrößerung $V\ua{1}$
wird die konstante Gegenstandsgröße $G = \SI{3.0(1)}{\centi\meter}$ verwendet. Neben den Ergebnissen ist jeweils
eine mittlere Prozentuale Abweichung zwischen $V_1$ und $V_2$ in Tabelle \ref{tab: methode_1} eingefügt.
Mittels Gleichung \eqref{} ergibt sich für den Mittelwert der Brennweite
\begin{equation}
  f\ua{1, mid} = \SI{+9.68(1)}{\centi\meter}.
\end{equation}
In Abbildung \ref{fig: methode_1} sind die Verbindungslinien der Punkte $(g\ua{i}, 0)-(0, b\ua{i})$ einzusehen. Als Schnittpunkt
der Geraden, der der Brennweite $f\ua{1, exp}$ entspricht wurde folgender Wert abgelesen
\begin{equation}
  f\ua{1, exp} = \SI{+9.7(1)}{\centi\meter}.
\end{equation}
\begin{figure}
  \centering
  \includegraphics[width=0.7\textwidth]{../Messdaten/plots/methode_1.pdf}
  \caption{Darstellung der Verbindungsgeraden $(g\ua{i}, 0) - (0, b\ua{i})$ zur Bestimmung der Brennweite einer
  Linse bekannter Brennweite $f = \SI{100}{\centi\meter}$.}
  \label{fig: methode_1}
\end{figure}
\begin{table} 
\centering 
\caption{test} 
\label{tab: tab: methode_1} 
\begin{tabular}{S S S S S } 
\toprule  
{$g/\si{\centi\meter}$} & {$b/\si{\centi\meter}$} & {$B/\si{\centi\meter}$} & {$V_1$} & {$V_1$}  \\ 
\midrule  
 36.0  & 10.3  & 0.0  & 0.3  & 0.0+/-0\\ 
34.0  & 10.3  & 2.6  & 0.3  & 0.0+/-0\\ 
32.0  & 11.0  & 2.3  & 0.3  & 0.0+/-0\\ 
30.0  & 12.0  & 2.0  & 0.4  & 0.0+/-0\\ 
28.0  & 13.3  & 1.8  & 0.5  & 0.5+/-0.0\\ 
26.0  & 14.7  & 1.6  & 0.6  & 0.6+/-0.0\\ 
24.0  & 16.2  & 0.0  & 0.7  & 0.7+/-0.0\\ 
22.0  & 17.8  & 0.0  & 0.8  & 0.8+/-0.0\\ 
20.0  & 19.5  & 0.0  & 1.0  & 0.9+/-0.0\\ 
18.0  & 21.3  & 0.0  & 1.2  & 0.0+/-0\\ 
\bottomrule 
\end{tabular} 
\end{table}

\subsection{Bestimmung der Brennweite einer unbekannten Linse}
Nach der selben Methode wie im vorangegangen Abschnitt wird die Brennweite einer unbekannten Linse bestimmt.
Die aufgenommen Daten sind in Tabelle \ref{} eingefügt. Die Graphische Darstellung der Verbindungslinien ist
in Abbildung \ref{fig: wasserlinse} einzusehen. Mit der Linsenformel ergibt sich durch Mittelung der Wert
\begin{equation}
  f\ua{u, mid} = \SI{+13.83(1)}{\centi\meter}
\end{equation}
Aus dem Graphen \ref{fig: wasserlinse} wird folgender Schnittpunkt abgelesen
\begin{equation}
  f\ua{u, exp} = \SI{14.1(1)}{\centi\meter}.
\end{equation}
\begin{table}[H] 
\centering 
\caption{Messdaten zur Bestimmung der Brennweite einer unbekannten Linse, Gegenstandsweite $g$, Bildweite $b$ und berechnte Brennweite $f$.} 
\label{tab: wasserlinse} 
\begin{tabular}{S S S } 
\toprule  
{$g / \si{\centi\meter}$} & {$b / \si{\centi\meter}$} & {$f / \si{\centi\meter}$}  \\ 
\midrule  
 20.0  & 47.3  & 14.1\\ 
21.0  & 41.3  & 13.9\\ 
22.0  & 36.8  & 13.8\\ 
23.0  & 34.5  & 13.8\\ 
24.0  & 31.4  & 13.6\\ 
25.0  & 32.3  & 14.1\\ 
30.0  & 25.5  & 13.8\\ 
35.0  & 22.3  & 13.6\\ 
\bottomrule 
\end{tabular} 
\end{table}
\begin{figure}
  \centering
  \includegraphics[width = 0.7\textwidth]{../Messdaten/plots/wasserlinse.pdf}
  \caption{Darstellung der Verbindungsgeraden $(g\ua{i}, 0) - (0, b\ua{i})$ zur Bestimmung der Brennweite einer
  Linse unbekannter Brennweite.}
  \label{fig: wasserlinse}
\end{figure}

\subsection{Bestimmung der Brennweite nach der Methode von Bessel}
Die gemessenen Werte für die beiden Punkte $(g_1, b_1)$ und $(g_2, b_2)$, bei denen unter konstantem Abstand
zwischen Schirm und Gegenstand jeweils Bild und Gegenstandsweite
vertauschen, sind in Tabelle \ref{tab: bessel} einzusehen. Mit Gleichung \eqref{} ergibt sich der Mittelwert für
die Brennweite $f\ua{2, mid}$ zu
\begin{equation}
  f\ua{2, mid} = \SI{+9.90(1)}{\centi\meter}.
\end{equation}
\begin{table} 
\centering 
\caption{Messdaten zur Bestimmung der Brennweite mit der Methode nach Bessel.} 
\label{tab: bessel} 
\begin{tabular}{S[table-format=2.1]@{${}\pm{}$} S[table-format=1.1]
S[table-format=2.1]@{${}\pm{}$} S[table-format=1.1]
S[table-format=2.1]@{${}\pm{}$} S[table-format=1.1]
S[table-format=2.1]@{${}\pm{}$} S[table-format=1.1]} 
\toprule  
\multicolumn{2}{c}{$g_1 \:/\: \si{\centi\meter}$} & \multicolumn{2}{c}{$b_1 \:/\: \si{\centi\meter}$} &\multicolumn{2}{c}{$g_2 \:/\: \si{\centi\meter}$} & \multicolumn{2}{c}{$b_2 \:/\: \si{\centi\meter}$}  \\ 
\midrule  
 22.8  & 0.1  & 17.2  & 0.1  & 16.7  & 0.1  & 23.3  & 0.1\\ 
27.8  & 0.1  & 15.2  & 0.1  & 14.9  & 0.1  & 28.1  & 0.1\\ 
34.3  & 0.1  & 13.7  & 0.1  & 13.6  & 0.1  & 34.4  & 0.1\\ 
39.7  & 0.1  & 13.3  & 0.1  & 13.1  & 0.1  & 39.9  & 0.1\\ 
45.2  & 0.1  & 12.8  & 0.1  & 12.2  & 0.1  & 45.8  & 0.1\\ 
50.5  & 0.1  & 12.5  & 0.1  & 12.4  & 0.1  & 50.6  & 0.1\\ 
55.7  & 0.1  & 12.3  & 0.1  & 12.0  & 0.1  & 56.0  & 0.1\\ 
60.8  & 0.1  & 12.2  & 0.1  & 11.8  & 0.1  & 61.2  & 0.1\\ 
66.2  & 0.1  & 11.8  & 0.1  & 11.7  & 0.1  & 66.3  & 0.1\\ 
71.2  & 0.1  & 11.8  & 0.1  & 11.5  & 0.1  & 71.5  & 0.1\\ 
\bottomrule 
\end{tabular} 
\end{table}


\begin{table} 
\centering 
\caption{Messdaten zur Untersuchung der chromatischen Abberation.} 
\label{tab: colors} 
\begin{tabular}{S S[table-format=2.1]@{${}\pm{}$} S[table-format=1.1]
S[table-format=2.1]@{${}\pm{}$} S[table-format=1.1]
S[table-format=2.1]@{${}\pm{}$} S[table-format=1.1]
S[table-format=2.1]@{${}\pm{}$} S[table-format=1.1]}  
\toprule  
{Farbe} & \multicolumn{2}{c}{$g_1 \:/\: \si{\centi\meter}$} & \multicolumn{2}{c}{$b_1 \:/\: \si{\centi\meter}$} &\multicolumn{2}{c}{$g_2 \:/\: \si{\centi\meter}$} & \multicolumn{2}{c}{$b_2 \:/\: \si{\centi\meter}$}  \\ 
\midrule  
 blau  & 22.9  & 0.1  & 17.1  & 0.1  & 17.1  & 0.1  & 22.9  & 0.1\\ 
  & 30.2  & 0.1  & 14.8  & 0.1  & 14.9  & 0.1  & 30.1  & 0.1\\ 
  & 36.4  & 0.1  & 13.6  & 0.1  & 13.4  & 0.1  & 36.6  & 0.1\\ 
  & 42.0  & 0.1  & 13.0  & 0.1  & 12.9  & 0.1  & 42.1  & 0.1\\ 
  & 47.2  & 0.1  & 12.8  & 0.1  & 12.3  & 0.1  & 47.7  & 0.1\\ 
\hline
rot  & 21.9  & 0.1  & 18.1  & 0.1  & 18.6  & 0.1  & 21.4  & 0.1\\ 
  & 30.3  & 0.1  & 14.7  & 0.1  & 14.7  & 0.1  & 30.3  & 0.1\\ 
  & 36.0  & 0.1  & 14.0  & 0.1  & 13.8  & 0.1  & 36.2  & 0.1\\ 
  & 41.8  & 0.1  & 13.2  & 0.1  & 13.2  & 0.1  & 41.8  & 0.1\\ 
  & 47.3  & 0.1  & 12.7  & 0.1  & 12.6  & 0.1  & 47.4  & 0.1\\ 
\bottomrule 
\end{tabular} 
\end{table}

\begin{table} 
\centering 
\caption{Messdaten zur Bestimmung der Brennweite und Lage der Hauptebenen einer Linsenanordnung mit der Methode nach Abbe.} 
\label{tab: abbe} 
\begin{tabular}{S[table-format=2.1]@{${}\pm{}$} S[table-format=1.1]
S[table-format=2.1]@{${}\pm{}$} S[table-format=1.1]
S[table-format=1.1]@{${}\pm{}$} S[table-format=1.1]
S[table-format=1.2]@{${}\pm{}$} S[table-format=1.2]
S[table-format=1.2]@{${}\pm{}$} S[table-format=1.2]} 
\toprule  
 \multicolumn{2}{c}{$g' \:/\: \si{\centi\meter}$} & \multicolumn{2}{c}{$b' \:/\: \si{\centi\meter}$} &\multicolumn{2}{c}{$B \:/\: \si{\centi\meter}$} & \multicolumn{2}{c}{$1 + \frac{1}{V}$} & \multicolumn{2}{c}{$1 + V$}  \\ 
\midrule  
 15.0  & 0.1  & 54.2  & 0.1  & 4.6  & 0.1  & 1.65  & 0.03  & 2.53  & 0.06\\ 
17.0  & 0.1  & 50.0  & 0.1  & 3.8  & 0.1  & 1.79  & 0.03  & 2.27  & 0.05\\ 
19.0  & 0.1  & 46.5  & 0.1  & 3.3  & 0.1  & 1.91  & 0.04  & 2.10  & 0.05\\ 
21.0  & 0.1  & 44.5  & 0.1  & 2.9  & 0.1  & 2.03  & 0.05  & 1.97  & 0.05\\ 
23.0  & 0.1  & 42.7  & 0.1  & 2.6  & 0.1  & 2.15  & 0.06  & 1.87  & 0.04\\ 
25.0  & 0.1  & 41.6  & 0.1  & 2.4  & 0.1  & 2.25  & 0.07  & 1.80  & 0.04\\ 
27.0  & 0.1  & 40.9  & 0.1  & 2.1  & 0.1  & 2.43  & 0.08  & 1.70  & 0.04\\ 
29.0  & 0.1  & 40.5  & 0.1  & 2.0  & 0.1  & 2.50  & 0.09  & 1.67  & 0.04\\ 
31.0  & 0.1  & 39.7  & 0.1  & 1.8  & 0.1  & 2.67  & 0.11  & 1.60  & 0.04\\ 
33.0  & 0.1  & 39.5  & 0.1  & 1.7  & 0.1  & 2.76  & 0.12  & 1.57  & 0.04\\ 
\bottomrule 
\end{tabular} 
\end{table}

