\section{Auswertung}
...
Für die gemessenen Strecken wurde jeweils ein Fehler von $\SI{1}{\mili\meter}$ angenommen.

\subsection{Überprüfung der Linsen-/Abbildungsgleichung}
Für die Messungen zur Verifizierung der Gleichungen \eqref{} und \eqref{} wird eine Linse
der bekannten Brennweite
\begin{equation}
  f\ua{t} = \SI{10}{\centi\meter}
\end{equation}
verwendet. In Tabelle \ref{} sind die gemessenen Wertepaare der Gegenstands- bzw. Bildweite $(g\ua{i}, b\ua{i})$,
sowie die jeweiligen Bildgrößen $B$ aufgeführt. Mit der konstanten Gegenstandsgröße $G = \SI{3}{\centi\meter}$
werden die Vergrößerungen $V_1$ und $V_2$ gemäß
\begin{equation}
  V_1 = \frac{b}{g} \quad \text{bzw.} \quad V_1 = \frac{B}{G}
\end{equation}
berechnet. Neben den Ergebnissen ist jeweils eine mittlere Prozentuale Abweichung zwischen $V_1$ und $V_2$ in Tabelle \ref{}
eingefügt.
