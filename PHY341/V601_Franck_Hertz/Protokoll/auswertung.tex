\section{Auswertung}
Die in der Auswertung bestimmten Ausgleichsrechnungen werden mit
dem Python Paket \emph{scipy.optimize}\cite{scipy} durchgeführt.
Des Weiteren werden die Fehler und insbesondere die Fehlerfortpflanzungen
mit dem Python Paket \emph{uncertainties}\cite{uncertainties} berechnet.

\subsection{Regressionsberechnung zur Bestimmung der Spannung im Verhältnis zum gemessenen Abstand}\label{sec: ausgleich}

Für die verschiedenen Auswertungsteile wird immer eine Umrechnung von gemessenen Abständen
in die Spannung $U$ benötigt. Dazu wird an die Messwerte die in den Tabellen \ref{tab: spannung_abstand_zim} bis \ref{tab: kodi_ioni} dargestellt sind,
eine Gerade der Form
\begin{equation}
  \label{eq: gerade}
  g(x)=mx+b
\end{equation}
berechnet.
Die sich daraus ergebenen Regressionsparameter sind in Tabelle \ref{tab: umrech} aufgelistet.
\FloatBarrier
\begin{table} 
\centering 
\caption{Aus Abbildung \ref{} abgelesene Spannung-Abstandspaare.} 
\label{tab: spannung_abstand_zim} 
\begin{tabular}{S S } 
\toprule  
{Abstand in $\si{\centi\meter}$} & {Spannung in $\si{\volt}$}  \\ 
\midrule  
 2.0  & 1.0\\ 
4.0  & 2.0\\ 
6.1  & 3.0\\ 
8.2  & 4.0\\ 
10.3  & 5.0\\ 
12.5  & 6.0\\ 
14.6  & 7.0\\ 
16.7  & 8.0\\ 
18.8  & 9.0\\ 
21.3  & 10.0\\ 
\bottomrule 
\end{tabular} 
\end{table}
\begin{table} 
\centering 
\caption{Aus Abbildung \ref{} abgelesene Spannung-Abstandspaare.} 
\label{tab: spannung_abstand_hot} 
\begin{tabular}{S S } 
\toprule  
{Abstand in $\si{\centi\meter}$} & {Spannung in $\si{\volt}$}  \\ 
\midrule  
 1.9  & 1.0\\ 
4.1  & 2.0\\ 
6.1  & 3.0\\ 
8.3  & 4.0\\ 
10.2  & 5.0\\ 
12.3  & 6.0\\ 
14.5  & 7.0\\ 
16.7  & 8.0\\ 
18.8  & 9.0\\ 
20.9  & 10.0\\ 
\bottomrule 
\end{tabular} 
\end{table}
\begin{table} 
\centering 
\caption{Aus Abbildung \ref{} abgelesene Spannung-Abstandspare.} 
\label{tab: spannung_abstand_frank} 
\begin{tabular}{S S } 
\toprule  
{Abstand in $\si{\centi\meter}$} & {Spannung in $\si{\volt}$}  \\ 
\midrule  
 2.1  & 5\\ 
4.0  & 10\\ 
6.0  & 15\\ 
7.9  & 20\\ 
9.8  & 25\\ 
11.8  & 30\\ 
13.8  & 35\\ 
15.8  & 40\\ 
19.7  & 50\\ 
21.7  & 55\\ 
\bottomrule 
\end{tabular} 
\end{table}
\begin{table} 
\centering 
\caption{Aus Abbildung \ref{} abgelesene Spannung-Abstandspaare.} 
\label{tab: spannung_abstand_ioni} 
\begin{tabular}{S S } 
\toprule  
{Abstand in $\si{\centi\meter}$} & {Spannung in $\si{\volt}$}  \\ 
\midrule  
 1.3  & 2.0\\ 
2.7  & 4.0\\ 
4.2  & 6.0\\ 
5.4  & 8.0\\ 
6.8  & 10.0\\ 
8.0  & 12.0\\ 
9.4  & 14.0\\ 
10.5  & 16.0\\ 
12.0  & 18.0\\ 
13.5  & 20.0\\ 
14.5  & 22.0\\ 
15.8  & 24.0\\ 
17.2  & 26.0\\ 
18.5  & 28.0\\ 
\bottomrule 
\end{tabular} 
\end{table}
\begin{table} 
\centering 
\caption{Regressiongerade für die Abstand in Spannungs Umrechnung. Im Versuchsteil $1$ wird die Energieverteilung bei $T=\SI{28}{\celsius}$ (vgl. Kapitel \ref{sec: eng_zim}) und in $2$ bei $T=\SI{155}{\celsius}$ (vgl. Kap. \ref{sec:en_hot}) untersucht, der Abschnitt $3$ beschäftigt sich mit der Analyse der Frank-Hertz-Kurve (vgl. Kap. \ref{sec: frank}) und im Abschnitt $4$ wird die Ionisierungsspannung bestimmt (vgl. Kap. \ref{sec:ioni}).} 
\label{tab: umrech} 
\begin{tabular}{S S S S S } 
\toprule  
{Versuchsteil} & { $m$ in $\si{\volt\centi\meter\per}$} & {$\sigma_\mathrm{m}$ in $\si{\volt\centi\per\meter}$} & {$b$ in $\si{\volt}$} & {$\sigma_\mathrm{b}$ in $\si{\volt}$}  \\ 
\midrule  
 1  & 0.469  & 0.003  & 0.13  & 0.04\\ 
2  & 0.475  & 0.002  & 0.13  & 0.03\\ 
3  & 2.549  & 0.006  & -0.20  & 0.08\\ 
4  & 1.522  & 0.009  & -0.19  & 0.10\\ 
\bottomrule 
\end{tabular} 
\end{table}
Die Messwerte aus den Tabellen \ref{tab: spannung_abstand_zim} bis \ref{tab: kodi_ioni}
und die dazugehörigen Ausgleichsgeraden sind in
den Abbildungen \ref{fig: darstellung_1} und \ref{fig: darstellung_2} illustriert.
\begin{figure}
  \centering
  \begin{subfigure}{0.48\textwidth}
    \centering
    \includegraphics[width=1 \textwidth]{../Messdaten/zim.pdf}
    \caption{Graphische Darstellung der Ausgleichgeraden für die Energieverteilung bei $\SI{28}{\degree}$.} %der, geraden
    \label{fig: energie_zim}
  \end{subfigure}
  \begin{subfigure}{0.48\textwidth}
    \centering
    \includegraphics[width=1 \textwidth]{../Messdaten/spannungsfit_energieverteilung_150grad.pdf}
    \caption{Graphische Darstellung der Ausgleichgeraden für die Energieverteilung bei $\SI{155}{\degree}$.} %s.o.
    \label{fig: enrgie_hot}
  \end{subfigure}
  \caption{Darstellung der Ausgleichsgeraden für $(1)$ und $(2)$}
  \label{fig: darstellung_1}
\end{figure}
\begin{figure}
  \centering
  \begin{subfigure}{0.48\textwidth}
    \centering
    \includegraphics[width=1 \textwidth]{../Messdaten/frank_hertz_kuvre.pdf}
    \caption{Graphische Darstellung des Ausgleichgraden für Franck-Hertz-Kurve.}
    \label{fig: frank_hertz}
  \end{subfigure}
  \begin{subfigure}{0.48\textwidth}
    \centering
    \includegraphics[width=1 \textwidth]{../Messdaten/ioni.pdf}
    \caption{Graphische Darstellung der Ausgleichgeraden für die Untrersuchung der Ionisierungsspannung.} %s.o.
    \label{fig: enrgie_hot}
  \end{subfigure}
  \caption{Darstellung der Ausgleichsgeraden für $(3)$ und $(4)$}
  \label{fig: darstellung_2}
\end{figure}

Die Fehler der Steigung und des y-Achsenabschnittes (vgl. Tab. \ref{tab: umrech}) werden
im Folgenden, auf Grund der geringen Größe im Vergleich zum eigentlichen Ablesefehler, nicht weiter betrachtet.
Die Umrechnung von Abstand in Spannung erfolgt somit fehlerfrei.
\FloatBarrier %Verwende lieber nach Figures [H]

\subsection{Untersuchung der mittleren freien Weglänge}
\FloatBarrier
In diesem Kapitel soll die in der der Theorie erwähnte Untersuchung des Verhältnisses
zwischen dier mittleren freie Weglänge $\ov{w}$ und Größe $a$
(Abstand zwischen Kathode und Beschleunigungselektode) erfolgen.
Für die Untersuchung der Franck-Hertz-Kurve (Abschnitt \ref{sec: frank}) %Erster Satz ohne Inhalt
sollte $\ov{w}$ etwa um den Faktor $1000-4000$ kleiner sein. Die freie Weglänge hängt über den  Druck mit der Temperatur zusammen. %den
Die Energieverteilung wird einmalbei einer Temperatur von $T=\SI{28}{\celsius}$ (vgl.\ref{sec: eng_zim}) und bei $T=\SI{155}{\celsius}$ \ref{sec:en_hot} untersucht.
Die Franck-Hertz-Kurve wird in Kapitel \ref{sec: frank} bei einer Temperatur von $T=\SI{188}{\celsius}$ analysiert.  %Präsenz
Der Teil $4$ (vgl. Abschnitt \ref{sec:ioni}) wurde bei $T=\SI{104}{\celsius}$ gemessen (selbe Nummerierung wie in Tab. \ref{tab: umrech}). %Satzbau
Die Resultate der Berechnungen werden mit den Gleichungen \eqref{eq: dampfdruck} und \eqref{eq: weglaenge} erbracht und %Präsenz,
sind in der Tabelle \ref{tab: weg} aufgeführt.
Für die Franck-Hertz-Kurve ist die Forderung somit erfüllt. %Forderung
\begin{table} 
\centering 
\caption{Ergebnisse für die Verhältnisberechenung $a/w$.} 
\label{tab: weg} 
\begin{tabular}{S S S S S } 
\toprule  
{$T$ in $\si{\celsius}$} & {$T$ in $\si{\kelvin}$} & {$p_{\mathrm{sät}}$ in $\si{\milli\bar}$} & {$\overline{w}$ in $\si{\centi\meter}$} & {$\frac{a}{w}$}  \\ 
\midrule  
 28  & 301.15  & 0.007  & 0.4346  & 2\\ 
104  & 377.15  & 0.665  & 0.0044  & 229\\ 
150  & 423.15  & 4.822  & 0.0006  & 1663\\ 
188  & 461.15  & 18.399  & 0.0002  & 6344\\ 
\bottomrule 
\end{tabular} 
\end{table}


\subsection{Analyse der Energierverteilung bei $T=\SI{28}{\celsius}$}\label{sec: eng_zim}

Die aufgenommene Messkurve ist in Abbildung \ref{fig: messkurve_energie_zim} dargestellt.
\begin{figure}
  \centering
  \includegraphics[width=0.8 \textwidth]{./pics/energieverteilung_zimmer.png}
  \caption{Aufgenommene Energieverteilung bei $T=\SI{28}{\celsius}$. Auf der $x$-Achse ist die Spannung $U\ua{A}$ in $\si{\volt}$ aufgetragen.
          Die $y$-Achse gibt die Stromstärke $I\ua{A}$ an. Zusätzlich sind in der Abbildung die für die Auswertung benötigten Steigungsdreiecke zu erkennen.} %Hingegen passt nicht
  \label{fig: messkurve_energie_zim}
\end{figure}
Die Energieverteilung wird dort in der \emph{integralen} Form dargestellt.
Durch das Anlegen von Steigungsdreiecken der Form %Anlegen
\begin{equation}
  \label{eq:steigung}
    M=\frac{\Delta y}{\Delta x}=\frac{\Delta y}{x_2-x_1},
\end{equation}
kann die Kurve in ihre \emph{differntielle} Form überführt werden. %überführt

Damit eine graphische Darstellung der differntielle Kurve möglich ist, wird für jedes Koordinaatenpaar $(\Delta x,\Delta y)$ der Mittelpunkt %differentiellen, Koordinatenpaar
von $\Delta x$ als Messpunkt festgelegt. An diesem Messpunkt soll die Steigung $\frac{\Delta y}{\Delta x}$ genährt vorliegen.
In Tabelle \ref{tab: steigungen_zim} sind die abgelsenen Werte für die Steigungsberechnung aufgelistet.
\begin{table} 
\centering 
\caption{Aus Abbildung \ref{fig: messkurve_energie_zim} abgelesene Steigungen.} 
\label{tab: steigungen_zim} 
\begin{tabular}{S S S S S S } 
\toprule  
{$x_1$ in $\si{\centi\meter}$} & {$x_2$ in $\si{\centi\meter}$} & { ${\Delta y}$ in $\si{\milli\meter}$} & {$\frac{\Delta y}{\Delta x}$ in \si{\milli\meter\per\centi\meter}} & {Messpunkt in $\si{\centi\meter}$} & {Messpunkt in $\si{\volt}$}  \\ 
\midrule  
 0.0  & 3.3  & 1.0  & 0.30  & 1.65  & 0.90\\ 
3.5  & 5.5  & 1.0  & 0.50  & 4.50  & 2.24\\ 
5.5  & 6.9  & 1.0  & 0.71  & 6.20  & 3.04\\ 
6.9  & 8.2  & 1.0  & 0.77  & 7.55  & 3.67\\ 
8.7  & 9.7  & 1.0  & 1.00  & 9.20  & 4.44\\ 
9.9  & 11.4  & 2.0  & 1.33  & 10.65  & 5.12\\ 
11.9  & 12.8  & 2.0  & 2.22  & 12.35  & 5.92\\ 
13.0  & 13.7  & 1.8  & 2.57  & 13.35  & 6.39\\ 
13.7  & 14.5  & 2.5  & 3.12  & 14.10  & 6.74\\ 
14.5  & 15.0  & 2.0  & 4.00  & 14.75  & 7.05\\ 
15.0  & 15.6  & 2.0  & 3.33  & 15.30  & 7.31\\ 
15.6  & 16.0  & 2.0  & 5.00  & 15.80  & 7.54\\ 
16.0  & 16.5  & 2.0  & 4.00  & 16.25  & 7.75\\ 
16.5  & 17.0  & 3.0  & 6.00  & 16.75  & 7.99\\ 
17.0  & 17.5  & 3.0  & 6.00  & 17.25  & 8.22\\ 
17.5  & 17.9  & 4.0  & 10.00  & 17.70  & 8.43\\ 
17.9  & 18.2  & 3.0  & 10.00  & 18.05  & 8.60\\ 
18.2  & 18.5  & 3.0  & 10.00  & 18.35  & 8.74\\ 
18.5  & 18.7  & 2.5  & 12.50  & 18.60  & 8.85\\ 
18.7  & 19.1  & 5.0  & 12.50  & 18.90  & 8.99\\ 
19.1  & 19.4  & 5.0  & 16.67  & 19.25  & 9.16\\ 
19.4  & 19.7  & 6.0  & 20.00  & 19.55  & 9.30\\ 
19.7  & 19.9  & 8.0  & 40.00  & 19.80  & 9.42\\ 
19.9  & 20.0  & 8.0  & 80.00  & 19.95  & 9.49\\ 
20.0  & 20.2  & 15.0  & 75.00  & 20.10  & 9.56\\ 
20.2  & 20.3  & 13.0  & 130.00  & 20.25  & 9.63\\ 
20.3  & 20.4  & 16.0  & 160.00  & 20.35  & 9.67\\ 
20.4  & 20.5  & 10.0  & 100.00  & 20.45  & 9.72\\ 
20.6  & 20.9  & 10.0  & 33.33  & 20.75  & 9.86\\ 
20.9  & 22.0  & 5.0  & 4.55  & 21.45  & 10.19\\ 
\bottomrule 
\end{tabular} 
\end{table}
Die Darstellung der differntiellen Kurve ist in Abb. \ref{fig: energie_zim_diff} zu finden.
\begin{figure}
  \centering
  \includegraphics[width=0.8 \textwidth]{../Messdaten/energie_zim.pdf}
  \caption{Die Energieverteilung bei $T=\SI{28}{\celsius}$ in differentieller Form.} %Energieverteilung, schreib differenziell oder differentiell einheitlich
  \label{fig: energie_zim_diff}
\end{figure}
Das Maximum der Abbildung liegt bei $U\ua{max}=\SI{9.67}{\volt}$.
Eine Berechnung des Kontaktpotentials $K$ ist wie folgt möglich:
\begin{equation}
\label{eq:k_energie_zim}
  K\ua{en}=U\ua{B}-U\ua{max}=\SI{1.33}{\volt} \qquad U\ua{B}=\SI{11}{\volt}. %.
\end{equation}
\FloatBarrier %s.o.
\subsection{Analyse der Energierverteilung bei $T=\SI{150}{\celsius}$}\label{sec:en_hot}
\FloatBarrier
Die aus dem Experiment resultierende Messkurve ist in Abbildung \ref{fig: messkurve_energie_hot} dargestellt. \\
\begin{figure}
  \centering
  \includegraphics[width=0.8 \textwidth]{./pics/energieverteilung_hot.png}
  \caption{Aufgenommene Energieverteilung bei $T=\SI{150}{\celsius}$. Auf der $x$-Achse ist die Spannung $U\ua{a}$ in $\si{\volt}$ aufgetragen.
          Die $y$-Achse gibt die Stromstärke $I\ua{A}$ an. Zusätzlich sind in der Abbildung die für die Auswertung benötigten Steigungsdreiecke zu erkennen.} %benötigten
  \label{fig: messkurve_energie_hot}
\end{figure}
Um die Kurve in die differntiele Form zu transformieren, wird wie im vorherigen Kaptiel verfahren. %differentielle, Kapitel
Die Steigungen werden mit Formel \eqref{eq:steigung} berechnet und sind in Tabelle \ref{tab: steigungen_hot} notiert. %Plural
\begin{table} 
\centering 
\caption{Aus Abbildung \ref{} abgelesene Steigungen.} 
\label{tab: steigungen_hot} 
\begin{tabular}{S S S S S S } 
\toprule  
{$x_1$ in $\si{\centi\meter}$} & {$x_2$ in $\si{\centi\meter}$} & { ${\Delta y}$ in $\si{\milli\meter}$} & {$\frac{\Delta y}{\Delta x}$ in \si{\centi\meter\per\milli\meter}} & {Messpunkt in $\si{\centi\meter}$} & {Messpunkt in $\si{\volt}$}  \\ 
\midrule  
 2.4  & 2.9  & 3.0  & 6.00  & 2.65  & 1.36\\ 
2.9  & 3.3  & 3.0  & 7.50  & 3.10  & 1.57\\ 
3.3  & 3.5  & 2.0  & 10.00  & 3.40  & 1.71\\ 
3.5  & 3.8  & 2.0  & 6.67  & 3.65  & 1.83\\ 
3.8  & 4.5  & 6.0  & 8.57  & 4.15  & 2.07\\ 
4.5  & 4.9  & 4.0  & 10.00  & 4.70  & 2.33\\ 
4.9  & 5.3  & 4.0  & 10.00  & 5.10  & 2.52\\ 
5.3  & 5.8  & 5.0  & 10.00  & 5.55  & 2.73\\ 
5.8  & 6.5  & 9.0  & 12.86  & 6.15  & 3.02\\ 
6.5  & 6.9  & 4.0  & 10.00  & 6.70  & 3.28\\ 
6.9  & 7.3  & 5.0  & 12.50  & 7.10  & 3.47\\ 
7.3  & 7.8  & 5.0  & 10.00  & 7.55  & 3.68\\ 
7.8  & 8.2  & 5.0  & 12.50  & 8.00  & 3.90\\ 
8.2  & 8.6  & 4.0  & 10.00  & 8.40  & 4.09\\ 
8.6  & 9.0  & 4.0  & 10.00  & 8.80  & 4.28\\ 
9.0  & 9.2  & 2.0  & 10.00  & 9.10  & 4.42\\ 
9.2  & 9.5  & 3.0  & 10.00  & 9.35  & 4.54\\ 
9.5  & 9.7  & 2.0  & 10.00  & 9.60  & 4.66\\ 
9.7  & 9.9  & 2.0  & 10.00  & 9.80  & 4.75\\ 
9.9  & 10.2  & 2.0  & 6.67  & 10.05  & 4.87\\ 
10.2  & 10.5  & 2.0  & 6.67  & 10.35  & 5.01\\ 
10.5  & 10.9  & 2.0  & 5.00  & 10.70  & 5.18\\ 
10.9  & 11.2  & 2.0  & 6.67  & 11.05  & 5.34\\ 
11.2  & 12.2  & 2.0  & 2.00  & 11.70  & 5.65\\ 
12.2  & 16.2  & 2.0  & 0.50  & 14.20  & 6.84\\ 
\bottomrule 
\end{tabular} 
\end{table}
Für die Umrechnung von Abstand in Spannung werden die Resultate aus Kapitel \ref{sec: ausgleich} verwendet.
Eine Illustration der differentiellen Kurve ist in Abbildung \ref{fig: energie_hot_diff} präsentiert. %s.o.
\begin{figure}
  \centering
  \includegraphics[width=0.9 \textwidth]{../Messdaten/energie_hot.pdf}
  \caption{Die Energiervertielung bei $T=\SI{150}{\celsius}$ in differentieller Form.}
  \label{fig: energie_hot_diff}
\end{figure}

Wie in Abbildung \ref{fig: messkurve_energie_hot} zu erkennen, beginnen die Steigungsdreiecke erst ab
einer Spannung von $U\ua{anf}=\SI{1.24}{\volt}$, dies ist damit zu begründen, dass für Spannungen $U<U\ua{anf}$ %begründen, dass
der Kurvenverlauf keinen physikalischen Zusammenhang beschreibt. Eine mögliche Ursache wird
in der Diskussion besprochen.

Abschließend ist noch der signifikante Unterschied zwischen den Graphen \ref{fig: messkurve_energie_hot} und \ref{fig: messkurve_energie_zim}
zu erläutern. Beim Betrachten der Tabelle \ref{tab: weg} fällt auf, dass sich die Werte für $\frac{a}{w}$ um den Faktor $\approx 800$ unterscheiden. %auf, dass
Das bedeutet, dass die ausgelösten Elektronen beim Messvorgang von \ref{fig: messkurve_energie_hot} deutlich öfters mit den %,dass
Quecksilberatomen gewechselwirkt (mittels Stoßprozessen) haben als in \ref{fig: messkurve_energie_zim}. Auf Grund der vermehrten Stoßvorgänge besitzen die Elektronen nicht mehr genügend %atomen
Energie, um das wachsende Gegenfeld $U\ua{A}$ zu durchqueren. Hiermit lässt sich das schnelle Konvergieren der Kurve erklären.
Eine Aussage über die Energieverteilung der Elektronen ist somit nicht möglich.
\FloatBarrier
\subsection{Untersuchung der Franck-Hertz-Kurve}\label{sec: frank} %Untersuchung
\FloatBarrier
Die vom $XY$ Schreiber aufgezeichnete Franck-Hertz-Kurve ist in der Abbildung \ref{fig: messkurve_frank_hertz} dargestellt. %$XY$% damit einheitlich zur Theorie, Illustration falsches Wort
\begin{figure}
  \centering
  \includegraphics[width=0.8 \textwidth]{./pics/frank_hertz_kurve.png}
  \caption{Aufgenommene Franck-Hertz-Kurve bei $T=\SI{188}{\celsius}$. Auf der $x$-Achse ist die Spannung $U\ua{B}$ in $\si{\volt}$ aufgetragen.
          Hingegen gibt die $y$-Achse die Stromstärke $I\ua{A}$ an. Zusätzlich sind die erkennbaren Maxima nummeriert.}
  \label{fig: messkurve_frank_hertz}
\end{figure}
Die aus der Abbildung abgelesenen Abstände der Maxima sind in der Tabelle \ref{tab: abstand_maxima} gelistet.
Die in der Tabelle angegebene Nummerierung bezeichnet mit $1$ den Abstand zwischen den Maxima $2$ und $3$, mit $2$ den Abstand zwischen Maxima $3$ und $4$ usw. (vgl. \ref{fig: messkurve_frank_hertz}). %Satzbau
\begin{table} 
\centering 
\caption{Aus Abbildung \ref{} abgelesene Abstände der Maxima.} 
\label{tab: abstand_maxima} 
\begin{tabular}{S S S } 
\toprule  
{Nummerierung} & {Abstand in $\si{\centi\meter}$} & {Abstand in $\si{\eV}$}  \\ 
\midrule  
 1  & 1.7  & 4.14\\ 
2  & 1.8  & 4.39\\ 
3  & 1.9  & 4.65\\ 
4  & 1.9  & 4.65\\ 
5  & 2.0  & 4.90\\ 
6  & 1.9  & 4.65\\ 
7  & 2.1  & 5.16\\ 
\bottomrule 
\end{tabular} 
\end{table}
Die Umrechnung von $\si{\centi\meter}$ zu $\si{\eV}$ erfolgt mit den Ergebnissen aus Kapitel \ref{sec: ausgleich}.
Der bestimmte Abstand entspricht dann der Anregungsenergie $E_1-E_0$, gemittelt folgt
\begin{equation}
  \label{eq:austrittarbeit}
E\ua{anr}=\ov{E_1-E_0}=\SI{4.65\pm0.12}{\eV}.
\end{equation}
Mit Gleichung \eqref{eq: wellenlänge} und dem Ergebnis \eqref{eq:austrittarbeit} kann die Wellenlänge des emitierten Licht untersucht werden: %emi
\begin{equation}
  \label{eq: lambda}
  \lambda=\SI{267\pm7}{\nano\meter}.
\end{equation}
Das entspricht ultraviolettem Licht.

Mit Hilfe der Franck-Hertz-Kurve ist auch eine Bestimmung des Kontaktpotentials
möglich. Hierzu wird das zweite Maximum ($n=2$) verwendet, da das vorherige Maximum %Maximum
nicht eindeutig erkennbar ist. Der Abstand beträgt
\begin{equation*}
  d\ua{cm}=\SI{3.1}{\centi\meter} \qquad d\ua{V}=\SI{7.7}{\volt}.
\end{equation*}
Mit Formel \eqref{eq: k_franck} ergibt sich dann für das Kontaktpotential:
\begin{equation}
  \label{eq:k_frank}
  K\ua{franck}= \SI{1.59\pm0.25}{\volt}.
\end{equation}
\FloatBarrier
\subsection{Bestimmung der Ionisierungsspannung}\label{sec:ioni}
\FloatBarrier
Die Messkurve ist in Abbildung \ref{fig: messkurve_ioni} dargestellt.
\begin{figure}
  \centering
  \includegraphics[width=0.8 \textwidth]{./pics/ionisierungsenergie.png}
  \caption{Aufgenommene Kurve für die Bestimmung der Ionisierungsspannung $U\ua{ion}$ bei $T=\SI{104}{\celsius}$. Auf der $x$-Achse ist die Spannung $U\ua{B}$ in $\si{\volt}$ aufgetragen.
          Hingegen gibt die $y$-Achse die Stromstärke $I\ua{A}$ an. Zusätzlich sind die vier Punkte markiert, die für die Bestimmung der Ionisierungsspannung verwedent werden.}
  \label{fig: messkurve_ioni}
\end{figure}
Die Bestimmung der Ionisierungsspannung $U\ua{ion}$ erfolgt mit Hilfe einer Ausgleichsrechnung. Dazu werden an dem Abschnitt der Kurve mit einer positiven Steigung,
Punkte ausgewählt. Für diese Punkte wird anschließend die Regressionsrechnung durchgeführt.
Der Nulldurchgang der Geraden sollte dann dem verschobenen Ionisierungsspannung entsprechen.
Gewählt wurden die in der Tabelle \ref{tab: spannung_abstand_ioni} notierten Punkte.
\begin{table} 
\centering 
\caption{Aus Abbildung \ref{fig: messkurve_ioni} abgelesene Spannung-Abstandspaare.} 
\label{tab: spannung_abstand_ioni} 
\begin{tabular}{S S } 
\toprule  
{$x$-Koordinate in $\si{\centi\meter}$} & {$y$-Koordinate in $\si{\centi\meter}$}  \\ 
\midrule  
 13.9  & 3.5\\ 
16.6  & 7.0\\ 
18.7  & 9.7\\ 
19.6  & 10.9\\ 
\bottomrule 
\end{tabular} 
\end{table}
An diese wurde eine Ausgleichgerade der Form \eqref{eq: gerade} berechnet.
Aus der Regressionsrechnung ergeben sich die Folgenden Parameter
\begin{equation*}
  m=\num{1.296\pm0.004} \qquad b=\SI{-14.52\pm0.07}{\centi\meter}.
\end{equation*}
Die Regressionsgerade ist zusätzlich in Abbildung \ref{fig: ioni_fit} dargestellt.
\begin{figure}
  \centering
  \includegraphics[width=0.8 \textwidth]{../Messdaten/gerade_io_final.pdf}
  \caption{Ausgleichsgerade der aus Abbildung \ref{fig: messkurve_ioni} abgelesenen Punkte.}
  \label{fig: ioni_fit}
\end{figure}
Der Nullpunkt der Ausgleichsgeraden entspricht der verschobenen
Ionisiationsspannung:
\begin{equation*}
  U\ua{ion,v}=-\frac{m}{b}=\SI{11.20\pm0.06}{\volt}.
\end{equation*}
Die Umrechnung von $\si{\centi\meter}$ zu $\si{\volt}$ erfolgt mit den Ergebnissen aus
Abschnitt \ref{sec: ausgleich}.
Mit Hilfe der zuvor bestimmten Kontakpotentiale $K$ (vgl. \eqref{eq:k_energie_zim} und \eqref{eq:k_frank})
kann $U\ua{ion}$ bestimmt werden. Es ergibt sich für die Ionisierungsspannung:
\begin{equation}
  \label{eq: u_ion}
  U\ua{ion}= U\ua{ion,v}-\ov{K}=\SI{9.74\pm0.14}{\volt} \quad \text{mit} \quad \ov{K}=\SI{1.46\pm0.13}{\volt}.
\end{equation}
Hierbei ist $\ov{K}$ der Mittelwert aus den zuvor bestimmten Kontaktpotentiale. %potentiale
%Die im Versuchs aufgenommene Messkurve weist kein, wie in der Versuchsdurchführung angedeutet (vgl. Abb. \ref{fig: theo_U_ion}), eindeutiges
%asymptotisches Verhalten auf.
\FloatBarrier
