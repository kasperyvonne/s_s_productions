\section{Diskussion}
Im Folgenden sollen die Messergebnisse in Bezug auf die Messgenauigkeit des %Messergebnisse in Bezug
Vesuchsaufbaus diskutiert werden.

Zu einer signifikaten Ungenauigkeit in dem Versuchsaufbau sorgt der verwendete
X-Y Schreiber, denn ein Teil der fein Justierung war defekt. Dadurch mussten teilweise
ungenauere Skalierungen verwendet werden. Bei nicht vorhanden sein dieses Problems
wäre zum Beispiel, bei der Frank-Hertz-Kurve das erste Maxima deutlicher und
verwendbar.

Neben dem X-Y Schreiber besitz die Spannungsquelle für $U\ua{A}$ und $U\ua{B}$
noch eine wichtige Rolle. Auf Grund von analogen Anzeigen, war das genaue ablesen
der Spannung erschwert und hätte durch eine digitale Vorrichtung verbessert werden
können. Das Problem mit der analoge Anzeige wirkt sich insbesondere bei der Messung
der Energieverteilung negativ aus. Dort war eine Spannung von $U\ua{A}=\SI{11}{\volt}$
verlangt, jedoch reicht die analoge Skala nur bis $\SI{10}{\volt}$, hier durch müssen
die $\SI{11}{\volt}$ abgeschätzt werden.
Hinzu kommt die beschädigte Isolierung des Kabels für das Picoampermeter.
Durch die Beschädigung sorgten schon kleine Magnetfelder z.B. erzeugt von der Heizung
zu einer Strommessung.
Dies könnte auch der Grund für den unphysikalischen Spannugnsverlauf in Abbildung
\ref{fig: messkurve_ioni} sein.

Als letzte entscheidene Fehlerquelle ist die für den Versuch verwendete Heizung
zu erwähnen. Da diese keine Funktion besaß, um die Temerpatur konstant zu halten, war
es bei den Messungen kaum möglich eine gleichbleibende Temperatatur zu gewährleisten.

 \begin{table}
   \centering
   \caption{Vergleich der Messergbnisse mit Literaturwerten}
   \label{tab: results}
   \begin{tabular} {S S[table-format=1.2] @{${}\pm{}$} S[table-format=1.2] S}
     \toprule
     {Messgröße} & \multicolumn{2}{c}{Exp. Wert}& {Literaturwert} \\
     \midrule
     $\text{E}_{\mathrm{anr}} \,/ \, \si{\eV}\, \, \text{\cite{anreg}}$  &  4.65 & 0.12  & 4.9\\
     $\text{U}_{\mathrm{ion}} \, / \si{\volt } \, \, \text{\cite{ioni}}$ & 9.74 & 0.14 &  10.4\\
     \bottomrule
   \end{tabular}
 \end{table}

Beim Vergleich der Messergebnisse mit der Literatur (vgl. Tabelle  \ref{tab: results})
zeigt sich, dass der Versuch trotz der oben aufgeführten Mängel gute Resultate liefert.
