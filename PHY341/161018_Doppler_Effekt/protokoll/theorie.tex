\section{Theorie}
Der Doppler-Effekt tritt immer dann auf, wenn sich Sender und Empfänger einer Welle relativ zueinander bewegen. In dem Versuch 104
soll insbesondere die Verschiebung von akustischen Signalen, sprich die Frequenzänderung von Schallwellen im Medium Luft untersucht
werden. Anders als bei elektromagnetischen Wellen, besteht hier ein Unterschied zwischen den Fällen in denen nur der Sender bzw. nur
der Empfänger in Bewegung ist.

\subsection{Sender in Ruhe, bewegter Empfänger}
In diesem Fall gilt für die empfangene Frequenz $\nu_E$:

\begin{equation}
  \nu_E = \nu_0 (1 + \frac{u}{c})
\end{equation}

Mit der Ruhefrequenz $\nu_0$, der Geschwindigkeit $u$ des Empfängers sowie der Phasengeschwindigkeit $c$ (hier: Schallgeschwidigkeit in Luft bei Raumtemperatur).
Das heißt die Frequenzverschiebung beträgt:

\begin{equation}
  \Delta \nu = \nu_0 \frac{u}{c}
  \label{eq:frequenzverschiebung_bew_empf}
\end{equation}

Entsprechend wird die Frequenz höher, wenn sich E auf S zu bewegt ($u > 0$) resp. tiefer, wenn sich E von S weg bewegt ($u<0$).

\subsection{Empfänger in Ruhe, bewegter Sender}
Hierbei ergibt sich für die empfangene Frequenz $\nu_Q$:
\begin{equation}
  \nu_Q = \nu_0 \cdot \frac{1}{1 - \frac{u}{c}}
  \label{eq:frequenzveschiebung_bew_sen}
\end{equation}
Dieser Ausdruck lässt sich in eine Reihe nach Potenzen von $\frac{u}{c}$ entwickeln:
\begin{align}
  \nu_Q &= \nu_0 \left( 1 + \frac{u}{c} + \left( \frac{u}{c} \right)^2 +  \left( \frac{u}{c} \right)^3 + \dots \right) \\
       &= \nu_E + \nu_0 \left( \left( \frac{u}{c} \right)^2 + \dots \right)
\end{align}
Hieran lässt sich erkennen, dass stets $\nu_Q > \nu_E > \nu_0$ (für $u > 0$) gilt und dasss für den Fall $|u| << c$, $\nu_E \approx \nu_Q$ folgt.

Das Experiment soll die angegebenen Zusammenhänge überprüfen, wozu Messungen der einzelnen Größen durchgeführt wurden.
